%! TEX root = thesis.tex
% vim: ft=tex et sts=2 sw=2

\chapter[Mechanisms and Singularities]{Mechanisms and Singularities\footnote{%
  This chapter is largely based on \crossref{10.1103/PhysRevLett.128.208005}{M.~Mannattil, J.~M.~Schwarz, C.~D.~Santangelo, Phys.\ Rev.\ Lett.\ \textbf{128}, 208005 (2022)}.
  The problem discussed in this paper emerged during conversations with my coauthors.
  I was responsible for all the analytical and numerical calculations, and wrote the paper with inputs from my coauthors.
}}

\chapterprecishere{
  This chapter is an introduction to the basic concepts and mathematics used in this thesis.
  In particular, we discuss the geometrical concepts needed.
}

\section{Introduction}

A \emph{mechanism} can be broadly defined as a mechanical system comprised of rigid parts that move under constraints.%
\footnote{There is some inconsistency in the definition of a mechanism.
  For instance, in most engineering contexts~\cite{hartenberg1964,hunt1978,myszka2012}, a mechanism is considered to be a subelement of a larger machine, or is synonymous with it.
  On the other hand, some authors~\cite{connelly2022} often define a mechanism to be a specific deformation of a mechanical system allowed by its constraints, e.g., a rotor with two degrees of freedom and one constraint is said to possess one mechanism.
  In this thesis, we prefer the engineering definition and a mechanism always refers to a mechanical system or its subelements, and not its individual motions.}
A mechanism could something simple like a linear rotor to something complex like an internal-combustion engine.
A large class of mechanisms are modeled as frameworks comprising of joints connected by rigid bars.

\section{Rigidity theory}

See the notes by \citet{connelly2022} for an introduction to tensegrity structures and the primer by \citet{williams2003} for a slightly advanced treatment.

Forces involved in a state of self stress obey the strong form of Newton's third law and thus cannot possibly result in an unbalanced torque \cite[\S 1.2]{goldstein2002}.
Thus, a tensegrity under self stress is in a state of mechanical equilibrium.
The equilibrium may or may not be stable: again, think of the example with a particle tethered to two walls using springs that are under compression (unstable) or under elongation (stable).

Note that SS exists outside of tensegrity structures.
The only requirement is that all particles interact via central forces.
E.g., one can think of electrostatic analogies, or sticky colloidal clusters.

SS is caused due to linear dependence of constraints.
They might be independent nonlinearly, but on a linear level they are dependent.
Linear constraints are, simply put, hyperplanes.

Maxwell--Calladine theorem is a finite-dimensional toy ``index'' theorem~\cite[\S 2.2]{nakahara2003}.

\subsection{Non-Euclidean origami}

It's easy to understand the configuration manifold of non-Euclidean origami studied by \cite{berry2020}.
A non-Euclidean origami in its ``unfolded'' (or rather, its unactuated) state is not flat.
This means that the linkages forming the origami ``skeleton'' cannot possibly support a self stress since there would be a net unbalanced force in the vertical direction.
Since there can't be a self stress, the Jacobian of the constraint map is full rank everywhere, and thus, the configuration manifold is a smooth 2-dimensional submanifold of $\mathbb{R}^9$ ($M = 7$ and $N = 9$, giving $M - N = 2$).
However, in general, this manifold could be a single connected 2-dimensional manifold or multiple disjoint 2-dimensional manifolds.
If there is positive Gaussian curvature (i.e., an angle deficit) at the center vertex, then the entire structure becomes metastable due to popping up/down of the central vertex.
This popping up and down process is discontinuous and breaks the constraints since the origami has to go through the flat state.
This must be equivalent to having (at least) two 2-dimensional disconnected submanifolds as the configuration space.
When the central vertex has negative Gaussian curvature (i.e., an angle excess), then the origami lacks is metastability.
This must be equivalent to having a single 2-dimensional submanifold as the configuration space.
Note that the manifold can still be disconnected in general, but for the case presenented in \cite{berry2020} it isn't.
What I'm saying is, here there is no reason for the manifold to be connected, but in the previous case (positive Gaussian curvature), it is always disconnected because of physical reasons.
It isn't topology that tells us if the manifolds are disconnected, it's physics.
When there is no Gaussian curvature at the center vertex, the ``unfolded'' state is flat and the origami supports a self stress and the Jacobian drops rank.
This leads to a singularity in the configuration manifold.

\begin{figure}
  \begin{center}
    \includegraphics{zerodof.pdf}
  \end{center}
\caption{A zero \ac{dof} linkage without self stress.  Note how the two constraint manifolds $\mathcal{M}_1$ and $\mathcal{M}_2$ are transverse to each other.}
  \label{fig:hello}
\end{figure}

See the thesis \cite{lengyel2002}.

\begin{figure}
  \begin{center}
    \includegraphics{zerodof_spring.pdf}
  \end{center}
\caption[foo]{A zero \ac{dof} linkage without self stress.  Note how the two constraint manifolds $\mathcal{M}_1$ and $\mathcal{M}_2$ are transverse to each other.}
  \label{fig:hello2}
\end{figure}
\pagebreak

Shape coordinate on a curve can be arclength.  But it's a useless one since it can't be measured experimentally.

Shape space of a triangle is a cone in $\mathbb{R}^{3}$? Joseph Avron's talk at KITP.

Foliation intuitively means that there is a unique hypersurface that passes through each point.

\subsection{Four-bar linkage}

Originally analyzed by Franz Grashof~\cite[pp.~113--118]{grashof1883}.
%
\begin{figure}
  \begin{center}
    \includegraphics{4bar_realworld.pdf}
  \end{center}
\caption{\textbf{(a)} Cyclist \textbf{(b)} locking pliers.
  {\footnotesize [Cyclist photograph by J.~Beau, \href{https://gallica.bnf.fr/ark:/12148/btv1b8433328}{\emph{Photographie Sportive} (1898)}, locking pliers illustration by M.~Eisenberg, \href{https://patents.google.com/patent/US2576286A}{US Patent 2,576,286 (1951)}.]}
}
  \label{fig:4bar_realworld}
\end{figure}

\subsection{Fold angles}

How does one assign signs for fold angles on an origami where the faces are triangles?
Imagine that you are standing with your head pointing in the positive $z$ direction at the corner of the triangle that is opposite to the fold and facing it.
Now, keep your right feet on one of the sides and your left feet on the other.
Now when you look at the fold, if it's a mountain fold, assign it a positive sign, and if it's a valley fold, assign it a negative sign.

\section{Self stress}

\subsection{Effect of pinning the vertices}

The number of self stresses depend strongly on the number of pinned vertices -- a square with an internal vertex, but pinned corners has two states of self stress.  The same square has only one state of self stress if the corners are not pinned.
This can be physical explained on the basis of force balance.

\begin{figure}
  \begin{center}
    \includegraphics{origami2/selfstress.pdf}
  \end{center}
  \caption{
    Two independent states of self stress in an origami with two internal vertices.
    The lengths of the self stress arrows on the $i$th edge is proportional to $\sqrt[4]{\sigma_{i}}$.
  }
  \label{fig:origami2_selfstress}
\end{figure}
%
\begin{figure}
  \begin{center}
    \includegraphics{pinned/pinned.pdf}
  \end{center}
  \caption{
    \textsf{\textbf{(a)}} A quadrilateral framework with one self stress.
    \textsf{\textbf{(b)}} The same framework has two independent self stresses when the corners are pinned.
  }
  \label{fig:quad_pinned}
\end{figure}
