%! TEX root = thesis

\chapter{Mechanisms and Singularities}

\section{Introduction}

A \emph{mechanism} can be broadly defined as a mechanical system comprised of rigid parts that move under constraints.%
\footnote{There is some inconsistency in the definition of a mechanism.
  For instance, in some engineering contexts, a mechanism and machine are synonymous.
  On the other hand, some authors~\cite{connelly2015,rocklin2018} often call the deformation of a mechanical system allowed by its constraints a mechanism.
In this thesis, a mechanism always refers to the mechanical system considered as a whole and not its individual motions.}
A mechanism could something simple like a linear rotor to something complex like an internal-combustion engine.
A large class of mechanisms are modeled as frameworks comprising of joints connected by rigid bars.

\subsection{Four-bar linkage}

Originally analyzed by Franz Grashof~\cite[pp.~113--118]{grashof1883}.

\subsection{Fold angles}

How does one assign signs for fold angles on an origami where the faces are triangles?
Imagine that you are standing with your head pointing in the positive $z$ direction at the corner of the triangle that is opposite to the fold and facing it.
Now, keep your right feet on one of the sides and your left feet on the other.
Now when you look at the fold, if it's a mountain fold, assign it a positive sign, and if it's a valley fold, assign it a negative sign.

\subsection{Geometrical interpretation of self stress}

Suppose one can deform the mechanism while remaining in a state of self stress, then the lengths of the bars would map out a measure zero subset of the codomain.
Assume that this set can be parameterized as an $l$-dimensional surface $\Gamma$ satisfying $g(\ell) = 0$, with $l < m$.
Then during the deformation, $g(f(q)) = 0$.
%
Taking derivatives,
\begin{equation}
  \nabla g (\ell) \nabla f (q) = 0\,,
\end{equation}
%
which shows that the rows of $\nabla g(\ell)$ belong to the left kernel of $\nabla f$.
Since the rows of $\nabla g (\ell)$ span $N_{q}\Gamma$, we get the geometrical interpretation\footnote{This interpretation is due to C.~Santangelo.} that normals to the hyper surface $\Gamma$ are self stresses.
Note that this does not mean that \emph{all} self stresses belong to $N_{q}\Gamma$.
Also, this interpretation is only valid when the mechanism can be deformed while remaining in a singular state.
But in general, there could be isolated states of self stress, just like there are isolated critical points in the case of maps.
For example, consider $f: \mathbb{R}^{2} \to \mathbb{R}^{2}$ defined by $(x, y) \mapsto (1 + x^{2}, 2 + y^{2})$.
The only critical point of this map is $(0, 0)$, corresponding to a critical value $(1, 2)$.
