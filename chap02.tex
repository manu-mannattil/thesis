%! TEX root = thesis.tex
%! TEX root = thesis.tex

\chapter[Mechanisms and Singularities]{Mechanisms and Singularities\footnote{%
  This chapter is largely based on \crossref{10.1103/PhysRevLett.128.208005}{M.~Mannattil, J.~M.~Schwarz, C.~D.~Santangelo, Phys.\ Rev.\ Lett.\ \textbf{128}, 208005 (2022)}.
  The problem discussed in this paper emerged during conversations with my coauthors.
  I was responsible for all the analytical and numerical calculations, and wrote the paper with inputs from my coauthors.
}}

\chapterprecishere{
  This chapter is an introduction to the basic concepts and mathematics used in this thesis.
  In particular, we discuss the geometrical concepts needed.
}

\section{Introduction}

A \emph{mechanism} can be broadly defined as a mechanical system comprised of rigid parts that move under constraints.%
\footnote{There is some inconsistency in the definition of a mechanism.
  For instance, in most engineering contexts~\cite{hartenberg1964,hunt1978,myszka2012}, a mechanism is considered to be a subelement of a larger machine, or is synonymous with it.
  On the other hand, some authors~\cite{connelly2015} often define a mechanism to be a specific deformation of a mechanical system allowed by its constraints, e.g., a rotor with two degrees of freedom and one constraint is said to possess one mechanism.
  In this thesis, we prefer the engineering definition and a mechanism always refers to a mechanical system or its subelements, and not its individual motions.}
A mechanism could something simple like a linear rotor to something complex like an internal-combustion engine.
A large class of mechanisms are modeled as frameworks comprising of joints connected by rigid bars.

\subsection{Four-bar linkage}

Originally analyzed by Franz Grashof~\cite[pp.~113--118]{grashof1883}.
%
\begin{figure}
  \begin{center}
    \includegraphics{4bar_realworld.pdf}
  \end{center}
\caption{\textbf{(a)} Cyclist \textbf{(b)} locking pliers.
  {\footnotesize [Cyclist photograph by J.~Beau, \href{https://gallica.bnf.fr/ark:/12148/btv1b8433328}{\emph{Photographie Sportive} (1898)}, locking pliers illustration by M.~Eisenberg, \href{https://patents.google.com/patent/US2576286A}{US Patent 2,576,286 (1951)}.]}
}
  \label{fig:4bar_realworld}
\end{figure}

\subsection{Fold angles}

How does one assign signs for fold angles on an origami where the faces are triangles?
Imagine that you are standing with your head pointing in the positive $z$ direction at the corner of the triangle that is opposite to the fold and facing it.
Now, keep your right feet on one of the sides and your left feet on the other.
Now when you look at the fold, if it's a mountain fold, assign it a positive sign, and if it's a valley fold, assign it a negative sign.

\section{Self stress}

\subsection{Effect of pinning the vertices}

The number of self stresses depend strongly on the number of pinned vertices -- a square with an internal vertex, but pinned corners has two states of self stress.  The same square has only one state of self stress if the corners are not pinned.
This can be physical explained on the basis of force balance.

\begin{figure}
  \begin{center}
    \includegraphics{origami2/selfstress.pdf}
  \end{center}
  \caption{
    Two independent states of self stress in an origami with two internal vertices.
    The lengths of the self stress arrows on the $i$th edge is proportional to $\sqrt[4]{\sigma_{i}}$.
  }
  \label{fig:origami2_selfstress}
\end{figure}
%
\begin{figure}
  \begin{center}
    \includegraphics{pinned/pinned.pdf}
  \end{center}
  \caption{
    \textsf{\textbf{(a)}} A quadrilateral framework with one self stress.
    \textsf{\textbf{(b)}} The same framework has two independent self stresses when the corners are pinned.
  }
  \label{fig:origami2_selfstress}
\end{figure}


\subsection{Geometrical interpretation of self stress}

Suppose one can deform the mechanism while remaining in a state of self stress, then the lengths of the bars would map out a measure zero subset of the codomain.
Assume that this set can be parameterized as an $l$-dimensional surface $\Gamma$ satisfying $g(\ell) = 0$, with $l < m$.
Then during the deformation, $g(f(q)) = 0$.
%
Taking derivatives,
\begin{equation}
  \nabla g (\ell) \nabla f (q) = 0\,,
\end{equation}
%
which shows that the rows of $\nabla g(\ell)$ belong to the left kernel of $\nabla f$.
Since the rows of $\nabla g (\ell)$ span $N_{q}\Gamma$, we get the geometrical interpretation\footnote{This interpretation is due to C.~Santangelo.} that normals to the hyper surface $\Gamma$ are self stresses.
Note that this does not mean that \emph{all} self stresses belong to $N_{q}\Gamma$.
Also, this interpretation is only valid when the mechanism can be deformed while remaining in a singular state.
But in general, there could be isolated states of self stress, just like there are isolated critical points in the case of maps.
For example, consider $f: \mathbb{R}^{2} \to \mathbb{R}^{2}$ defined by $(x, y) \mapsto (1 + x^{2}, 2 + y^{2})$.
The only critical point of this map is $(0, 0)$, corresponding to a critical value $(1, 2)$.
