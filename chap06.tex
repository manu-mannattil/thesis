%! TEX root = thesis.tex
% vim: ft=tex fdm=manual et sts=2 sw=2

\chapter{Elastic waves}

This chapter is a brief discussion of elastic waves.
In particular, we discuss waves propagating on a rod and shell, both with varying curvatures.
The equations we derive in this chapter would form the basis of our discussions in the next chapter.

\section{Introduction}

The physical assumptions usually made (expressed here in terms of the wave number $k$ and the structure's curvature $m$) while writing down such a set of equations are~\cite{pierce1993,norris1994,kernes2021}:
%
\begin{enumerate}
  \setlength\itemsep{0em}
  \item[(i)] The wavelength ($\sim k^{-1}$) is much larger that the thickness of the structure.
    If we work in length units such that the thickness is of order unity, we must then have $k \ll 1$.
    For bulk waves with wavelengths much smaller than the thickness (i.e., when $k \gg 1$), the structure should be treated as an infinite elastic medium~\cite{landau1986}.
  \item[(ii)] The radius of curvature is much larger than the thickness, so that $\abs{m} \ll 1$ (in length units such that thickness is order unity).
    This is a geometric requirement---a structure whose thickness is larger than the radius of curvature of its mid-surface (or mid-curve) cannot be constructed.
  \item[(iii)] The wavelength is smaller or of the order magnitude as the radius of curvature ($= m^{-1}$), so that $k > m$ is a safe choice (in all length units).
\end{enumerate}

\section{Filament equations}

%
\begin{equation}
  \mathscr{E}[\zeta, u] = \mathscr{K} - \mathscr{U} = \int \dd{t}\,\dd{x}\,\frac{1}{2}\left\{\rho\left(\dot{u}^{2} + \dot{\zeta}^{2}\right) - A\left[u' - m(x)\zeta\right]^{2} - B\left(\zeta''\right)^{2}\right\}
\end{equation}

