%! TEX root = thesis.tex
% vim: ft=tex fdm=manual et sts=2 sw=2

\chapter{Conclusion}

Several remarkable applications of elastic waves from acoustic cloaking to negative refraction rely on subtle aspects of wave localization.
The vast majority of these applications, however, require metamaterials with highly nontrivial microstructure.
Our work, therefore, sets the stage for the design of even simpler devices capable of inducing wave localization.

Ray tracing methods have continued to receive extensive attention in recent years.
An interesting line of work involves the phenomenon of branched flows---the spatial branching of rays as a result of random, but weak inhomogeneities in the medium, the end result being the formation of tree-like structures and fluctuations of extreme intensity~\cite{heller2021}.
Branched flows have been observed in a host of systems, including tsunami waves in the ocean, light propagation in soap films, the flow of electrons in semiconductors, etc.
More recently, branched flows have been shown to exist in thin elastic plates and cylinders with varying thickness profiles~\cite{jose2022,jose2023}.
For obvious reasons, it would be interesting to extend these results to consider shells with random curvature profiles, akin to crumpled paper~\cite{gopinathan2002}.
