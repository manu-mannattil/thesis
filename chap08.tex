\section{Concluding remarks}
\label{sec:conclusion}

In this chapter we have considered the localization of waves in thin elastic structures induced by variations in the structure's curvature profile.
For both the example structures we considered, bound states develop around points where the structure's absolute curvature has a minimum.
In case of the shell, flexural, shear, and extensional waves form bound states.
Additionally, flexural bound states can also develop around points of a shell where the absolute curvature has a maximum.
In contrast to shells, bound states in a curved rod (which are always extensional in nature) only exist around points where the absolute curvature has a minimum.
These findings set the stage for the design of simple devices capable of inducing wave localization without relying on metamaterials with nontrivial microstructure.

Semiclassical approximation presents challenges of its own when used to study multicomponent waves, particularly due to the presence of nontrivial phases in the quantization rule.
The rod and shell equations we use in this paper, however, have properties that cause these phases to vanish.
Nevertheless, topologically protected waves in continuous media can arise when this phase is nonzero, especially when time-reversal symmetry is broken~\cite{venaille2023}.
For this reason, it is worthwhile to explore the use of semiclassical methods in problems with broken time-reversal symmetry such as those in rotating elastic media~\cite{marijanovic2022}, fluids with odd viscosity~\cite{souslov2019}, and magnetoelastic waves~\cite{banos1956}, where one would generically expect this phase to be nonzero.


