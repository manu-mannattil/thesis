%! TEX root = thesis.tex
% vim: ft=tex et sts=2 sw=2

\chapter{Comments and TODO}

\section{Comments on Paper 1}

\begin{enumerate}
  \item In the derivation for regular points, it's not readily clear that the matrix in the exponential would be positive definite.  But this is so since it's the Hessian of a function ($=U(q) + |\hat{\vartheta}(q) - \vartheta|^{2}$) which has a minimum at $q_{i}$.
\end{enumerate}

\section{Comments on Paper 2}

\begin{enumerate}

  \item What's the connection between the Faddeev--Popov method and the coarea formula?  See Zee's book on QFT.
  \item Language: point \emph{in} manifold or point \emph{on} manifold? N.B. a manifold is a set.
  \item I didn't use the constant-rank theorem since that'd require more explanation, e.g., there could be rows of the Jacobian that are \emph{not} independent if the requirement is only constant rank.  Perhaps we can add that preimage theorem, constant-rank theorem, etc., are variations of the implicit function theorem.
  \item I'm being purposefully sloppy in the definition of the tangent space.
    The vectors $\bm{v}$ should be picked from $T_{\bm{q}} \mathcal{Q}$ (in which case I'll need to define that first) and not $\mathbb{R}^n$.
    Here it's okay since $\mathcal{Q}$ is always considered as a submanifold of $\mathbb{R}^n$, which makes $T_{\bm{q}}\mathcal{Q} = \mathbb{R}^n$.
  \item What's the guarantee that a rank deficiency leads to a bifurcation?  Also, rank-deficiency singularities could be parameterization singularities.  I think CS singularities require rank deficiency, but rank deficiency doesn't always guarantee CS singularity.
  \item A zero mode is technically a motion on the tangent bundle $TM$ since you want $q \in \Omega$ and $v \in T_{q}\Omega$.
  \item Even though $q$ and $v$ belong to different spaces we are taking $q \to q + v$.  Although I'm confident that there's nothing wrong with this, I want to understand this better.  Goes back to CM where we take $q \approx q_0 + v t$.
  \item What's the connection between energy near a singularity and the ``affine energy'' in Lubensky et al.'s RPP review (Eq.~3.10)?
  \item Even though zero modes corresponding to flexes don't cost energy, we still have $x\trans\hess f_ix \neq 0$.  This is because flex is ``nonlinear''.  However, the partition function calculation is still fine because the projection operator will kill these nonzero terms.
  \item In our discussion, we have focused on self stress states of points that belong to the constraint manifold.
  However, we should note all points $q \notin \Omega$ for which $C(q)$ drops rank, also admit self stress states.
  Discuss issues, e.g., drop in rank doesn't guarantee singularity, etc.
  \item Can you call an ordinary function positive definite or positive semidefinite?
  \item If $A + B = C$, where $A, B, C$ are vector spaces, does one say that $A$ and $B$ spans $C$ or $A$ and $B$ generates $C$?  Or are both wrong?
  \item Plot $\beta F(\xi)$ instead of saying that you're plotting free-energy in units of $\beta^{-1}$.
  \item Histogram method for free energy is also called ``visited states method''.
  \item What's the guarantee that states of stress (and singularities) are gauge invariant w.r.t. the local body frame of the mechanism?
    Gauge invariance in the sense of Littlejohn's papers on few-body systems.
  \item Adjacency matrix = sign(compatibility matrix)?
  \item To write the partition function as the Laplace's transform of the density of states, shouldn't the energy function $U(\bm{q})$ foliate $\mathbb{R}^{n}$? Coarea formula.

  \item Out of plain buckling of thin plates, Foppel-von Karman limit, stresses.

    Strain tensor (Audoly's book Eq.~6.60) for a displacement field $(u_{1}, u_{2}, f)$ is
    \begin{equation}
      \epsilon_{ij} = \frac{1}{2}\left(\partial_{i} u_{j} + \partial_{j} u_{i} + \partial_{i}f \partial_{j}f\right)
    \end{equation}
    The first two terms is equivalent to $\mathsf{C}\bm{v}$ and the last quadratic term is $\bm{w}(\bm{u})$.
    Mechanical equilibrium requires (Eq.~6.64 of Audoly's book):
    \begin{equation}
      \partial_{i}\partial_{j} f \sigma_{ij} = 0.
    \end{equation}
    Is this equivalent to the Fredholm alternative $\bm{\sigma}\cdot\bm{w}(\bm{u}) = 0$?
    Also in the strain $\partial_{i}f \partial_{j} f$ has 3 independent components, whereas there are only two independent displacement fields, namely, $u_{1}$ and $u_{2}$.  Thus, for a given set of $u$, an arbitrary $f$ will not solve $\epsilon_{ij} = 0$.  Only those satisfying $K = 0$ will work (i.e., isometries).
    How is this related to the Fredholm alternative, and the solvability of the constraint map equation?
\end{enumerate}

\section{Waves}

\begin{enumerate}
  \item Parity of the filament operator.%
    \footnote{%
    More quantum mechanically, this can be shown by considering the commutation of $\hat{\mathsf{D}}$ with the operators $\mathsf{P}_{\pm} = \diag\left(\hat{\pi}, \pm\hat{\pi}\right)$, where $\hat{\pi}$ is the usual parity operator~\cite{cohen-tannoudji2019}.
      Clearly, $\hat{\mathsf{P}}_{\pm}\Psi(x) = \left[\zeta(-x), \pm u(-x)\right]$ so that
      the eigenstates of $\mathsf{P}_{+}$ always have $\zeta(x)$ and $u(x)$ of the same parity, whereas the eigenstates of $\mathsf{P}_{-}$ always have $\zeta(x)$ and $u(x)$ of different parity.
      Furthermore, for odd and even $m(x)$, we can show that $\hat{\mathsf{D}}$ commutes with $\hat{\mathsf{P}}_{+}$ and $\hat{\mathsf{P}}_{-}$, respectively.
      As commuting operators share the same eigenstates (assuming nondegeneracy), this proves the claim made above.%
    }
\end{enumerate}

