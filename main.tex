%! TEX root = thesis.tex
% vim: ft=tex fdm=manual et sts=2 sw=2

\begin{document}

\frontmatter

% Cover ----------------------------------------------------------------

% % Don't include cover in draft or deadtree or SU version.
% \ifcoverpdf
%   % Cover is p. "cover"
%   \def\thepage{cover}
%   \includepdf{cover/cover.pdf}
%   % Switch to roman numerals after cover page.
%   \pagenumbering{roman}
% \fi

% Abstract -------------------------------------------------------------

\chapter*{Abstract}
% \addcontentsline{toc}{chapter}{Abstract}
\thispagestyle{empty}

This dissertation is concerned with two problems that lie at the interface of soft-matter physics, geometry, and asymptotic analysis, but otherwise have no bearing on one another.  In the first problem, I consider the equilibrium thermal fluctuations of deformable mechanical frameworks.  These frameworks have served as highly idealized representations of mechanical structures that underlie a plethora of soft, few-body systems at the submicron scale such as colloidal clusters and DNA origami.  When the holonomic constraints in a framework cease to be linearly independent, singularities can appear in its configuration space, where it becomes energetically softer.  Consequently, the framework's free-energy landscape becomes dominated by the neighborhoods of points corresponding to these singularities.  In the second problem, I study the localization of elastic waves in thin elastic structures with spatially varying curvature profiles, using a curved rod and a uniaxially-curved shell as concrete examples.  Waves propagating on such structures have multiple components owing to the curvature-mediated coupling of the tangential and normal components of the displacement field.  Here, using the semiclassical approximation, I show that these waves form localized, bound states around points where the absolute curvature of the structure has a minimum.  Both these problems exemplify the subtle interplay between the mechanical properties of soft materials and their geometry, which further sets the stage for many interesting consequences.

\ifdeadtree
  \blankpage
\fi

% Title page -----------------------------------------------------------

\newpage\thispagestyle{empty}

\begin{center*}
  \begin{Spacing}{2.0}
  {\LARGE{\widetext{{Asymptotics, Geometry,\\ and Soft Matter}}{{Asymptotics, Geometry, and Soft Matter}}}}\\
  \end{Spacing}
  \vspace{3em}
  \emph{by}\\[3em]
  \widetext{Manu Mannattil}{Manu Mannattil}\\
  \vspace{1em}
  \begin{Spacing}{1.25}
  M.Sc.~(Integrated), Indian Institute of Technology Kanpur, 2015\\[5em]
  Dissertation\\
  Submitted in Partial Fulfillment of the Requirements\\
  for the Degree of\\
  Doctor of Philosophy in Physics\\[5em]
  \end{Spacing}

  \ifsustyle
    \relax
  \else
    \includegraphics[scale=1.25]{misc/sulogo.pdf}\\[1em]
  \fi
  \widetext{Syracuse University}{Syracuse University}\\[1em]
  August 2023
\end{center*}

% Copyright page -------------------------------------------------------

\newpage\thispagestyle{empty}

\begin{center*}
  \begin{Spacing}{1.25}
  Copyright {\copyright} 2023 Manu Mannattil\\
  All rights reserved
  \ifsustyle
    \relax
  \else
    \phantom{}\\[2em]\color{gray}Git commit \href{https://github.com/manu-mannattil/thesis/tree/\gitHash}{\texttt{\gitShortHash}}; \gitCommitterDate
  \fi
  \end{Spacing}
\end{center*}

% Acknowledgments ------------------------------------------------------

\newpage\thispagestyle{empty}
\chapter*{Acknowledgments}
% \addcontentsline{toc}{chapter}{Acknowledgments}

I thank my adviser Chris Santangelo for taking me on as his Ph.D.~student, for many useful conversations, and for giving me ample independence (among other things).
I also thank Jen Schwarz, Joey Paulsen, and Will Wylie for serving on my dissertation committee.
I am also grateful to Chris and Jen for helping me transition to greener pastures in soft-matter physics after an unwise stint doing high-energy theory.
Special thanks are due to my undergraduate adviser Sagar Chakraborty, with whom I did my first serious research work.
Let me also collectively acknowledge all others (too numerous to list individually, yet too generous to forget) who helped me reach this stage of my academic career.
Finally, I would like to thank my parents for their benevolent encouragement over the years.

\ifdeadtree
  \blankpage
\fi

% TOC ------------------------------------------------------------------

\newpage\pagestyle{headings}

% More compact TOC if we're not using SU style.
% https:/tex.stackexchange.com/a/60322
\ifsustyle
  \relax
\else
  \setlength{\cftbeforechapterskip}{0.5em}
\fi

\setcounter{tocdepth}{2}

% Use the starred version of \tableofcontents to avoid a
% self reference.  See Chapter 9 of the memoir
% documentation.
\tableofcontents*

% Notation -------------------------------------------------------------

\chapter*{Notation}
% \addcontentsline{toc}{chapter}{Notation}

%! TEX root = thesis

\chapter*{Notation}

\begin{tabular}{ll}
  $\bm{a}$     & Cartesian vector\\
  $\Abs{\bm{a}}$ & norm $\sqrt{\bm{a}\trans\bm{a}}$ of vector $\bm{a}$\\
  $\mat{A}$ & matrix $\mat{A}$\\
  $\mat{I}_n$ & $n\times n$ identity matrix\\
  $\nabla \phi$ & gradient of $\phi: \mathbb{R}^n \to \mathbb{R}$ considered as a row vector, or\\
                & the $m\times n$ Jacobian matrix of a map $\phi: \mathbb{R}^n \to \mathbb{R}^m$\\
  $(\nabla\phi)\trans$ & transpose gradient of $\phi: \mathbb{R}^n \to \mathbb{R}$ considered as a column vector, or\\
                & the $n\times m$ transpose of the Jacobian matrix of a map $\phi: \mathbb{R}^n \to \mathbb{R}^m$\\
  $\hess \phi$ & $n\times n$ Hessian matrix of $\phi: \mathbb{R}^n \to \mathbb{R}$\\
  $\det\mat{A}$ & determinant of matrix $\mat{A}$\\
\end{tabular}

\section*{Acronyms}

% TODO: Make this a proper aligned table.
\def\aclabelfont#1{\textsf{\textsc{\MakeLowercase{#1}}}}
\begin{acronym}\itemsep0pt
  \acro{cas}[CAS]{computer algebra system(s)}
  \acro{cv}[CV]{collective variable(s)}
  \acro{dof}[DOF]{degree(s) of freedom}
  \acro{pdf}[PDF]{probability density function}
\end{acronym}


% Chapters -------------------------------------------------------------

\mainmatter

\pagestyle{headings}

%! TEX root = thesis.tex
% vim: ft=tex et sts=2 sw=2

\chapter{Introduction}

\chapterprecishere{%
  This dissertation discusses two problems that are more or less unrelated apart from having a common origin in soft-matter physics.
  During the course of the discussion, we shall borrow concepts from a potpourri of fields ranging from classical and quantum mechanics to statistical mechanics to engineering.
  Elementary notions from differential geometry and asymptotic analysis will also play a prominent role.
  This chapter provides a quick overview of the dissertation, highlighting the key results, and concludes with an organizational summary.\\
}

\section{Frameworks, thermal fluctuations, and free energy landscapes}

\textsc{Simply} put, a bar-joint mechanism is a deformable assembly of bars that connect freely rotating joints. Such mechanisms have served as highly idealized representations of mechanical structures that underlie a plethora of soft few-body systems, e.g., colloidal clusters, proteins, viruses, etc. More recently, DNA origami has made it possible to make these mechanisms at the nanoscale, where they undergo low-temperature thermal fluctuations due to the surrounding medium. Although the bars in a mechanism are only stiff and not rigid, it is useful to analyze a mecha- nism in the limit that the bars become fully rigid, i.e., when the bar lengths are fixed. Indeed, such a situation is an instance of a holonomically constrained classical system, often encountered in elementary mechanics.
However, the imposed holonomic constraints need not always be well-behaved.
To illustrate this point more generally, consider a particle constrained to move on two intersecting cylinders of equal radius with mutually perpendicular axes [Fig. 1(a)].
The configuration space of this particle is not a smooth manifold, precisely because the cylinders have a nontransversal intersection at two singular points, where they share a common tangent plane.
Such singularities, which arise when the constraints imposed on a system cease to be linearly independent, are not mere pathological irregularities, and they have been extensively studied in many fields, e.g., robotics and locomotion.

To shed some more light on the above discussion, consider a mechanical system with $n$ \ac{dof}, whose configuration at any given moment is fully described by a single configuration vector $\bm{q} \in \mathbb{R}^{n}$.
Constraints in such a system are most clearly introduced by defining a constraint map $f: \mathbb{R}^{n} \to \mathbb{R}^{m}$
that vanishes when the constraints are satisfied, with $m$ being the number of constraints introduced.
The map $f$ is a general nonlinear map in $\bm{q}$ and its linear approximation is given by the $m\times n$ Jacobian matrix $\nabla f$.
Constraints in the two-cylinder system in Fig.~XXX, for instance is defined by the map $f(x, y, z) = (x^{2} + z^{2} - 1,\, y^{2} + z^{2} - 1)$.
With these definitions, the configuration space of the system is the set $\Sigma = \left\{\bm{q} \in \mathbb{R}^{n} : f(\bm{q}) = \bm{0}\right\}$, which is the set of points where the constraints are satisfied exactly.
Standard theorems in manifold theory%
\footnote{These theorems are almost never explicitly invoked in classical mechanics.
However, they are implicit in the frequently used argument that a mechanical system with $n$ degrees of freedom and $m$ constraints have $(n-m)$ degrees of freedom, with the configuration space $\Sigma$ parameterizable by $(n-m)$ generalized coordinates.}
ensure that $\Sigma$ is a smooth $(n-m)$-dimensional manifold if the Jacobian $\nabla f$ has full rank for all points in $\Sigma$.
At singularities, such as the ones in Fig.~XXX, the Jacobian $\nabla f$ drops rank%
\footnote{Since the Jacobian is an $m\times n$ matrix, it drops rank whenever its rows---each representing a single linearized constraint---cease to become linearly independent.}
and $\Sigma$ fails to be a manifold.

In practice, there is no such thing as a system with perfect constraints, and it is always possible to violate them by paying some sort of an energy cost.
For realistic systems, the set $\Sigma$ would then form the ground-state manifold (assuming that energy costs solely arise due to the constraints being violated).
At the most basic level, we can assume that energy $U$ of a stiffly constrained system depends only the value of the constraint map $f(\bm{q})$, so that we can take $U \equiv U[f(\bm{q})]$ with $U$ having a minimum for all points on $\Sigma$ where $f(\bm{q}) = \bm{0}$.

To see this at an even elementary level, assume that we have just one constraint so that the constraint map $f: \mathbb{R} \to \mathbb{R}$ and the Jacobian $\nabla f$ is just the gradient of $f$.
Taylor expanding the energy to $\mathcal{O}(\Abs{\bm{q}}^{2})$, we get the familiar Harmonic approximation result
%
\begin{equation}
  U[f(\bm{q})] \approx \tfrac{1}{2}U''\left[f(0)\right]\left(\nabla f\cdot \bm{q}\right)^{2}.
\end{equation}
%
Clearly, near points where $\nabla f$ vanishes (which is the equivalent of a rank-deficient Jacobian for a scalar map $f$) the harmonic approximation would give $U \approx 0$.
This shows that in the vicinity of points where $\nabla f$ becomes vanishingly small, we need to expand $U$ beyond the harmonic order for any meaningful description of the energy landscape.
Higher-order corrections to $U$ are usually weaker (after all $\Abs{\bm{q}}$ is assumed to be small), and consequently the system becomes energetically soft.
Although we restricted ourselves to the case with just one constraint here, the situation is analogous when there are more.

\begin{figure}
  \begin{center}
    \includegraphics[scale=1.0]{misc/4bar_collage.pdf}
  \end{center}
\caption{(a) Configuration space of a four-bar linkage visualized as two intersecting curves on a torus. (b) its free energy $\mathscr{A}(\theta_{1})$ as a function of the angle $\theta_{1}$. (c) Four-bar linkage in the real world; photograph by J.~Beau, \emph{Photographie Sportive} (1898).}
  \label{fig:4bar_collage}
\end{figure}


\section{Thin structures, elastic waves, and bound states}

Wave propagation on thin structures.

Musical saw. -- energy leaks through the fingers.

Bound states.

The key to understanding localization of waves and the formation of bound states is the eigenvalue problem
%
\begin{equation}
  \hat{\mathsf{D}}\psi = \omega^{2}\psi
\end{equation}
%
where $\hat{\mathsf{D}}$ is an $n\times n$ operator composed of spatial derivatives (i.e., powers of $\partial_x$) and $\omega$ is the frequency of oscillation.
A plane wave solution of the form $\psi \sim e^{\pm i kx}$ is only applicable if the coefficients of the derivatives are constants.

Semiclassical method.

In asymptotic analysis, it is usually introduced as an approximate method to find solutions to differential equations whose highest-order derivative is multiplied by a small parameter $\epsilon$.

Time-independent Schr\"{o}dinger equation for a particle in a potential $V(x)$
%
\begin{equation}
  \left[-\frac{\hbar^{2}}{2m}\partial_{x}^{2} + V(x)\right]\psi(x) = E\psi(x),
\end{equation}
%
The semiclassical approximation is often introduced as the limit where the (reduced) Planck's constant $\hbar \to 0$.
At first glance, such a limit makes \emph{no} physical sense as $\hbar$ is a fundamental constant, whose value is fixed by the units we choose to work in.
Rather, in considering the limit $\hbar \to 0$, we are considering the limit where the value of $\hbar$ is much smaller compared to the angular momentum scale, which is often the case with macroscopic systems described by classical physics.%
\footnote{%
  This can be more clearly seen by nondimensionalizing the time-independent Schr\"{o}dinger equation by introducing an energy scale $U$ and a length scale $L$. After setting $x \to L{x}$, $E \to U{E}$, and $V(x) \to U{V}(x)$, we find
%
\begin{equation}
  -\epsilon^{2}\partial_{{x}}^{2}\psi(x) + {V}(x)\psi(x) = {E}\psi(x),
\end{equation}
%
where all quantities as well as the parameter $\epsilon = \hbar/\sqrt{2mL^{2}U}$ are now dimensionless.
As $\sqrt{2mL^{2}U}$ has the dimension of the angular momentum, taking the limit $\epsilon \to 0$ is equivalent to saying that typical values of angular momentum is much larger than $\hbar$.}

\begin{figure}
  \begin{center}
    \includegraphics{dna.pdf}
  \end{center}
  \caption{DNA origami has been widely used to self assemble a variety of objects at the nanoscale.
Depicted in the figure are (a) tensegrity structures \cite{liedl2010}; (b), (c) linkage-based mechanisms \cite{marras2015,zhou2015}; (d) a rhombus-shaped nanoactuator~\cite{ke2016}; and (e) self-assembled polyhedra~\cite{iinuma2014}.}
  \label{fig:dna_origami}
\end{figure}

\begin{figure}
  \begin{center}
    \includegraphics{misc/confspace.pdf}
  \end{center}
  \caption{A classical system, composed of many point particles and rigid bodies, can be represented by a single configuration vector $\bm{q}$ of a high-dimensional configuration space $\Sigma$, which may or may not be a smooth manifold. (Inspired by Fig.~20.1 of Ref.~\cite{penrose2004}.)}
  \label{fig:confspace}
\end{figure}

\begin{figure}
  \begin{center}
    \includegraphics{saw/saw.pdf}
  \end{center}
  \caption{%
    An ordinary hand saw, when bent into the shape of the letter $\mathsf{S}$ can be played like a musical instrument using a violin's bow or a mallet.
    A sustained note is produced on bowing or hitting the saw at its inflection point, which is called a sweet spot by musicians.
    Photographs sourced from Ref.~\cite{shankar2022}.
  }
  \label{fig:saw}
\end{figure}

\section{Organizational summary and other comments}

This dissertation is organized as follows:
Finally, in Appendix~\ref{app:math}, we collect some helpful mathematical results that are used throughout the dissertation.

%! TEX root = thesis

\chapter[Mechanisms and Singularities]{Mechanisms and Singularities\footnote{%
  This chapter is largely based on \crossref{10.1103/PhysRevLett.128.208005}{M.~Mannattil, J.~M.~Schwarz, C.~D.~Santangelo, Phys.\ Rev.\ Lett.\ \textbf{128}, 208005 (2022)}.
  The problem discussed in this paper emerged during conversations with my coauthors.
  I was responsible for all the analytical and numerical calculations, and wrote most of the paper.
}}

\chapterprecishere{
  This chapter is an introduction to the basic concepts and mathematics used in this thesis.
  In particular, we discuss the geometrical concepts needed.
}

\section{Introduction}

A \emph{mechanism} can be broadly defined as a mechanical system comprised of rigid parts that move under constraints.%
\footnote{There is some inconsistency in the definition of a mechanism.
  For instance, in most engineering contexts~\cite{hartenberg1964,hunt1978,myszka2012}, a mechanism is considered to be a subelement of a larger machine, or is synonymous with it.
  On the other hand, some authors~\cite{connelly2015} often define a mechanism to be a specific deformation of a mechanical system allowed by its constraints, e.g., a rotor with two degrees of freedom and one constraint is said to possess one mechanism.
  In this thesis, we prefer the engineering definition and a mechanism always refers to a mechanical system or its subelements, and not its individual motions.}
A mechanism could something simple like a linear rotor to something complex like an internal-combustion engine.
A large class of mechanisms are modeled as frameworks comprising of joints connected by rigid bars.

\subsection{Four-bar linkage}

Originally analyzed by Franz Grashof~\cite[pp.~113--118]{grashof1883}.
%
\begin{figure}
  \begin{center}
    \includegraphics{4bar_realworld.pdf}
  \end{center}
\caption{\textbf{(a)} Cyclist \textbf{(b)} locking pliers.
  {\footnotesize [Cyclist photograph by J.~Beau, \href{https://gallica.bnf.fr/ark:/12148/btv1b8433328}{\emph{Photographie Sportive} (1898)}, locking pliers illustration by M.~Eisenberg, \href{https://patents.google.com/patent/US2576286A}{US Patent 2,576,286 (1951)}.]}
}
  \label{fig:4bar_realworld}
\end{figure}

\subsection{Fold angles}

How does one assign signs for fold angles on an origami where the faces are triangles?
Imagine that you are standing with your head pointing in the positive $z$ direction at the corner of the triangle that is opposite to the fold and facing it.
Now, keep your right feet on one of the sides and your left feet on the other.
Now when you look at the fold, if it's a mountain fold, assign it a positive sign, and if it's a valley fold, assign it a negative sign.

\section{Self stress}

\subsection{Effect of pinning the vertices}

The number of self stresses depend strongly on the number of pinned vertices -- a square with an internal vertex, but pinned corners has two states of self stress.  The same square has only one state of self stress if the corners are not pinned.
This can be physical explained on the basis of force balance.

\begin{figure}
  \begin{center}
    \includegraphics{origami2/selfstress.pdf}
  \end{center}
  \caption{
    Two independent states of self stress in an origami with two internal vertices.
    The lengths of the self stress arrows on the $i$th edge is proportional to $\sqrt[4]{\sigma_{i}}$.
  }
  \label{fig:origami2_selfstress}
\end{figure}
%
\begin{figure}
  \begin{center}
    \includegraphics{pinned/pinned.pdf}
  \end{center}
  \caption{
    \textsf{\textbf{(a)}} A quadrilateral framework with one self stress.
    \textsf{\textbf{(b)}} The same framework has two independent self stresses when the corners are pinned.
  }
  \label{fig:origami2_selfstress}
\end{figure}


\subsection{Geometrical interpretation of self stress}

Suppose one can deform the mechanism while remaining in a state of self stress, then the lengths of the bars would map out a measure zero subset of the codomain.
Assume that this set can be parameterized as an $l$-dimensional surface $\Gamma$ satisfying $g(\ell) = 0$, with $l < m$.
Then during the deformation, $g(f(q)) = 0$.
%
Taking derivatives,
\begin{equation}
  \nabla g (\ell) \nabla f (q) = 0\,,
\end{equation}
%
which shows that the rows of $\nabla g(\ell)$ belong to the left kernel of $\nabla f$.
Since the rows of $\nabla g (\ell)$ span $N_{q}\Gamma$, we get the geometrical interpretation\footnote{This interpretation is due to C.~Santangelo.} that normals to the hyper surface $\Gamma$ are self stresses.
Note that this does not mean that \emph{all} self stresses belong to $N_{q}\Gamma$.
Also, this interpretation is only valid when the mechanism can be deformed while remaining in a singular state.
But in general, there could be isolated states of self stress, just like there are isolated critical points in the case of maps.
For example, consider $f: \mathbb{R}^{2} \to \mathbb{R}^{2}$ defined by $(x, y) \mapsto (1 + x^{2}, 2 + y^{2})$.
The only critical point of this map is $(0, 0)$, corresponding to a critical value $(1, 2)$.

%! TEX root = thesis.tex

\chapter{Free-energy landscapes}

\section{Introduction}

Consider a system of $N$ classical particles in $d$ dimensions.  If the position of the $i$th particle is given by the position vector $\bm{r}_{i} \in \mathbb{R}^{d}$, then the entire configuration of the system can be fully described at any moment using a configuration vector $\bm{r} \in \mathscr{R} \subseteq \mathbb{R}^{Nd}$ defined by $\bm{r} = (\bm{r}_{1}, \bm{r}_{2}, \ldots, \bm{r}_{N})$.
  Here $\mathscr{R}$ is the ambient space.%
  \footnote{Many authors~\cite{littlejohn1997,lelievre2010} call $\mathscr{R}$ as the configuration space.  However, when constraints are imposed on the system, the actual configuration space is usually a lower-dimensional subset of $\mathscr{R}$.  To make this distinction, we will call $\mathscr{R}$ as the ambient space.}
Corresponding to $\bm{r}$ we define a momentum vector $\bm{p} = (\bm{p}_{1}, \bm{p}_{2}, \ldots, \bm{p}_{N})$, where $\bm{p}_{i}$ is the momentum of the $i$th particle.
Since there is no apriori reason for the momenta to be bounded, $\bm{p} \in \mathbb{R}^{n}$.
The microscopic state of the system at a given moment is described by the position-momentum pair $(\bm{r}, \bm{p}) \in T^{*}\mathscr{R}$, where $T^{*}\mathscr{R}$ is the cotangent bundle.
The expectation value of a macroscopic observable $\hat{\xi}$ is
%
\begin{equation}
  \braket{\hat{\xi}} = \int_{T^{*}\mathscr{R}} \dd\mu(\bm{r}, \bm{p})\, \hat{\xi}(\bm{r}, \bm{p}).
\end{equation}

\subsection{Marginal probability densities}

The marginal probability density $\mathscr{P}(\xi)$ in terms of $\xi \in \mathbb{R}^{m}$ with $m \geq 1$ defined as%
\footnote{This equation has been called the ``random variable transformation theorem''~\cite{gillespie1983} and can be used to prove a number of useful results in elementary probability theory; also see Ref.~\cite[Section~I.5]{kampen2007}.}
%
\begin{equation}
  \mathscr{P}(\xi) = \int \dd{\bm{r}}\, \delta[\hat{\xi}(\bm{r}) - \xi] \mathscr{P}(\bm{r}).
  \label{c03:eq:probtrans}
\end{equation}
%
To show that the above equation gives the correct marginal density, we first note that the average $\braket{\phi}$ of an observable $\phi \equiv \phi(\xi) = \phi[\hat{\xi}(\bm{r})]$ that only depends on the transformed variable $\xi$ can be computed using either the transformed density $\mathscr{P}(\xi)$ or the original density $\mathscr{P}(\bm{r})$, i.e.,
%
\begin{equation}
  \braket{\phi} = \int \dd{\tilde{\xi}}\, \phi(\tilde{\xi}) \mathscr{P}(\tilde{\xi}) = \int \dd{\bm{r}}\, \phi[\hat{\xi}(\bm{r})] \mathscr{P}(\bm{r}).
\end{equation}
%
Taking $\phi(\tilde{\xi}) = \delta(\tilde{\xi} - \xi)$ yields
%
\begin{equation}
  \int \dd{\tilde{\xi}}\, \delta(\tilde{\xi} - \xi) \mathscr{P}(\tilde{\xi}) = \mathscr{P}(\xi) = \int \dd{\bm{r}}\, \delta[\hat{\xi}(\bm{r}) - \xi] \mathscr{P}(\bm{r}).
\end{equation}

\subsubsection{Free energy of a stiff rotor}

Consider a rotor of unit natural length in two dimensions centered around the origin with a rotationally symmetric potential energy\footnote{We could have also chosen the more conventional energy function $U(x, y) = \kappa(\sqrt{x^2 + y^2} - 1)^2$, but that makes the integrals that follow difficult to evaluate exactly.  One can still asymptotically evaluate it by Taylor expanding $U(x, y)$ around the saddle point $y = \sqrt{1 - x^2}$ to $\mathcal{O}(y^4)$.} $U(x, y) = \kappa(x^2 + y^2 - 1)^2$, with $\kappa$ being a ``spring constant'' to set the units.
This potential energy function achieves its minimum on the unit circle $x^2 + y^2 = 1$, which is our constraint level set $\Omega$.
Choosing the $x$ coordinate of the rotor as our \ac{cv}, we have the \ac{cv} map $\hat{x}: (x,y) \mapsto x$, and the \ac{pdf} of the \ac{cv} is
%
\begin{equation}
  \begin{aligned}
    \mathcal{P}_{\hat{x}}(x) &= \int_{-\infty}^{\infty} \dd{y}\, \exp\left[-\beta\kappa(x^2 + y^2 - 1)^2\right]\\
                             &= (2\beta\kappa)^{-1/4}\exp\left[-\beta\kappa(x^2 - 1)^2\right]\,\int_{0}^{\infty} \dd{t}\,t^{-1/2}\exp\left[-\tfrac{1}{2}t^2 - \sqrt{2\beta\kappa}(x^2 - 1)t\right]\\
                             &= \sqrt{\pi}(2\beta\kappa)^{-1/4}\exp\left[-\tfrac{1}{2}\beta\kappa(x^2 - 1)^2\right]D_{-1/2}\left[\sqrt{2\beta\kappa}(x^2 - 1)\right]\,,
  \end{aligned}
\end{equation}
%
where $D_{-1/2}[\,\cdot\,]$ is the parabolic cylinder function%
\footnote{We can also write $\mathcal{P}_{\hat{x}}(x)$ in terms of the modified Bessel function of the second kind $K_{1/4}(\,\cdot\,)$ after making use of the relation $D_{-1/2}(z) = \sqrt{z/(2\pi)}K_{1/4}(z^2/4)$ \cite[Eq.~12.7.10]{olver2010}.
This is the result that \ac{cas} often produce.} \cite[Eq.~12.5.1]{olver2010}.
Using $\mathcal{P}_{\hat{x}}(x)$ we find the free-energy difference to be
\begin{equation}
\begin{aligned}\label{eq:rotor_exact_a}
  \mathcal{A}_{\hat{x}}(x) - \mathcal{A}_{\hat{x}}(0) &= -\beta^{-1}\log{\mathcal{P}_{\hat{x}}(x)} +\beta^{-1}\log{\mathcal{P}_{\hat{x}}(0)}\\
                                                      &= \tfrac{1}{2}\kappa(x^4 - 2x^2) - \beta^{-1}\log\left\{\frac{D_{-1/2}\left[\sqrt{2\beta\kappa}(x^2-1)\right]}{D_{-1/2}\left(-\sqrt{2\beta\kappa}\right)}\right\}\,.
\end{aligned}
\end{equation}
But what if one evaluates the integral in Eq.~XXX asymptotically for large $\beta$ using Laplace's method?
Two (standard) parameterizations in $x$ that cover the entire circle $S^1$ are
\begin{equation}
  \psi_{\pm}(x) = (x, \pm\sqrt{1 - x^2})\,,
\end{equation}
and the corresponding induced metrics are
\begin{equation}
  |\det g_{\pm}(x)| = (1-x^2)^{-1}\,.
\end{equation}
It is clear that there is a coordinate singularity at $x = \pm 1$ where $|\det g_{\pm}(x)|$ diverges.
Meanwhile, the dynamical matrix is
\begin{equation}
  \hess U(x, \pm\sqrt{1-x^2}) = 8\kappa
  \begin{pmatrix}
    x^2 & \pm x\sqrt{1-x^2}\\
    \pm x\sqrt{1-x^2} & 1-x^2
  \end{pmatrix}\,,
\end{equation}
which has only one nonzero eigenvalue, $8\kappa$.
Making use of Eq.~XXX, we get the asymptotic \ac{pdf} (which is a rescaled $\beta$ PDF for parameters $(1/2, 1/2)$.)
\begin{equation}
  \mathcal{P}_{\hat{x}}(x) = \sqrt{\frac{\pi}{\beta\kappa(1-x^2)}} + \mathcal{O}(\beta^{-3/2})\,,
\end{equation}
which leads to
\begin{equation}\label{eq:rotor_asy_a}
  \mathcal{A}_{\hat{x}}(x) - \mathcal{A}_{\hat{x}}(0) = \beta^{-1}\log\sqrt{1-x^2}\,.
\end{equation}
\begin{figure}
  \begin{center}
    \includegraphics[scale=1.0]{rotor.pdf}
  \end{center}
  \caption{
    Free-energy difference $\mathcal{A}_{\hat{x}}(x) - \mathcal{A}_{\hat{x}}(0)$ of the stiff rotor at inverse temperature $\beta = 100$, from numerical simulations (black solid line), asymptotic expression (Eq.~\ref{eq:rotor_asy_a}; blue dashed line), and exact expression (Eq.~\ref{eq:rotor_exact_a}; red dashed line).
    Note how the asymptotic expression has a logarithmic divergence at $x = \pm 1$.
    Also note that the free-energy minimum is \emph{not} at $x = \pm 1$ as one might guess (or how the asymptotic expression suggests).
  }
  \label{fig:rotor}
\end{figure}

\section{Hard vs. soft constraints}

Trimer discussed in Refs.~\cite{kampen1981,kampen1984} (and also in Refs.~\cite[Section 15.1]{frenkel2001} and \cite{walter2011})

\section{CV under a transformation that is not smooth}

The free energy difference becomes
%
\begin{equation}
  \Delta\free{\xi} = -\beta^{-1}\log\left[\frac{\sqrt{\pi}\left(\abs{\cos\xi'} + \abs{\sin\xi'}\right)}{2 + \sqrt{X}D_{-1/2}(0)\left(1 + Y\right)}\right]
\end{equation}
\begin{equation}
  \begin{aligned}
    \Delta\free{\xi} &= \beta^{-1}\log\left[2 + \sqrt{X}D_{-1/2}(0)\left(1 + Y\right)\right] \\
                     &\qquad -\beta^{-1}\log\Big\{2\abs{\cos\xi'}^{-1} + \sqrt{X}\big[\exp(-X^{2}\xi'^{4})D_{-1/2}(-2X\xi'^{2})\\
                     & \phantom{\qquad-\beta^{-1}\log\Big\{2\abs{\cos\xi'}^{-1} + \sqrt{X}\big[}
                 \quad+Y\exp(-X^{2}Y^{2}\xi'^{4})D_{-1/2}(-2XY\xi'^{2})\big]\Big\}   \end{aligned}
\end{equation}
%
Here $\xi' = \pi\xi/4$.

\subsubsection{Cone-plane intersection}

A singularity is formed at the origin when a cone $z^2 = x^2 + y^2$ intersects with the $yz$ plane.
However, there's no lowering of the free energy since the intersection isn't a nontransversal intersection.
The only plane tangent to the $yz$ plane is the $yz$ plane itself, which is not tangential to the cone at the origin.
One might object that the cone itself ceases to be a manifold at the origin.
However, one can always ``file off'' the tip of the cone and make it infinitesimally smooth like a physicist would do.
This would still produce no lowering of the free energy.

%! TEX root = thesis.tex
% vim: ft=tex et sts=2 sw=2

\chapter{Semiclassical physics}
\label{chap04}

\chapterprecishere{%
In this chapter we present a quick rundown of the semiclassical approximation as applied to multicomponent waves.
Using a variational approach, we will derive the ray equations and describe the modifications necessary to extract the bound-state spectrum of a multicomponent wave operator through quantization.
The general theory prescribed in this chapter will form a basis for our study of elastic waves in the next chapter.
\\
}

The study of propagating waves is important to many disciplines across all branches of science and engineering.
A widely employed asymptotic method to solve wave problems is the WKB/semiclassical approximation, which is sometimes also referred to as the eikonal approximation, geometrical optics, etc.
Although the WKB approximation is discussed in almost all books on quantum mechanics, there are several subtle issues, especially when multicomponent waves are involved, that need to be addressed.
Our primary goal in this chapter, therefore, is to review some existing results on the semiclassical approximation as applied to multicomponent wave fields.
In particular, we will derive expressions for two extra phases that appear in the Bohr--Sommerfeld quantization rule for such wave fields.
To keep the exposition simple, we will restrict ourselves to wave equations in one variable.
For more detailed descriptions, we refer to the book by \citet{tracy2014} and references therein.

\section{Introduction}
\label{sec:varintro}

We begin by considering a general wave equation of the form
%
\begin{equation}
  \partial_{t}^{2}\Psi(x,t) + \widehat{\mathsf{H}}\Psi(x,t) = 0,
  \label{eq:full_wave_eq}
\end{equation}
%
where $\Psi(x,t)$ is an $N$-component wave field described by a one-dimensional coordinate $x$ and time $t$.
As as example, in elastodynamics, $\Psi$ is usually composed of displacements from the neutral, undeformed state of some elastic structure, e.g., a rod or a shell.
Also, $\widehat{\mathsf{H}}$ is taken to be a Hermitian operator in the form of an $N\times N$ matrix, composed solely of spatial derivatives (i.e., powers of $\partial_{x}$) with time-independent coefficients.
Assuming that the waves are time harmonic with frequency $\omega$, i.e., $\Psi(x, t) = \psi(x)e^{\pm i\omega t}$, where $\psi(x)$ is the time-independent part of the wave field, Eq.~\eqref{eq:full_wave_eq} can be recast as
%
\begin{equation}
  \widehat{\mathsf{D}}\psi = 0,\quad \text{with}\quad \widehat{\mathsf{D}} = \widehat{\mathsf{H}} - \omega^{2}\mathsf{I}_{N},
  \label{eq:ev_problem}
\end{equation}
%
where $\mathsf{I}_{N}$ is the $N\times N$ identity matrix.

If the coefficients of the spatial derivatives that appear in $\widehat{\mathsf{D}}$ are constants, then the eigenmodes $\psi$ are plain waves.
In what follows we assume that these coefficients are slowly varying, with the variation controlled by a single positive parameter $\epsilon \ll 1$, called the \emph{eikonal parameter}.
It is useful to treat $\epsilon$ as an ordering parameter so that we can look for solutions at various orders of $\epsilon$.
To this end, we rescale $x \to \epsilon^{-1}x$ so that a derivative $\partial_{x}$ becomes $\epsilon \partial x$.
With analogy to quantum mechanics, this allows us to recast the derivatives in $\widehat{\mathsf{D}}$ in terms of the wave number/momentum operator $\hat{k} = -i\epsilon \partial_{x}$, with $\epsilon$ playing the role of Planck's constant.
Since we shall be considering $\widehat{\mathsf{D}}$ in the coordinate representation, the position operator $\hat{x} = x$.

\subsection{Wave action}

Central to the variational approach we will use in this chapter is the realization that
a wave equation like Eq.~\eqref{eq:ev_problem} can be derived from a wave action of the form
%
\begin{equation}
  \begin{aligned}
    \mathscr{U}\left[\psi^{*}, \psi\right] &= \tfrac{1}{2}\bra{\psi}\widehat{\mathsf{D}}\ket{\psi}
  = \tfrac{1}{2}\int \dd{x}\,\dd{x'}\, \braket{\psi|x}\bra{x}\widehat{\mathsf{D}}\ket{x'}\braket{x'|\psi}\\
                                           &= \tfrac{1}{2}\int \dd{x}\dd{x'}\, \psi^{*}_{\mu}(x)\,\widehat{\mathsf{D}}_{\mu\nu}(x, x')\,\psi_{\nu}(x').
  \label{eq:wave_action}
  \end{aligned}
\end{equation}
%
Above we have inserted the resolution of identity $\int \dd{x}\,\ket{x}\bra{x} = 1$ in appropriate places to express $\mathscr{U}$ in terms of $\psi$ and its conjugate $\psi^{*}$.
Also, in the last step, we have made the matrix product between the matrix operator $\widehat{\mathsf{D}}$ and the wave vector $\psi$ explicit, with $\braket{x|\widehat{\mathsf{D}}_{\mu\nu}|x'} = \widehat{\mathsf{D}}_{\mu\nu}(x, x')$ being the matrix element of $\widehat{\mathsf{D}}_{\mu\nu}$ in the position basis.
Varying $\mathscr{U}[\psi^{*}, \psi]$ with respect to $\psi^{*}$ gives the eigenvalue problem, Eq.~\eqref{eq:ev_problem}.
For later use, we also note that $\mathscr{U}$ can also be written as
%
\begin{equation}
  \begin{aligned}
    \mathscr{U}\left[\psi^{*}, \psi\right] &= \tfrac{1}{2}\int \dd{x}' \braket{\psi_{\mu}|x'}\bra{x'}\widehat{\mathsf{D}}_{\mu\nu}\ket{\psi_{\nu}}
= \tfrac{1}{2}\int \dd{x}' \bra{x'}\widehat{\mathsf{D}}_{\mu\nu}\ket{\psi_{\nu}} \braket{\psi_{\mu}|x'}\\
&= \tfrac{1}{2}\int \dd{x}' \bra{x'}\widehat{\mathsf{D}}_{\mu\nu}\widehat{\mathsf{W}}_{\nu\mu}\ket{x'} = \tfrac{1}{2}\tr\left(\widehat{\mathsf{D}}_{\mu\nu}\widehat{\mathsf{W}}_{\nu\mu}\right),
  \label{eq:wave_action_trace_form}
  \end{aligned}
\end{equation}
%
where the matrix operator
%
\begin{equation}
  \widehat{\mathsf{W}}_{\nu\mu} = \ket{\psi_{\nu}}\bra{\psi_{\mu}},
\end{equation}
%
is the density operator.

In order to solve Eq.~\eqref{eq:ev_problem} at various orders of $\epsilon$, it is convenient to make use of Weyl calculus, which allows one to map differential operators that are functions of $\hat{x}$ and $\hat{k}$ to ordinary functions, called Weyl symbols, defined on an $x$-$k$ phase space, and vice versa~\cite{chaichian2001,cohen2012}.
For the purposes of this chapter, the following simple rules suffice to convert operators to symbols:
%
\begin{equation}
  f(x) \to f(x),\enspace
  g(\hat{k}) \to g(k),\enspace\text{and}\enspace
  f(x)g(\hat{k}) \to f(x)g(k) + \frac{i\epsilon}{2}f'(x)g'(k) + \mathcal{O}(\epsilon^{2}).
  \label{eq:weylrules}
\end{equation}
%
Above, $f$ and $g$ are functions of $x$ and $\hat{k}$, with the primes denoting derivatives.
Converting each entry of the matrix operator $\widehat{\mathsf{D}}$ into a Weyl symbol, we get the $N\times N$ dispersion matrix $\mathsf{D}$, which we express in various orders of $\epsilon$ as $\mathsf{D} = \mathsf{D}^{(0)} + \epsilon\mathsf{D}^{(1)} + \mathcal{O}(\epsilon^{2})$.

The Wigner tensor $\mathsf{W}_{\nu\mu}$ is defined as the symbol of the density operator $\widehat{\mathsf{W}} = \ket{\psi}\bra{\psi}$ with kernel $\widehat{\mathsf{W}}_{\nu\mu}(x',x) = \psi_{\nu}(x')\psi_{\mu}^{*}(x)$, and is given by
%
\begin{equation}
  \mathsf{W}_{\nu\mu}(x, k) = \int \dd{s}\, e^{-iks/\epsilon}\, \psi_{\nu}\left(x + \tfrac{1}{2}s\right)\psi^{*}_{\mu}\left(x - \tfrac{1}{2}s\right).
\end{equation}
%
By inverting the above expression, and setting $x + \tfrac{1}{2}s \to x'$ and $x- \tfrac{1}{2}s \to x$, we can write the kernel of the density operator in terms of its Weyl symbol as
%
%\begin{equation}
%  \psi_{k}\left(x + \tfrac{1}{2}s\right) \psi^{*}_{j}\left(x - \tfrac{1}{2}s\right) = \frac{1}{2\pi \epsilon} \int \dd{k}\, e^{iks/\epsilon}\,\mathsf{W}_{\nu\mu}(x, k).
%\end{equation}
%%
%so that
%
\begin{equation}
  \psi_{\nu}(x') \psi^{*}_{\mu}(x) = \frac{1}{2\pi \epsilon} \int \dd{k}\, e^{ik(x' -x)/\epsilon}\,\mathsf{W}_{\nu\mu}\left[\tfrac{1}{2}(x' + x), k\right].
  \label{eq:density_operator}
\end{equation}
%
In a similar fashion, we find
%
\begin{equation}
  \widehat{\mathsf{D}}_{\mu\nu}(x, x') = \frac{1}{2\pi\epsilon} \int \dd{l}\, e^{il(x -x')/\epsilon}\,\mathsf{D}_{\mu\nu}\left[\tfrac{1}{2}(x + x'), l\right].
  \label{eq:Dhat_integral}
\end{equation}
%
Putting Eq.~\eqref{eq:Dhat_integral} and \eqref{eq:density_operator} in Eq.~\eqref{eq:wave_action}, we arrive at
%
\begin{equation}
  \mathscr{U} = \frac{1}{2(2\pi\epsilon)^{2}}\int \dd{x}\,\dd{x'}\,\dd{k}\,\dd{l}\, e^{i(l-k)(x -x')/\epsilon}\,\mathsf{D}_{\mu\nu}\left[\tfrac{1}{2}(x + x'), l\right]\, \mathsf{W}_{\nu\mu}\left[\tfrac{1}{2}(x' + x), k\right].
\end{equation}
%
Performing the change of variables $x \to \tfrac{1}{2}(x + x')$ and $x' \to x - x'$, which carries a Jacobian factor of unity, we arrive at
%
\begin{equation}
  \begin{aligned}
    \mathscr{U} &= \frac{1}{2(2\pi\epsilon)^{2}}\int \dd{x}\,\dd{x'}\,\dd{k}\,\dd{l}\, e^{i(l-k)x'/\epsilon}\,\mathsf{D}_{\mu\nu}(x, l)\,\mathsf{W}_{\nu\mu}(x, k)\\
                &= \frac{1}{4\pi\epsilon}\int \dd{x}\,\dd{k}\,\mathsf{D}_{\mu\nu}(x, k)\,\mathsf{W}_{\nu\mu}(x,k)\\
  \end{aligned}
  \label{eq:wave_action_symbol_form}
\end{equation}

So far there has been no approximations involved and the wave action in Eq.~\eqref{eq:wave_action_symbol_form} is exact to all orders of the eikonal parameter.
That said, the curious reader may wonder there is an inconsistency here, in light of Eq.~\eqref{eq:wave_action_trace_form}, where we have expressed $\mathscr{U}$ as the trace of the (scalar) operator $\widehat{O} = \widehat{\mathsf{D}}_{\mu\nu}\widehat{\mathsf{W}}_{\nu\mu}$.
In terms of its symbol, the trace of an operator $\widehat{O}$ is
%
\begin{equation}
  \tr\widehat{O} = \frac{1}{2\pi\epsilon} \int \dd{x}\,\dd{k}\, O(x, k),
\end{equation}
%
so that $\mathscr{U}$ must be
%
\begin{equation}
  \mathscr{U} = \tr{\widehat{O}} = \frac{1}{4\pi\epsilon}\int \dd{x}\,\dd{k}\, \mathsf{D}_{\mu\nu}(x,k)e^{i\widehat{\mathcal{L}}}\mathsf{W}_{\nu\mu}(x, k),
\end{equation}
%
where we have used the Moyal formula to write the symbol $O$ in terms of the symbols of $\mathsf{D}_{\mu\nu}$ and $\mathsf{W}_{\nu\mu}$.
The issue here is a bit subtle---the resolution is that each correction term in the Moyal series can be expressed as divergence of a vector field in the $(x, k)$ phase space~\cite[Problem 3.16]{tracy2014}.
Hence, using the divergence theorem, the integrals over the correction terms vanish for all well-behaved wave fields.
Thus, only the first term in the Moyal series remains, and we get Eq.~\eqref{eq:wave_action_symbol_form} again.

Returning to our main problem, i.e., to employ the semiclassical approximation to solve Eq.~\eqref{eq:ev_problem}, we will proceed as follows: (i) insert the eikonal ansatz into the wave action; (ii) expand the action to the lowest order in the eikonal parameter and form the \emph{reduced action} $\mathscr{U}_{\text{R}}$; (iii) extract the semiclassical equations of motions by varying the reduced action.

\subsubsection*{Reduced wave action}
\label{page:redaction}

As we have assumed that the coefficients of the differential operators in $\widehat{\mathsf{D}}$ are slowly varying, it makes sense to look for an almost plain-wave-like solution of the form
$\psi(x) = A(x)e^{iS(x)/\epsilon}$, where the amplitude $A(x)$ is an $N$-component spinor with complex components, and $S(x)/\epsilon$ is a rapidly varying phase, playing the role of an action.
This is the \emph{eikonal} ansatz, and the parameter $\epsilon$ that controls the slowness of the variation is the eikonal parameter.
To employ the eikonal ansatz, we set $\psi_{\mu} = A_{\mu}e^{iS(x)/\epsilon}$ in Eq.~\eqref{eq:wave_action_symbol_form} to obtain%
%
\begin{equation}
    \mathsf{W}_{\nu\mu}(x, k) = \int \dd{r}\,e^{-ikr/\epsilon} A_{\nu}\left(x + \tfrac{1}{2} r\right)A_{\mu}^{*}\left(x - \tfrac{1}{2} r\right)\exp\left\{\frac{i}{\epsilon}\left[S\left(x + \tfrac{1}{2} r\right) - S\left(x - \tfrac{1}{2} r\right)\right]\right\}.
\end{equation}
%
Next, we set $r \to \epsilon r$ and expand the amplitude $A_{\nu}(x + \tfrac{1}{2}\epsilon r) = A_{\nu}(x) + \tfrac{1}{2}(\partial_{x}A_{\nu})\epsilon r + \mathcal{O}(\epsilon^{2})$ and the phase $S(x + \tfrac{1}{2}\epsilon r) = S(x) + \tfrac{1}{2}[\partial_{x}S(x)]\epsilon r + \mathcal{O}(\epsilon^{2})$ to obtain
%
\begin{equation}
  \begin{aligned}
    \mathsf{W}_{\nu\mu}(x, k) &= \epsilon\int \dd{r}\,e^{-ikr} A_{\nu}(x)A_{\mu}^{*}(x)\exp\left\{ir\left[\partial_{x}S(x) - k\right]\right\} + \mathcal{O}(\epsilon^{2}) \\
                        &= 2\pi\epsilon A_{\nu}(x)A^{*}_{\mu}(x)\delta\left[k - \partial_{x}S(x)\right] + \mathcal{O}(\epsilon^{2}).
  \end{aligned}
\end{equation}
%
We assume that the dispersion matrix is ordered in the eikonal parameter $\epsilon$ so that we can write
$\mathsf{D} = \mathsf{D}^{(0)} + \epsilon \mathsf{D}^{(1)} + \mathcal{O}(\epsilon^{2})$.
Hence, after putting the ``reduced'' Wigner tensor in Eq.~\eqref{eq:wave_action_symbol_form}, we find the action to the lowest order as
%
\begin{equation}
  \mathscr{U} = \tfrac{1}{2}\int \dd{x}\, \mathsf{D}^{(0)}_{\mu\nu}\left[x, k=\partial_{x}S(x)\right]A_{\nu}(x)A^{*}_{\mu}(x) + \mathcal{O}(\epsilon^{2}).
  \label{eq:wave_action_reduced_1}
\end{equation}
%
Next, we perform a spectral decomposition of the lowest-order dispersion matrix $\mathsf{D}^{(0)}_{\mu\nu}(x, k)$ and write it in terms of its eigenvectors $\tau_{a}$ and eigenvalues $\lambda_{a}$ as%
\footnote{In Eq.~\eqref{eq:Dmat_polarization}, the subscripts $\mu$ and $\nu$ indicate the components of $\tau_{a}$.
Also, we have made the summation over the polarization index $a$ explicit---a convention that we will use throughout this dissertation.}
%
\begin{equation}
  \mathsf{D}^{(0)}_{\mu\nu}(x, k) = \sum_{a} \lambda_{a}(x, k) \tau_{a, \mu}(x, k) \tau_{a, \nu}^{*}(x, k).
  \label{eq:Dmat_polarization}
\end{equation}
%
An eigenvector $\tau_{a}$ describes a specific wave type or ``polarization'' represented by the wave equation, Eq.~\eqref{eq:ev_problem}.
And by polarization, we mean a linear subspace of the total wave field that is usually of a distinct physical nature, e.g., flexural waves on a rod.

We put Eq.~\eqref{eq:Dmat_polarization} into Eq.~\eqref{eq:wave_action_reduced_1}, and define $B_{a} = \tau_{a, \nu}^{*}A_{\nu} = \braket{\tau_{a}|A}$, which is the projection of the amplitude $A_{\nu}$ along the direction of the polarization vector $\tau_{a}$.
Although both $A_{\nu}$ and $\tau_{a}$ are general complex vectors, we can take the scalar projection $B_{a}$ to be positive by absorbing the complex phase of $A_{\nu}$ into $\tau_{a}$.
This results in a reduced action of the form
%
\begin{equation}
  \mathscr{U}_{\text{R}}\left[B_{a}, S\right] = \tfrac{1}{2} \sum_{a} \int \dd{x}\, \lambda_{a}(x,k) B_{a}^{2}(x, k),
  \quad\text{with}\quad
  k = \partial_{x}S(x).
  \label{eq:reduced_action}
\end{equation}

\subsection{Phase-space representation of waves}

Variation of the reduced action, Eq.~\eqref{eq:reduced_action}, with respect to the amplitude $B_{a}$ yields
%
\begin{equation}
  \lambda_{a}[x, k(x)]\,B_{a}[x, k(x)] = 0
  \quad\text{for all $a$}.
\end{equation}
%
Assume that the amplitudes $B_{a}$ is nonzero only for one polarization, say, $a = I$, which means that the corresponding eigenvalue must vanish, yielding
%
\begin{equation}
  \lambda_{I}\left[x, k(x)\right] = \lambda_{I}\left[x, \partial_{x}S(x)\right] = 0.
  \label{eq:eikonal}
\end{equation}
%
The above equation, which is known as the eikonal equation, is nothing but the Hamilton--Jacobi equation with $S(x)$ playing the role of the classical action.
 This leads us to the phase-space representation of waves as rays that satisfy the Hamilton's equations
%
\begin{equation}
  \frac{\dd{x}}{\dd{\sigma}} = \partial_{k}\lambda_{I}(x, k) = \left\{x, \lambda_{I}\right\}
  \quad\text{and}\quad
  \frac{\dd{k}}{\dd{\sigma}} = -\partial_{x}\lambda_{I}(x, k) = \left\{k, \lambda_{I}\right\},
\end{equation}
%
where $\sigma$ is some parameter that parameterizes the rays $[x(\sigma), k(\sigma)]$ in the phase space, and $\left\{\cdot, \cdot\right\}$ is the $x$-$k$ Poisson bracket.
Since the waves we consider propagate in a one-dimensional space, these rays are identical to the level curve defined by $\lambda_{I}(x, k) = 0$.

In higher dimensions, solving the Hamilton-Jacobi equation can be nontrivial.
However, in one dimension, it is always theoretically possible to solve for $k(x)$ from $\lambda_{I}(x, k) = 0$.
% TODO: Tori... quantization, more stuff about Hamilton-Jacobi equations.
Variation of the action, Eq.~\eqref{eq:reduced_action}, with respect to the phase $S(x)$ yields the amplitude transport equation%
\footnote{Here we have ignored the term $\partial_{k}{B_{I}^{2}(x,k)}\lambda_{I}$, which trivially vanishes as $\lambda_{I}(x, k) = 0$ on an orbit.
Also, since $k$ is a function of $x$ on an orbit, we can simply write the projected amplitude as $B(x)$ instead of $B(x,k)$.
Note also that after evaluating the partial derivative of $\lambda_{I}$ with respect to $k$, we need to set $k \to k(x)$ for the full derivative with respect to $x$ to make sense.}
%
\begin{equation}
  \frac{\dd}{\dd{x}}\left[B^{2}_{I}(x)\partial_{k}\lambda^{(0)}_{I}(x, k)\right] = 0
\end{equation}
%
so that
$B^{2}(x,k)\partial_{k}\lambda^{(0)}_{I}(x, k) = C$ (constant) and
%
\begin{equation}
  B_{I}(x) = \frac{C}{\sqrt{\partial_{k}\lambda_{I}(x, k)}}.
\end{equation}
%
Clearly, the amplitude projection $B_{I}$ becomes singular at points on the phase space where $\dot{x} = \partial_{k}\lambda_{I}(x, k) = 0$, i.e., when the local tangent along a ray points along the $k$ axis.
Such points are called caustics, and they arise because of ill-defined projections from the ray, which lives in the $x$-$k$ phase space, to the coordinate space, i.e., $x$ space.
The issue can be sidestepped using the Keller--Maslov method~\cite{keller1958,maslov1981} where one Fourier transforms the wave field and uses the eikonal ansatz in the momentum space.
Although we shall later rely on some results of this procedure, we will not discuss its technical details here.
Interested readers are encouraged to look at the book by \citet[Chapter 6]{tracy2014} (and references therein) for more information.

In writing down the eikonal equation, Eq.~\eqref{eq:eikonal}, we have assumed that the amplitude projection $B_{a} \neq 0$ only for $a = I$.
Now, as $\mathsf{D}^{(0)}$ is a Hermitian matrix, its eigenvectors $\tau_{a}$ are mutually orthogonal, and the wave amplitude $A_{\mu} = \sum_{a} B_{a} \tau_{a,\mu} = B_{I}\tau_{I,\mu}$.
This would clearly be wrong if the matrix $\mathsf{D}^{(0)}$ is degenerate.
For this reason, usual semiclassical solutions also break down near points in the phase-space where more than one eigenvalue of $\mathsf{D}^{(0)}$ simultaneously vanish, leading to a exchange of energy and mode conversion between waves of different polarizations~\cite{tracy2014}.
As the wave problems we study in this dissertation do not suffer from mode-conversion issues (at least in the parameter ranges we are interested in), we forgo a more detailed discussion on these issues here.

Assuming that we are far away from caustics and ignoring mode-conversion effects, we can write the final eikonal solution as
%
\begin{equation}
  \psi_{\mu}(x) \sim B_{I}[x, k(x)]\,\tau_{I,\mu}[x, k(x)]e^{iS(x)/\epsilon}.
  \label{eq:psi_wkb_wrong}
\end{equation}
%
At this juncture, we must discuss a subtle problem.
The polarization vector $\tau_{I}$ is only defined up to an overall phase.
Thus, instead of the above equation, it is more appropriate to write
%
\begin{equation}
  \psi_{\mu}(x) \sim B_{I}[x, k(x)]\,\tau_{I,\mu}e^{i\gamma[x, k(x)]}e^{iS(x)/\epsilon},
\end{equation}
%
where $\gamma[x, k(x)]$ stands for an adiabatic phase that we have not determined so far.
As $\tau_{I}$ is obtained from $\mathsf{D}^{(0)}$, whose entries are slowly varying functions of $(x, k)$, it makes sense to consider the phase $\gamma[x, k(x)]$ to be slowly varying as well.
For this reason, $\gamma$ cannot be absorbed into the eikonal phase $S(x)/\epsilon$, which is rapidly varying.

The phase $\gamma$ becomes important when quantizing orbits---a step that boils down to the single valuedness of $\psi_{\mu}$ as we move around a closed orbit, i.e., the sum of $\gamma$ and $S/\epsilon$ must be a half-integral multiple of $2\pi$ (see Section~\ref{sec:bound}).
Unfortunately, our lowest-order analysis is insufficient to determine this phase.
We cannot also find an evolution equation for this phase from the reduced action, Eq.~\eqref{eq:reduced_action}, as it is independent of $\gamma$.
It can also be easily checked that the phase $\gamma$ cannot be introduced in the action by rephasing $\tau$ by the phase factor $e^{i\gamma(x,k)}$, as the action remains invariant.
The message is clear: to find $\gamma$, we need to include higher-order corrections in the wave action.
To this end, rather than repeat the original, but cumbersome derivations scattered across Refs.~\cite{bernstein1975,berk1980,yabana1986,kaufman1987}, it is more sagacious to switch gears and follow a different approach due to Littlejohn and coworkers~\cite{littlejohn1991,littlejohn1991a} for (approximately) diagonalizing multicomponent differential operators.

%Employing the eikonal ansatz, at $\mathcal{O}(\epsilon^{0})$, we find the matrix equation $\mathsf{D}^{(0)}A = 0$.
%To satisfy this equation, at least one of the $N$ eigenvalues of $\mathsf{D}^{(0)}$, say $\lambda(x, k;\, \omega)$, must vanish so that $\det \mathsf{D}^{(0)}(x, k; \omega) = 0$.
%A vanishing eigenvalue $\lambda$ and the associated normalized eigenvector $\tau$ describes different wave types or ``polarizations'' represented by Eq.~\eqref{eq:ev_problem}.
%By a polarization we mean a linear subspace of the total wave field that is usually of a distinct physical nature, e.g., flexural waves on a curved rod.
%Semiclassical approximation breaks down near points where more than one eigenvalues of $\mathsf{D}^{(0)}$ simultaneously vanish, precipitating an exchange of energy and mode conversion between waves of different polarizations~\cite{tracy2014}.

%In the absence of mode conversion, the vanishing eigenvalues of $\mathsf{D}^{(0)}$ also serve as the \emph{ray Hamiltonian} of waves of a specific polarization.
% This leads us to the phase-space representation of waves as rays that satisfy the Hamilton's equations
%%
%\begin{equation}
%  \dot{x} = \partial_{k} \lambda(x, k;\, \omega) = \left\{x, \lambda\right\}
%  \quad\text{and}\quad
%  \dot{k} = -\partial_{x} \lambda(x, k;\, \omega) = \left\{k, \lambda\right\},
%\end{equation}
%%
%where the overdot denotes derivatives with respect to a parameter that parameterizes the rays and $\left\{\cdot, \cdot\right\}$ is the $x$-$k$ Poisson bracket.
%Since the waves we consider propagate in a one-dimensional space, these rays are identical to the level curve defined by $\lambda(x, k; \omega) = 0$.
%%
% \begin{figure}
%   \begin{center}
%     \includegraphics{localization/caustic.pdf}
%   \end{center}
%   \caption{%
%     (a) In phase space, bound states are represented by rays in the form of closed orbits, which is analogous to that of a bound particle oscillating between two classical turning points ($\pm x^{\star}$ in the cartoon).
%     Other trajectories represent unbound states.
%     (b) A bound state represented by a ``peanut''-shaped orbit has six caustics.%
%   }
%   \label{fig:caustic}
% \end{figure}

%%
%\begin{equation}
%\gamma_{\text{G}} = \oint \dd{\sigma}\, \dot{\gamma}_{\text{G}} = i\oint \dd{\sigma}\,\left\langle{\tau_{a}}\middle|\dot{\tau}_{\alpha}\right\rangle =
%  i\oint \dd\xi\cdot\left\langle{\tau_{a}}\middle|\nabla_{\xi}\tau_{\alpha}\right\rangle,
%\end{equation}
%%
%where $\xi = (x, k)$ denotes the ``parameters'' that are being varied.

%%In Appendix~\ref{app:additional_phase} we prove that $\gamma_{\text{G}}$ vanishes if the relative phases between the components of the eigenvector $\tau_{a}$ are constants.
%%The second (non-geometric) phase $\gamma_{\text{NG}}$ need not vanish in such a situation, however.
%Instead of explicitly accounting for the extra phase $\gamma$ in the quantization rule, we could have diagonalized the wave equation at various orders of $\epsilon$~\cite{littlejohn1991,littlejohn1991a,weigert1993,venaille2023}.
%During such a procedure, terms proportional to $\dot{\gamma}_{\text{G}}$ and $\dot{\gamma}_{\text{NG}}$ naturally appear in the ray Hamiltonian $\lambda$ as a first-order correction.
%Despite the elegance of the method, we do not use it in our analysis.
%This is because, as we discuss in Appendix~\ref{app:additional_phase}, for both the problems we consider, the extra phases vanish.


\section{Intermission: diagonalizing multicomponent operators}

We start again, with a Hermitian linear differential operator $\widehat{\mathsf{D}}$ in the form of an $N\times N$ matrix satisfying a wave equation
%
\begin{equation}
  \widehat{\mathsf{D}}\Psi = 0,
\end{equation}
%
where $\Psi$ is an $N$-component wave field.
%
We want to find a unitary operator $\widehat{\mathsf{U}}$ such that
%
\begin{equation}
  \widehat{\mathsf{U}}^{\dagger}\widehat{\mathsf{D}}\widehat{\mathsf{U}} = \widehat{\Lambda}\label{eq:diagonalization}
\end{equation}
%
is a diagonal operator.
Clearly, this equivalent to demanding that $\widehat{\mathsf{D}}\widehat{\mathsf{U}} = \widehat{\mathsf{U}}\widehat{\Lambda}$.
If we can find such an operator and solve the decoupled set of equations given by $\widehat{\Lambda}\Phi = 0$ somehow,
the solutions to the original equation can then be recovered from $\Phi$ using $\Psi = \widehat{\mathsf{U}}\Phi$.
However, Eq.~\eqref{eq:diagonalization} is an equation involving operator products and standard linear algebra methods do not (directly) help in finding the unitary operator $\widehat{\mathsf{U}}$.
Instead, we start by finding the symbol form of the following two equations:
%
\begin{equation}
  \begin{aligned}
    \widehat{\mathsf{D}}\widehat{\mathsf{U}} &= \widehat{\mathsf{U}}\widehat{\Lambda}\\
    \widehat{\mathsf{U}}^{\dagger}\widehat{\mathsf{U}} &= \hat{\mathsf{I}}_{N}.
  \end{aligned}
  \label{eq:diagonal2}
\end{equation}
%
Here we assume that the operators $\widehat{\mathsf{D}}$, $\widehat{\mathsf{U}}$, and $\widehat{\Lambda}$ have some problem-relevant ordering parameter $\epsilon$ so that we can expand%
\footnote{In their papers, Littlejohn and coworkers~\cite{littlejohn1991,weigert1993} assume that $\widehat{\mathsf{D}}$ is \emph{not} ordered in $\epsilon$, i.e., $\widehat{\mathsf{D}} = \widehat{\mathsf{D}}_{0}$.
  This might seem confusing at first since the parameter $\epsilon$ must appear somewhere in the problem.
  The thing is that very often, $\epsilon$ appears as a factor to a spatial derivative operator, e.g., $\epsilon\partial_{x}$, which once expressed in terms of the momentum operator, does not have additional dependence on $\epsilon$.
  Of course, $\widehat{\mathsf{D}}$ could have further nontrivial dependence on $\epsilon$, which is a more general situation, and the one we want to consider here.
}
the operators (and their symbols) in terms of $\epsilon$ as follows:
%
\begin{equation}
  \begin{aligned}
    \widehat{\mathsf{D}} &= \widehat{\mathsf{D}}^{(0)} + \epsilon\widehat{\mathsf{D}}^{(1)} + \epsilon^{2}\widehat{\mathsf{D}}^{(2)} + \cdots\\
    \widehat{\mathsf{U}} &= \widehat{\mathsf{U}}^{(0)} + \epsilon\widehat{\mathsf{U}}^{(1)} + \epsilon^{2}\widehat{\mathsf{U}}^{(2)} + \cdots\\
    \widehat{\Lambda} &= \widehat{\Lambda}^{(0)} + \epsilon\widehat{\Lambda}^{(1)} + \epsilon^{2}\widehat{\Lambda}^{(2)} + \cdots
  \end{aligned}
  \qquad\text{and}\qquad
  \begin{aligned}
    {\mathsf{D}} &= {\mathsf{D}}^{(0)} + \epsilon{\mathsf{D}}^{(1)} + \epsilon^{2}{\mathsf{D}}^{(2)} + \cdots\\
    {\mathsf{U}} &= {\mathsf{U}}^{(0)} + \epsilon{\mathsf{U}}^{(1)} + \epsilon^{2}{\mathsf{U}}^{(2)} + \cdots\\
    {\Lambda} &= {\Lambda}^{(0)} + \epsilon{\Lambda}^{(1)} + \epsilon^{2}{\Lambda}^{(2)} + \cdots
  \end{aligned}
\end{equation}
%
Because the Weyl transform preserves Hermiticity, $\mathsf{D}$ is a Hermitian matrix to all orders.
Our requirement is that $\Lambda$ remains diagonal at all orders of $\epsilon$.
To proceed, we use the Moyal star product to find the symbol form of Eq.~\eqref{eq:diagonal2} at different orders of $\epsilon$.
At $\mathcal{O}(\epsilon^0)$ we find $\mathsf{D}^{(0)}\mathsf{U}^{(0)} = \mathsf{U}^{(0)}\Lambda^{(0)}$ and $\mathsf{U}^{(0)\dagger}\mathsf{U}^{(0)} = \mathsf{I}_{n}$, with the first equation being equivalent to
%
\begin{equation}
  \mathsf{U}^{(0)\dagger}\mathsf{D}^{(0)}\mathsf{U}^{(0)} = \Lambda^{(0)}.\label{eq:omega0}
\end{equation}
%
Since we demand $\Lambda^{(0)}$ to be diagonal, this is the usual linear algebra problem of diagonalizing the matrix $\mathsf{D}^{(0)}$.
We thus deduce that the columns of the matrix $\mathsf{U}^{(0)}$ are composed of eigenvectors $\tau_{a}$ with eigenvalues $\lambda^{(0)}_{a}$ satisfying $\mathsf{D}^{(0)}\tau_{a} = \lambda^{(0)}_{a}\tau_{a}$, with $a = 1,2,\ldots,N$.
More explicitly,
%
\begin{equation}
  \mathsf{U}^{(0)} =
  \begin{pmatrix}
    \tau_{1} & \tau_{2} & \cdots & \tau_{n}
  \end{pmatrix}.
\end{equation}
%
The matrix $\Lambda^{(0)} = \diag(\lambda_{1}^{(0)}, \lambda_{2}^{(0)}, \ldots, \lambda_{N}^{(0)})$ is composed of the eigenvalues $\lambda_{a}^{(0)}$.

At $\mathcal{O}(\epsilon^{1})$, demanding $\mathsf{D}\mathsf{U} = \mathsf{U}\Lambda$ gives us
%
\begin{equation}
\mathsf{D}^{(1)}\mathsf{U}^{(0)} + \mathsf{D}^{(0)}\mathsf{U}^{(1)} + (i/2)\left\{\mathsf{D}^{(0)}, \mathsf{U}^{(0)}\right\} =
  \mathsf{U}^{(1)}\Lambda^{(0)} + \mathsf{U}^{(0)}\Lambda^{(1)} + (i/2)\left\{\mathsf{U}^{(0)}, \Lambda^{(0)}\right\}.
\end{equation}
%
Multiplying from the left by $\mathsf{U}^{(0)\dagger}$ and making use of $\mathsf{U}^{(0)\dagger}\mathsf{D}^{(0)} = \Lambda^{(0)}\mathsf{U}^{(0)\dagger}$, we get the $\mathcal{O}(\epsilon)$ correction%
\footnote{For further higher-order corrections to $\Lambda$, see Eq.~(19) of Ref.~\cite{weigert1993}.}
%
\begin{equation}
  \Lambda^{(1)} = \left(\mathsf{U}^{(0)\dagger}\mathsf{D}^{(1)}\mathsf{U}^{(0)} +
  (i/2)\mathsf{U}^{(0)\dagger}\left\{\mathsf{D}^{(0)},\mathsf{U}^{(0)}\right\} - (i/2)\mathsf{U}^{(0)\dagger}\left\{\mathsf{U}^{(0)},\Lambda^{(0)}\right\}\right) + \left[\Lambda^{(0)},\mathsf{U}^{(0)\dagger}\mathsf{U}^{(1)}\right].
  \label{eq:Lambda1}
\end{equation}
%
Above $[\cdot,\cdot]$ denotes the matrix commutator.
We can find $\mathsf{U}^{(0)}$ and $\Lambda^{(0)}$ by diagonalizing $\mathsf{D}^{(0)}$.
The matrix $\mathsf{D}^{(1)}$ (if it is nonzero) can be found by expanding the symbol matrix $\mathsf{D}$.
But that will still not let us find $\Lambda^{(1)}$ since we also need $\mathsf{U}^{(1)}$ to evaluate the commutator term $[\Lambda^{(0)},\mathsf{U}^{(0)\dagger}\mathsf{U}^{(1)}]$.
To proceed, we recall that we want $\Lambda$ to be diagonal at all orders, which means that $\Lambda^{(1)}$ must be a diagonal matrix as well.
The $ab$th entry of the commutator term evalutes to
%
\begin{equation}
%  \begin{aligned}
    \left[\Lambda^{(0)},\mathsf{U}^{(0)\dagger}\mathsf{U}^{(1)}\right]_{ab} = %\Lambda_{0,\alpha\gamma}\left(\mathsf{U}^{(0)\dagger}\mathsf{U}^{(1)}\right)_{\gammab} -
  %\left(\mathsf{U}^{(0)\dagger}\mathsf{U}^{(1)}\right)_{a\gamma}\Lambda_{0,\gammab}\\
    %&= \lambda_{0}^{(i)}\delta_{a\gamma}\left(\mathsf{U}^{(0)\dagger}\mathsf{U}^{(1)}\right)_{\gammab}
  %-\left(\mathsf{U}^{(0)\dagger}\mathsf{U}^{(1)}\right)_{a\gamma} \lambda_{0}^{(k)}\delta_{\gammab}\\
    \left[\lambda^{(0)}_{a} - \lambda^{(0)}_{b}\right]\left(\mathsf{U}^{(0)\dagger}\mathsf{U}^{(1)}\right)_{ab},
%  \end{aligned}
    \label{eq:diagonal}
\end{equation}
%
where we have made use of the fact that $\Lambda^{(0)}$ is diagonal so that $\Lambda^{(0)}_{ab} = \lambda^{(0)}_{a}\delta_{ab}$.
And we see that the diagonal entries (with $a=b$) of the commutator vanish.
So the commutator term does not contribute towards the diagonal entries of $\Lambda^{(1)}$ and we can find the $\mathcal{O}(\epsilon^{1})$ correction $\lambda^{(1)}_{a}$ to the eigenvalues by carefully evaluating diagonal entries of the remaining terms in the RHS of Eq.~\eqref{eq:Lambda1}.
Before we do that, we need to discuss the role of the commutator term further.
In fact, without this crucial term, the expansion will break down.

\subsection{Role of the commutator term}

None of our arguments so far guarantees that $\Lambda^{(1)}$ is diagonal.
In fact, the off-diagonal entries of four matrices in the RHS of Eq.~\eqref{eq:Lambda1} is generally not equal to zero.
The only way for $\Lambda^{(1)}$ to be diagonal then is if these off-diagonal entries somehow cancel each other.
We do not have the freedom to choose the off-diagonal entries in the first three terms since they only involve the matrices $\mathsf{U}^{(0)}$ and $\Lambda^{(0)}$, both of which are constrained by Eq.~\eqref{eq:omega0}.
However, the commutator term involves the matrix $\mathsf{U}^{(1)}$, which is something we have not found yet.
At the same time, $\mathsf{U}^{(1)}$ is not a completely arbitrary matrix, because on demanding unitarity of $\mathsf{U}$ at $\mathcal{O}(\epsilon)$ we get an additional equation that puts constraints on $\mathsf{U}^{(1)}$:
%
\begin{equation}
  \mathsf{U}^{(0)\dagger}\mathsf{U}^{(1)} + \mathsf{U}^{(1)\dagger}\mathsf{U}^{(0)} + (i/2)\left\{\mathsf{U}^{(0)\dagger}, \mathsf{U}^{(0)}\right\}= 0.
  \label{eq:unitarity}
\end{equation}
%
As we show below, this equation is not good enough to determine $\mathsf{U}^{(1)}$ completely, which is good for us since that gives us some freedom in choosing the off-diagonal elements of the commutator term in the way we want.
To simplify the commutator term we first define $\mathsf{X} = \mathsf{U}^{(0)\dagger}\mathsf{U}^{(1)}$, so that the previous equation becomes
%
\begin{equation}
  \mathsf{X} + \mathsf{X}^{\dagger} = -(i/2)\left\{\mathsf{U}^{(0)\dagger}, \mathsf{U}^{(0)}\right\}.
\end{equation}
%
We see that Hermitian part of $\mathsf{X}$, given by $\mathsf{A} = (\mathsf{X} + \mathsf{X}^{\dagger})/2$, is fixed by the above equation, so that $\mathsf{A} = (-i/4)\{\mathsf{U}^{(0)\dagger},\mathsf{U}^{(0)}\}$, which is clearly a Hermitian matrix.
This still leaves us with the possibility of picking the anti-Hermitian part of $\mathsf{X}$, which we denote by $i\mathsf{B}$. Here $\mathsf{B}$ is a Hermitian matrix defined%
\footnote{%
  Writing the anti-Hermitian part of $\mathsf{X}$ this way might look nonstandard and it is more natural to write it as $(\mathsf{X} - \mathsf{X}^{\dagger})/2$.
  This is to ensure that the diagonal entries of matrix $\mathsf{B}$ are real numbers, for the sake of a future argument.
Since the matrices $\mathsf{A}$ and $\mathsf{B}$ are both Hermitian, they have real diagonals.
  These matrices each have $n^{2}$ independent entries composed of $n^{2} - n$ complex off-diagonal entries and $n$ real diagonal entries.
  All $n^{2}$ entries of $\mathsf{A}$ are fixed by $\mathsf{U}^{(0)}$.
  That leaves us with the freedom to pick the $n^{2}$ remaining entries of $\mathsf{B}$, which is good enough to ensure that $\Lambda^{(1)}$ remains diagonal.
}
by $\mathsf{B} = -i(\mathsf{X} - \mathsf{X}^{\dagger})/2$.
After replacing $\mathsf{X} = \mathsf{U}^{(0)\dagger}\mathsf{U}^{(1)}$ in the commutator term with $\mathsf{A} + i\mathsf{B}$ and using Eqs.~\eqref{eq:Lambda1} and \eqref{eq:diagonal}, we find the off-diagonal entries of $\mathsf{B}$ to be
%
\begin{equation}
  \mathsf{B}_{ab} = i\frac{\mathsf{Q}_{ab}}{\lambda^{(0)}_{a} - \lambda^{(0)}_{b}},
  \quad \text{with }a \neq b,
  \label{eq:BantiHermitian}
\end{equation}
%
where
%
\begin{equation}
  \begin{aligned}
    \mathsf{Q} &= \mathsf{U}^{(0)\dagger}\mathsf{D}^{(1)}\mathsf{U}^{(0)} + (i/2)\mathsf{U}^{(0)\dagger}\left\{\mathsf{D}^{(0)},\mathsf{U}^{(0)}\right\} - (i/2)\mathsf{U}^{(0)\dagger}\left\{\mathsf{U}^{(0)},\Lambda^{(0)}\right\}\\
               &\phantom{}\qquad -(i/4)\Lambda\left\{\mathsf{U}^{(0)\dagger}, \mathsf{U}^{(0)}\right\} +
  (i/4)\left\{\mathsf{U}^{(0)\dagger}, \mathsf{U}^{(0)}\right\}\Lambda.
  \end{aligned}
\end{equation}
%
Although we have found an expression that gives the off-diagonal elements of $\mathsf{B}$, nothing in our arguments so far guaratees the Hermiticity of $\mathsf{B}$ and we have to explicitly check this.%
\footnote{Littlejohn and coworkers~\cite{littlejohn1991,weigert1993} seem to not have stressed this subtle point in their papers and they do not prove the Hermiticity of $\mathsf{B}$ explicitly.
  At first glance, we might think that $\mathsf{B}$ would Hermitian by construction because we \emph{took} $\mathsf{\mathsf{A}}$ and $i\mathsf{B}$ to be the Hermitian and anti-Hermitian parts of $\mathsf{X}$.
  This is not true however, as we obtained $\mathsf{A}$ from requiring unitarity of $\mathsf{U}$ at $\mathcal{O}(\epsilon)$ and we are now attempting to find the off-diagonal entries of $\mathsf{B}$ from Eq.~\eqref{eq:Lambda1}, which is an independent equation that guarantees the diagonalizability of $\mathsf{D}$ at $\mathcal{O}(\epsilon)$.
  However, there is no obvious reason why unitarity at $\mathcal{O}(\epsilon)$ should be compatible with diagonalizability at $\mathcal{O}(\epsilon)$.
  Indeed, we still need an additional requirement, i.e., the Hermiticity of $\mathsf{D}^{(1)}$ (guaranteed by the Weyl transform), for $\mathsf{B}$ to be Hermitian.
}
From Eq.~\eqref{eq:BantiHermitian}, we see that the matrix $\mathsf{Q}$ should be Hermitian if $\mathsf{B}$ is to be Hermitian.
For two general matrices $\mathsf{F}$ and $\mathsf{G}$ we have $\{\mathsf{F},\mathsf{G}\}^{\dagger} = -\{\mathsf{G}^{\dagger},\mathsf{F}^{\dagger}\}$.
Along with the assumption that $\mathsf{D}^{(1)}$ is Hermitian, we then find,
%
\begin{equation}
  \begin{aligned}
    \mathsf{Q}^{\dagger} &=  \mathsf{U}^{(0)\dagger}\mathsf{D}^{(1)}\mathsf{U}^{(0)} + (i/2)\left\{\mathsf{U}^{(0)\dagger},   \mathsf{D}^{(0)}\right\}\mathsf{U}^{(0)} - (i/2)\left\{\Lambda^{(0)}, \mathsf{U}^{(0)\dagger}\right\}\mathsf{U}^{(0)}\\
                         &\phantom{}\qquad - (i/4)\left\{\mathsf{U}^{(0)\dagger}, \mathsf{U}^{(0)}\right\}\Lambda
  + (i/4)\Lambda\left\{\mathsf{U}^{(0)\dagger}, \mathsf{U}^{(0)}\right\},
  \end{aligned}
  \label{eq:Qmat}
\end{equation}
%
which can be further simplified%
\footnote{The simplification proceeds (by explicitly computing matrix entries) by first showing that
  $\{\mathsf{U}^{(0)\dagger}, \mathsf{D}^{(0)}\}\mathsf{U}^{(0)} = \{\mathsf{U}^{(0)\dagger}, \mathsf{U}^{(0)}\}\Lambda - \mathsf{U}^{(0)\dagger}\{\mathsf{U}^{(0)}, \Lambda\} - \{ ^{1}\mathsf{U}^{(0)\dagger}, ^{3}\mathsf{U}^{(0)}\}^{2}\mathsf{D}^{(0)}$,
  and
  $\{\Lambda^{(0)}, \mathsf{U}^{(0)\dagger}\} = \Lambda^{(0)}\{\mathsf{U}^{(0)\dagger},\mathsf{U}^{(0)}\} - \mathsf{U}^{(0)\dagger}\{\mathsf{D}^{(0)},\mathsf{U}^{(0)}\} - \{^{1}\mathsf{U}^{(0)\dagger},^{3}\mathsf{U}^{(0)}\}^{2}\mathsf{D}^{(0)}$.
  Here the superscripts 1, 2, and 3 appearing before the matrices denote the order in which they are to be multiplied; see Ref.~\cite{littlejohn1991a} for further details on this convention.
  Upon using these results in the RHS of Eq.~\eqref{eq:Qmat}, we see that $\mathsf{Q}=\mathsf{Q}^{\dagger}$.
}
to show that $\mathsf{Q}^{\dagger} = \mathsf{Q}$, from which the Hermiticity of $\mathsf{B}$ follows.

Clearly, from Eq.~\eqref{eq:BantiHermitian} we see that our scheme would break down if the lowest-order symbol matrix $\mathsf{D}^{(0)}$ has an $\mathcal{O}(\epsilon^{0})$ degeneracy, i.e., if $\mathsf{D}^{(0)}$ is degenerate with $\lambda^{(0)}_{a} = \lambda^{(0)}_{b}$ for some $a \neq b$.
For similar reasons, we cannot also determine the diagonal entries $\mathsf{B}_{aa}$ from Eq.~\eqref{eq:BantiHermitian}.
However, as we show below, we can always take them to be zero, since they do not affect the unitarity of $\mathsf{U}$ to $\mathcal{O}(\epsilon)$.
The symbol matrix $\mathsf{U}$ to $\mathcal{O}(\epsilon)$ is
%
\begin{equation}
  \mathsf{U} = \mathsf{U}^{(0)} + \epsilon \mathsf{U}^{(1)} + \mathcal{O}(\epsilon^{2}) = \mathsf{U}^{(0)}\left[\mathsf{I}_{n} + \epsilon\left(\mathsf{A} + i\mathsf{B}' + i\mathsf{B}'' \right)\right] + \mathcal{O}(\epsilon^{2}),
  \label{eq:U_with_A_and_B}
\end{equation}
%
where we have made use of $\mathsf{U}^{(1)} = \mathsf{U}^{(0)}(\mathsf{A} + i\mathsf{B})$ and have written $\mathsf{B}$ as the sum of its diagonal part $\mathsf{B}'$ and off-diagonal part $\mathsf{B}''$.%
\footnote{Since $\mathsf{B}$ is Hermitian, its diagonal part $\mathsf{B}'$ is a real matrix.}
Now, note that we have complete freedom in choosing phase factors for the columns of $\mathsf{U}^{(0)}$, namely the eigenvectors of $\mathsf{D}^{(0)}$.
Rephasing the $a$th column by $e^{-i\epsilon\mathsf{B}'_{aa}}$ turns
%
\begin{equation}
    \mathsf{U}^{(0)} \to
      \begin{pmatrix}
        e^{-i\epsilon\mathsf{B}'_{11}}\tau_{1} &
        e^{-i\epsilon\mathsf{B}'_{22}}\tau_{2} &
        \cdots &
        e^{-i\epsilon\mathsf{B}'_{nn}}\tau_{n} &
      \end{pmatrix}
      = \mathsf{U}^{(0)}(\mathsf{I}_{n} - i\epsilon\mathsf{B}') + \mathcal{O}(\epsilon^{2})
\end{equation}
%
At the same time, the matrices $\mathsf{A} \to \mathsf{A} + \mathcal{O}(\epsilon)$ and $\mathsf{B}'' \to \mathsf{B}'' + \mathcal{O}(\epsilon)$.
Thus, the RHS of Eq.~\eqref{eq:U_with_A_and_B} becomes
%
\begin{equation}
  \left[\mathsf{U}^{(0)}(\mathsf{I}_{n} - i\epsilon\mathsf{B}') + \mathcal{O}(\epsilon^{2})\right]\left[\mathsf{I}_{n} + \epsilon\left(\mathsf{A} + i\mathsf{B}' + i\mathsf{B}'' + \mathcal{O}(\epsilon)\right)\right]
  =
  \mathsf{U}^{(0)}\left[\mathsf{I}_{n} + \epsilon\left(\mathsf{A} + i\mathsf{B}'' \right)\right] + \mathcal{O}(\epsilon^{2}).
\end{equation}
%
In other words, we can absorb any nonzero diagonal entries of $\mathsf{B}$ by a suitable rephasing of the columns of $\mathsf{U}^{(0)}$.
Thus, without loss of generality we take the diagonal part $\mathsf{B}'$ to be zero.

\subsection{First-order correction}

The matrix $\Lambda^{(1)}$, whose entries can be found from the parenthetical terms in the RHS of Eq.~\eqref{eq:Lambda1}, is now diagonal to $\mathcal{O}(\epsilon)$.
The diagonal entries of $\Lambda^{(1)}$ give the first-order correction $\lambda_{a}^{(1)}$ to the eigenvalue $\lambda_{a}$ for a specific polarization, i.e., $\lambda_{a}^{(1)} = \Lambda^{(1)}_{aa}$.
We proceed by further simplifying the second and third parenthetical terms in Eq.~\eqref{eq:Lambda1} by noting that
%
\begin{equation}
  \Lambda^{(0)} = \diag\left[\lambda^{(0)}_{1}, \lambda^{(0)}_{2}, \ldots, \lambda^{(0)}_{N}\right],
  \quad
  \mathsf{U}^{(0)}_{\mu a} = \tau_{a, \mu},
  \quad\text{and}\quad
  \mathsf{U}^{(0)\dagger}_{a \mu} = \tau^{*}_{a, \mu}.
  \label{eq:Utau}
\end{equation}
%
Above, $\tau_{a, \mu}$ refers to the $\mu$th component of the polarization vector $\tau_{a}$ and $(\cdot)^{*}$ denotes complex conjugation.
Here, we have also used mixed Latin/Greek indices to separate the polarization index from the component indices (both run from $1$ to $N$, however).
Using Eq.~\eqref{eq:Utau}, the diagonal components of the first term in the parenthetical expression in Eq.~\eqref{eq:Lambda1} can be written as
%
\begin{equation}
\left(\mathsf{U}^{(0)\dagger}\mathsf{D}^{(1)}\mathsf{U}^{(0)}\right)_{aa} = \tau_{a,\mu}^{*}\mathsf{D}^{(1)}_{\mu\nu}\tau_{a,\nu}.
\label{eq:1st_term}
\end{equation}
%
Next, we simplify the diagonal entries of the second parenthetical term of Eq.~\eqref{eq:Lambda1} as
%
\begin{equation}
  \begin{aligned}
    (i/2)\left(\mathsf{U}^{(0)\dagger}  \left\{\mathsf{D}^{(0)}, \mathsf{U}^{(0)}\right\}\right)_{aa} &= (i/2)\mathsf{U}^{(0)\dagger}_{a\mu}\left(\partial_{x}\mathsf{D}^{(0)}_{\mu\nu}\partial_{k}\mathsf{U}^{(0)}_{\nu a} - \partial_{k}\mathsf{D}^{(0)}_{\mu\nu}\partial_{x}\mathsf{U}_{\nu a}^{(0)}\right)\\
                                                                                                                  &=  (i/2)\left[\partial_{x}\left(\Lambda^{(0)}_{a\mu}\mathsf{U}^{(0)\dagger}_{\mu\nu}\right) - \partial_{x}\mathsf{U}^{(0)\dagger}_{a\mu}\mathsf{D}^{(0)}_{\mu\nu}\right]\partial_{k}\mathsf{U}_{\nu a}^{(0)}\\
                                                                                                                    &\phantom{}\qquad - (i/2)\left[\partial_{k}\left(\Lambda^{(0)}_{a\mu}\mathsf{U}^{(0)\dagger}_{\mu\nu}\right) - \partial_{k}\mathsf{U}^{(0)\dagger}_{a\mu}\mathsf{D}^{(0)}_{\mu\nu}\right]\partial_{x}\mathsf{U}_{\nu a}^{(0)}\\
                                                                                                                      &= (i/2)\lambda^{(0)}_{a}\left\{\tau_{a,\mu}^{*}, \tau_{a,\mu}\right\} - (i/2)\tau_{a,\mu}^{*}\left\{\tau_{a,\mu},\lambda^{(0)}_{a}\right\}\\
                                                                                                                        &\phantom{}\qquad - (i/2)\mathsf{D}^{(0)}_{\mu\nu}\left\{\tau_{a,\mu}^{*}, \tau_{a,\nu}\right\}.
  \end{aligned}
  \label{eq:2nd_term}
\end{equation}
%
In the second step above, we have also made use of $\mathsf{U}^{(0)\dagger}\mathsf{D}^{(0)} = \Lambda^{(0)}\mathsf{U}^{(0)\dagger}$.
Finally, we can write the diagonal entries of the third parenthetical term in Eq.~\eqref{eq:Lambda1} as
%
\begin{equation}
  -(i/2)\left(\mathsf{U}^{(0)\dagger}\left\{\mathsf{U}^{(0)}, \Lambda^{(0)}\right\}\right)_{aa} = -(i/2)\tau_{a,\mu}^{*}\left\{\tau_{a,\mu},\lambda_{a}^{(0)}\right\}.
  \label{eq:3rd_term}
\end{equation}
%
Putting the simplified expressions from Eqs.~\eqref{eq:1st_term}--\eqref{eq:3rd_term} in Eq.~\eqref{eq:Lambda1}, we find the first-order correction in the eigenvalue $\lambda$ to be%
\footnote{This equation is identical to Eq.~(3.21) of Ref.~\cite{littlejohn1991a}, except for the first term involving $\mathsf{D}^{(1)}$, as these authors assume that the operator $\widehat{\mathsf{D}}$ is not ordered in $\epsilon$.}
%
\begin{equation}
  \boxed{
  \lambda^{(1)}_{a} = \tau_{a,\mu}^{*}\mathsf{D}_{\mu\nu}^{(1)}\tau_{a,\nu} - i\tau_{a,\mu}^{*}\left\{\tau_{a,\mu}, \lambda^{(0)}_{a}\right\} - (i/2)\left(\mathsf{D}^{(0)}_{\mu\nu} - \lambda^{(0)}_{a}\delta_{\mu\nu}\right)\left\{\tau_{a,\mu}^{*}, \tau_{a,\nu}\right\}}
  \label{eq:lambda1}
\end{equation}
%
Note that in the above equation there is no sum over the polarization index $a$, whereas the repeated Greek indices $\mu$ and $\nu$ indicate summation.

\section{Evolution of the polarization phase}

To make the transition between the variational theory presented in Section~\ref{sec:varintro} and the results in the previous section easier, we first note that
%
\begin{equation}
  \widehat{\mathsf{D}} = \widehat{\mathsf{U}}\widehat{\Lambda}\widehat{\mathsf{U}}^{\dagger}.
\end{equation}
%
Putting this in Eq.~\eqref{eq:wave_action_trace_form}, the wave action can be written as
%
\begin{equation}
  \mathscr{U} = \tfrac{1}{2}\tr\left(\widehat{\mathsf{D}}\widehat{\mathsf{W}}\right) = \tfrac{1}{2}\tr\left(\widehat{\mathsf{U}}\widehat{\Lambda}\widehat{\mathsf{U}}^{\dagger}\ket{\psi}\bra{\psi}\right) = \tfrac{1}{2}\tr\left(\widehat{\Lambda}\widehat{\mathsf{U}}^{\dagger}\ket{\psi}\bra{\psi}\widehat{\mathsf{U}}\right)
  = \tfrac{1}{2}\sum_{a} \hat{\lambda}_{a}\ket{\widetilde{\psi}_{a}}\bra{\widetilde{\psi}_{a}}
\end{equation}
%
where we have used standard results pertaining to the trace of matrix products and have defined $\ket{\widetilde{\psi}_{a}} = \widehat{\mathsf{U}}_{a\mu}\ket{\psi_{\mu}}$.
Proceeding by similar arguments as in Section~\ref{sec:varintro}, we express the wave action in terms of the Weyl symbol of the operator $\hat{\lambda}_{a}$, given by $\lambda_{a}$,  and the Wigner function $W_{a}$ corresponding to the density matrix $\ket{\psi_{a}}\bra{\psi_{a}}$ to find [cf.~\eqref{eq:wave_action_symbol_form}]
%
\begin{equation}
  \mathscr{U} = \frac{1}{4\pi\epsilon}\sum_{a}\int \dd{x}\,\dd{k}\, \lambda_{a}(x, k) W_{a}(x, k).
\end{equation}
%
The Wigner function $W_{a}$ to the lowest is
%
\begin{equation}
  W_{a} = \mathsf{U}^{(0)\dagger}_{a\nu}\mathsf{W}_{\nu\mu}\mathsf{U}^{(0)}_{\mu a} + \epsilon(\square) + \mathcal{O}(\epsilon^{2})
\end{equation}
%
where $\epsilon(\square)$ represents all $\mathcal{O}(\epsilon)$ correction terms, e.g., those that appear during the application of the Moyal formula to compute $W_{a}$, terms involving $\mathsf{U}^{(1)}$, etc.
One might think that it is important to find these terms, which is of the same order as the phase correction term we are after.
However, as we shall see, these terms drop out once we work out the equations of motion, and for this reason, we leave them unevaluated.
Next, we insert the eikonal ansatz for the wave fields, $\psi_{\mu} = A_{\mu}e^{iS(x)/\epsilon}$, and using the steps we followed in p.~\pageref{page:redaction}, find the reduced Wigner function to be
%
\begin{equation}
  \begin{aligned}
    W_{a,\text{R}} &= 2\pi \epsilon\big[\tau_{a,\nu}^{*} A_{\nu}(x)\big]\big[A^{*}_{\mu}(x)\tau_{a,\mu}\big]\delta\big[k - \partial_{x}S(x)\big] + \mathcal{O}(\epsilon^{2})\\
                   &= 2\pi\epsilon {B^{2}_{a}}(x, k)\delta\big[k - \partial_{x}S(x)\big] + \epsilon(\square) + \mathcal{O}(\epsilon^{2}).
  \end{aligned}
\end{equation}
%
Above, the correction $\epsilon(\square)$ now includes previous correction terms as well as those arising from higher-order expansions of the eikonal phase $S(x)$.
To $\mathcal{O}(\epsilon^{2})$, the eigenvalue $\lambda_{a} = \lambda_{a}^{(0)} + \epsilon\lambda_{a}^{(1)} + \mathcal{O}(\epsilon)$, so that
%
\begin{equation}
  \begin{aligned}
    \mathscr{U}\big[B_{a}, S\big] &= \mathscr{U}^{(0)}\left[B_{a}, S\right] + \epsilon\mathscr{U}^{(1)}\left[B_{a}, S\right] + \mathcal{O}(\epsilon^{2})\\
                                  &= \tfrac{1}{2} \sum_{a} \int \dd{x}\, \lambda_{a}^{(0)}(x)B^{2}_{a}(x) + \epsilon\lambda_{a}^{(0)}(x)\times(\square) + \tfrac{1}{2}\epsilon\sum_{a}\int\dd{x}\,\lambda_{a}^{(1)}(x)B^{2}_{a}(x) + \mathcal{O}(\epsilon^{2}).
  \end{aligned}
  \label{eq:action_ho}
\end{equation}
%
To the lowest order, the action above is equivalent to the reduced action, Eq.~\eqref{eq:reduced_action}, we saw previously.
Varying the lowest-order action $\mathscr{U}[B_{a}, S]$ with respect to $B_{a}$ and $S(x)$, we find, as before, the eikonal and amplitude transport equations
%
\begin{equation}
  \begin{gathered}
    \lambda_{I}^{(0)}\big[x, k=\partial_{x}S(x)\big] = 0,\\
\frac{\dd}{\dd{x}}\left(\partial_{k}\left\{\lambda_{I}^{(0)}\big[x, k=\partial_{x}S(x)\big]B^{2}_{I}\big[x, k=\partial_{x}S(x)\big]\right\}\right) = 0,
  \end{gathered}
\end{equation}
%
where we have assumed again that only one eigenvalue, with polarization index $a = I$, vanishes.

Varying the higher-order action $\mathscr{U}^{(1)}[B_{a}, S]$ in Eq.~\eqref{eq:action_ho} and varying with respect to $B_{a}$, we find
%
\begin{equation}
  \lambda_{I}^{(0)}\delta(\square) + \lambda_{I}^{(1)}\left[x, k=\partial_{x}S(x)\right] = 0,
\end{equation}
%
where $\delta(\square)$ represents the variation of the correction terms $(\square)$ that we chose to not determine.
However, from the eikonal equation we know that $\lambda_{I}^{(0)} = 0$ so that once we insert Eq.~\eqref{eq:lambda1} into the above equation, we get
%
\begin{equation}
\tau_{I,\mu}^{*}\mathsf{D}_{\mu\nu}^{(1)}\tau_{I,\nu} - (i/2)\mathsf{D}^{(0)}_{\mu\nu}\left\{\tau_{I,\mu}^{*}, \tau_{I,\nu}\right\}
  - i\tau_{I,\mu}\left\{\tau_{I,\mu}, \lambda^{(0)}_{a}\right\} = 0
\end{equation}
%
Upon explicitly introducing the phase dependence on $\gamma$ by setting $\tau \to \tau e^{i\gamma}$, we find an evolution equation for the phase, which can be written down as
%
\begin{equation}
  \dot{\gamma} = \dot{\gamma}_{\text{G}} + \dot{\gamma}_{\text{NG}},
\end{equation}
%
where
%
\begin{equation}\boxed{%
  \dot{\gamma}_{\text{G}} = i\tau_{\mu}^{*}\left\{\tau_{\mu}, \lambda^{(0)}\right\}
  \quad\text{and}\quad
\dot{\gamma}_{\text{NG}} = (i/2)\mathsf{D}^{(0)}_{\mu\nu}\left\{\tau_{\mu}^{*}, \tau_{\nu}\right\} - \tau_{\mu}^{*}\mathsf{D}_{\mu\nu}^{(1)}\tau_{\nu}.}
\label{eq:extra_phases}
\end{equation}
%
Above, we have suppressed the polarization index $I$ for simplicity.
The first of the extra phases $\gamma_{\text{G}}$ has a general form of a geometric phase upon treating the $x$-$k$ phase space as a parameter space.
To see this, note that $\dot{\gamma}_{\text{G}} = i\tau^{*}_{\mu}\left\{\tau_{\mu},\lambda^{(0)}\right\} = i\tau^{\dagger}\cdot\dot{\tau}$, so that around a closed orbit $\mathcal{C}$ in phase space, the accumulated phase is
%
\begin{equation}
\gamma_{\text{G}}
= \oint_{\mathcal{C}} \dot{\gamma}_{\text{G}}\,\dd{\sigma}
= i\oint_{\mathcal{C}} \tau^{\dagger}\cdot\dd{\tau}
= i\oint_{\mathcal{C}} \braket{\tau |\nabla_{\vartheta}\tau}
\cdot\dd\vartheta,
\end{equation}
%
where $\vartheta = (x, k)$ denotes the ``parameters'' that are being varied along the orbit.
The phase $\gamma_{\text{G}}$, which is the integral of a differential 1-form, depends only on the path $\mathcal{C}$ and not how it is parameterized (like all geometric phases).
The second phase $\gamma_{\text{NG}}$ has no such interpretation and is not a geometric phase.

\section{Bound waves in phase space}
\label{sec:bound}

We expect the rays of bound waves to be bounded in phase space as well, with these rays being topologically equivalent to a circle~\cite{keller1958,mcdonald1988}.
Such rays oscillate between two classical turning points where $k = 0$ and $\dot{x} = 0$ [Fig.~\ref{fig:caustic}(a)].
From our previous discussion, we see that turning points are examples of caustics, i.e., points on the ray where $\dot{x} = 0$.
Of course, on a bound ray, apart from the classical turning points, there could be other caustics as well [see Fig.~\ref{fig:caustic}(b)].

The eikonal solution for the wave field is of the form $\psi(x) = B(x)\tau(x) e^{i\gamma(x)} e^{iS(x)/\epsilon}$, where $\gamma$ is the adiabatic phase that evolves according to Eq.~\eqref{eq:extra_phases}.
The eikonal phase $S(x)$ can be recovered by integrating $k(x)$ along a ray.
As $\psi(x)$ must be single valued, the overall phase must be an integer multipe of $2\pi$, which leads to a modified Bohr--Sommerfeld quantization condition of the form
%
\begin{equation}
  \epsilon^{-1}\oint \dd{x}\,k(x;\, \omega) = 2\left(n + \frac{\alpha}{4}\right)\pi - \gamma,
  \label{eq:quantization}
\end{equation}
%
from which bound-state frequencies $\omega$ can be obtained.
Above, the quantum number $n \in \mathbb{N}_{0}$ and $\alpha$ is the Keller--Maslov index~\cite{keller1958,maslov1981}, which accounts for phase jumps at the turning points.
Closed orbits in a two-dimensional phase space that can be smoothly deformed to a small circle always have $\alpha = 2$~\cite{percival1977}.
%
\begin{figure}
  \begin{center}
    \includegraphics{localization/caustic.pdf}
  \end{center}
  \caption{%
    (a) In phase space, bound states are represented by rays in the form of closed orbits, which is analogous to that of a bound particle oscillating between two classical turning points ($\pm x^{\star}$ in the cartoon).
    Other trajectories represent unbound states.
    (b) A bound state represented by a ``peanut''-shaped orbit has six caustics.%
  }
  \label{fig:caustic}
\end{figure}

\section{Weyl symbols}

The Weyl symbol of an operator $\widehat{A}(x, \hat{k})$ is defined by
%
\begin{equation}
  A(x, k) = \int \dd{s}\, e^{-iks/\epsilon} \Bra{x + \tfrac{1}{2}s}\widehat{A}\Ket{x - \tfrac{1}{2}s}.
  \label{app:eq:weyl_def}
\end{equation}
%
A consequence of the above definition is that if operator $\widehat{A}$ is Hermitian with $\widehat{A}^{\dagger} = \widehat{A}$, then, taking the complex conjugate of Eq.~\eqref{app:eq:weyl_def}, we find
%
\begin{equation}
  A^{*}(x, k) = \int \dd{s}\, e^{+iks/\epsilon} \Bra{x + \tfrac{1}{2}s}\widehat{A}\Ket{x - \tfrac{1}{2}s}^{*}
  = \int \dd{s}\, e^{+iks/\epsilon} \Bra{x - \tfrac{1}{2}s}\widehat{A}\Ket{x + \tfrac{1}{2}s} = A(x, k),
  \label{eq:weylsym}
\end{equation}
%
showing that the symbol $A(x, k)$ is a real function of $x$ and $k$.

%! TEX root = thesis.tex
% vim: ft=tex et sts=2 sw=2

\chapter{Semiclassical physics}

This chapter is a brief discussion of elastic waves.
In particular, we discuss waves propagating on a filament and shell, both with varying curvatures.
The equations we derive in this chapter would form the basis of our discussions in the next chapter.

\section{WKB analysis}

Consider the differential equation
%
\begin{equation}
  y''(x) + y(x) + \varepsilon \lambda(x) = 0.
\end{equation}
%
Here $\varepsilon$ is a small parameter and suppose we wish to find solutions to this differential equation when $\varepsilon \to 0$.
This is a \emph{regular perturbation} problem because the general characteristic features of this differential equation, namely the fact that it is second order with two independent solutions remains preserved on setting $\varepsilon = 0$.
Hence we can attempt to find a solution of the form
%
\begin{equation}
  y(x) = y_0(x) + \varepsilon y_{1}(x) + \varepsilon^{2} y_{2}(x) + \cdots
\end{equation}
%
Putting this in the original differential equation, we can solve it at different orders of $\varepsilon$ to find
%
\begin{equation}
  \begin{aligned}
    \mathcal{O}(\epsilon^{0}) &: y_{0}''(x) + y_{0}(x) = 0\\
    \mathcal{O}(\epsilon^{1}) &: y_{1}''(x) + y_{0}(x) = 0
  \end{aligned}
\end{equation}
%

The WKB method is a method to find solutions to \emph{singularly perturbed} different equations.

Trouble might occur for instance when

1) the highest order derivative is multiplied by $\varepsilon$.
2) the problem totally changes characteristics when the parameter $\varepsilon$ is equal to zero.
3) the problem is defined on infinite regions.
4) singular points are present.
5) the equation models physical processes with several time- or length scales.

1-5 are called singular perturbation problems.

In many cases we deal with problem containing boundary layers. We can roughly treat these problems by
i) letting  we get a good approximation for the outer region.
ii) rescale the problem to get an inner approximation.
iii) match inner and outer approximations.

Singular perturbation is matched asymptotic expansion.

\section{Multicomponent wave problems}

Consider the wave equation
%
\begin{equation}
  \hat{\mathsf{D}}\ket{\psi} = 0
  \qquad
  (\text{i.e., }\hat{\mathsf{D}}_{jk}\ket{\psi_{k}} = 0.)
\end{equation}
%
An equation of this sort can be derived from an action of the form
%
\begin{equation}
  \mathscr{U}\left[\psi^{*}, \psi\right] = \frac{1}{2}\bra{\psi}\hat{\mathsf{D}}\ket{\psi}
  = \int \dd{x}\,\dd{x'}\, \braket{\psi|x}\bra{x}\hat{\mathsf{D}}\ket{x'}\braket{x'|\psi}
  = \frac{1}{2}\int \dd{x}\dd{x'}\, \psi^{*}_{j}(x)\,\hat{\mathsf{D}}_{jk}(x, x')\,\psi_{k}(x') = 0.
\end{equation}
%
Above we have inserted the resolution of identity $\int \dd{x}\,\ket{x}\bra{x} = 1$ in appropriate places to express $\mathscr{U}$ in terms of $\psi$ and its conjugate $\psi^{*}$.
Also, we have made the product between the matrix operator $\hat{\mathsf{D}}$ and the wave vector $\psi$ explicit, with $\braket{x|\hat{\mathsf{D}}_{jk}|x'} = \hat{\mathsf{D}}_{jk}(x, x')$ being the matrix element of $\hat{\mathsf{D}}_{jk}$ in the position basis.
Varying $\mathscr{U}[\psi^{*}, \psi]$ with respect to $\psi^{*}$ gives the original wave equation, Eq.~XXX.

The Wigner tensor $\mathsf{W}_{kj}$ is defined as the symbol of the density operator $\hat{\mathsf{W}} = \ket{\psi}\bra{\psi}$ with kernel $\hat{\mathsf{W}}_{kj}(x',x) = \psi_{k}(x')\psi_{j}^{*}(x)$.
%
\begin{equation}
  \mathsf{W}_{kj}(x, k) = \int \dd{s}\, e^{-iks/\epsilon}\, \psi_{k}\left(x + \tfrac{1}{2}s\right)\psi^{*}_{j}\left(x - \tfrac{1}{2}s\right).
\end{equation}
%
Inverting this expression we obtain
%
\begin{equation}
  \psi_{k}\left(x + \tfrac{1}{2}s\right) \psi^{*}_{j}\left(x - \tfrac{1}{2}s\right) = \frac{1}{2\pi \epsilon} \int \dd{k}\, e^{iks/\epsilon}\,\mathsf{W}_{kj}(x, k)
\end{equation}
%
so that
%
\begin{equation}
  \psi_{k}(x') \psi^{*}_{j}(x) = \frac{1}{2\pi \epsilon} \int \dd{k}\, e^{ik(x' -x)/\epsilon}\,\mathsf{W}_{kj}\left[\tfrac{1}{2}(x' + x), k\right]
\end{equation}
%
In a similar fashion, we find
%
\begin{equation}
  \hat{\mathsf{D}}_{jk}(x, x') = \frac{1}{2\pi\epsilon} \int \dd{l}\, e^{il(x -x')/\epsilon}\,\mathsf{D}_{jk}\left[\tfrac{1}{2}(x + x'), l\right].
\end{equation}
%
Putting Eq.~XXX and XXX in Eq.~XXX, we arrive at
%
\begin{equation}
  \mathscr{U} = \frac{1}{(2\pi\epsilon)^{2}}\int \dd{x}\,\dd{x'}\,\dd{k}\,\dd{l}\, e^{i(l-k)(x -x')/\epsilon}\,\mathsf{W}_{kj}\left[\tfrac{1}{2}(x' + x), k\right]\, \mathsf{D}_{jk}\left[\tfrac{1}{2}(x + x'), l\right].
\end{equation}
%
Performing the change of variables $x \to \frac{1}{2}(x + x')$ and $x' \to x - x'$, which carries a Jacobian factor of unity, we arrive at
%
\begin{equation}
  \begin{aligned}
    \mathscr{U} &= \frac{1}{(2\pi\epsilon)^{2}}\int \dd{x}\,\dd{x'}\,\dd{k}\,\dd{l}\, e^{i(l-k)x'/\epsilon}\,\mathsf{W}_{kj}(x, k)\, \mathsf{D}_{jk}(x, l)\\
                &= \frac{1}{2\pi\epsilon}\int \dd{x}\,\dd{k}\,\mathsf{W}_{kj}(x,k)\,\mathsf{D}_{jk}(x, k)
  \end{aligned}
\end{equation}
%


To employ the semiclassical approximation to solve Eq.~XXX, we will proceed as follows: (i) write the matrix elements of $\hat{\mathsf{D}}$ and $W_{jk}$ in terms of their Weyl symbols; (ii) insert the eikonal ansatz; (iii) expand the action to the lowest order in the eikonal parameter and form the \emph{reduced action} $\mathscr{U}_{\text{R}}$; (iv) extract the semiclassical equations of motions by varying the reduced action.

\section{Intermission: diagonalizing multicomponent operators}

Consider a Hermitian linear differential operator $\hat{\mathsf{D}}$ satisfying the wave equation
%
\begin{equation}
  \hat{\mathsf{D}}\Psi = 0,
\end{equation}
%
where $\Psi$ is a multicomponent wave field.
%
We want to find a unitary operator ${\mathsf{U}}$ such that%
\footnote{It should be emphasized that diagonalizability and unitarity are independent. For instance, let $\mathsf{U}$ be the usual unitary matrix that diagonalizes a matrix $\mathsf{D}$.
  If $\Sigma$ is some nonzero diagonal matrix, not necessarily unitary, then $(\Sigma^{\dagger}\mathsf{U}^{\dagger})\mathsf{D}(\mathsf{U}\Sigma)$ is a diagonal matrix, which follows from the fact that the product of diagonal matrices is diagonal.
  In this sense $\mathsf{U}\Sigma$ can diagonalize $\mathsf{D}$ without being unitary.}
%
\begin{equation}
  \hat{\mathsf{U}}^{\dagger}\hat{\mathsf{D}}\hat{\mathsf{U}} = \hat{\Lambda}\label{eq:diagonalization}
\end{equation}
%
is a diagonal operator.
Clearly, this equivalent to demanding that $\hat{\mathsf{D}}\hat{\mathsf{U}} = \hat{\mathsf{U}}\hat{\Lambda}$.
If we can find such an operator and solve the decoupled set of equations given by $\hat{\Lambda}\Phi = 0$ somehow,
the solutions to the original equation can then be recovered from $\Phi$ using $\Psi = \hat{\mathsf{U}}\Phi$.
However, Eq.~\eqref{eq:diagonalization} is an equation involving operators and standard linear algebra methods do not (directly) help in finding the unitary operator $\hat{\mathsf{U}}$.

We start by finding the symbol form of the following two equations.
%
\begin{equation}
  \begin{aligned}
    \hat{\mathsf{D}}\hat{\mathsf{U}} &= \hat{\mathsf{U}}\hat{\Lambda}\\
    \hat{\mathsf{U}}^{\dagger}\hat{\mathsf{U}} &= \mathsf{I}_{n}.
  \end{aligned}
\end{equation}
%
Here we assume that the operators $\hat{\mathsf{D}}$, $\hat{\mathsf{U}}$, and $\hat{\Lambda}$ have some problem-relevant ordering parameter $\epsilon$ so that we can expanded the operators (and their symbols) in terms of $\epsilon$.%
\footnote{In their papers, Littlejohn and coworkers~\cite{littlejohn1991,weigert1993} assume that $\hat{\mathsf{D}}$ is \emph{not} ordered in $\epsilon$, i.e., $\hat{\mathsf{D}} = \hat{\mathsf{D}}_{0}$.
  This might seem confusing at first since the parameter $\epsilon$ must appear somewhere in the problem.
  The thing is that very often, $\epsilon$ appears as a factor to a spatial derivative operator, e.g., $-i\epsilon\partial_{x}$, which put together becomes a momentum operator that doesn't have further $\epsilon$ dependence.
  Of course, $\hat{\mathsf{D}}$ could have further nontrivial dependence on $\epsilon$, which is a more general situation, and the one we want to consider here.
}
%
\begin{equation}
  \begin{aligned}
    \hat{\mathsf{D}} &= \hat{\mathsf{D}}_{0} + \epsilon\hat{\mathsf{D}}_{1} + \epsilon^{2}\hat{\mathsf{D}}_{2} + \cdots\\
    \hat{\mathsf{U}} &= \hat{\mathsf{U}}_{0} + \epsilon\hat{\mathsf{U}}_{1} + \epsilon^{2}\hat{\mathsf{U}}_{2} + \cdots\\
    \hat{\Lambda} &= \hat{\mathsf\Lambda}_{0} + \epsilon\hat{\mathsf\Lambda}_{1} + \epsilon^{2}\hat{\mathsf\Lambda}_{2} + \cdots
  \end{aligned}
  \qquad\text{and}\qquad
  \begin{aligned}
    {\mathsf{D}} &= {\mathsf{D}}_{0} + \epsilon{\mathsf{D}}_{1} + \epsilon^{2}{\mathsf{D}}_{2} + \cdots\\
    {\mathsf{U}} &= {\mathsf{U}}_{0} + \epsilon{\mathsf{U}}_{1} + \epsilon^{2}{\mathsf{U}}_{2} + \cdots\\
    {\Lambda} &= {\mathsf\Lambda}_{0} + \epsilon{\mathsf\Lambda}_{1} + \epsilon^{2}{\mathsf\Lambda}_{2} + \cdots
  \end{aligned}
\end{equation}
%
Because the Weyl correspondence preserves Hermiticity, $\mathsf{D}$ is a Hermitian matrix.
We demand that $\Lambda$ remains diagonal at all orders of $\epsilon$.
We then use the Moyal star product to find the symbol form at different orders of $\epsilon$.
At $\mathcal{O}(\epsilon^0)$ we find $\mathsf{D}_{0}\mathsf{U}_{0} = \mathsf{U}_{0}\Lambda_{0}$ and $\mathsf{U}_{0}^{\dagger}\mathsf{U}_{0} = \mathsf{I}_{n}$, which is equivalent to
%
\begin{equation}
  \mathsf{U}_{0}^{\dagger}\mathsf{D}_{0}\mathsf{U}_{0} = \Lambda_{0}.\label{eq:omega0}
\end{equation}
%
Since we demand $\Lambda_{0}$ to be diagonal, this is the usual linear-algebra problem of diagonalizing the matrix $\mathsf{D}_{0}$.
We thus deduce that columns of the lowest order symbol $\mathsf{U}_{0}$ is composed of eigenvectors $\bm{\tau}^{(i)}$ with eigenvalues $\lambda_{0}^{(i)}$ satisfying $\mathsf{D}_{0}\bm{\tau}^{(i)} = \lambda_{0}^{(i)}\bm{\tau}^{(i)}$:
%
\begin{equation}
  \mathsf{U}_{0} =
  \begin{pmatrix}
    \bm{\tau}^{(1)} & \bm{\tau}^{(2)} & \cdots & \bm{\tau}^{(n)}
  \end{pmatrix},
\end{equation}
%
and $\Lambda_{0}$ is the diagonal matrix composed of the eigenvalues $\lambda^{(i)}$.

At $\mathcal{O}(\epsilon^{1})$, demanding $\mathsf{D}\mathsf{U} = \mathsf{U}\Lambda$ gives us
%
\begin{equation}
\mathsf{D}_{1}\mathsf{U}_{0} + \mathsf{D}_{0}\mathsf{U}_{1} + \frac{i}{2}\left\{\mathsf{D}_{0}, \mathsf{U}_{0}\right\} =
  \mathsf{U}_{1}\Lambda_{0} + \mathsf{U}_{0}\Lambda_{1} + \frac{i}{2}\left\{\mathsf{U}_{0}, \Lambda_{0}\right\}.
\end{equation}
%
Multiplying from the left by $\mathsf{U}_{0}^{\dagger}$ and making use of $\mathsf{U}_{0}^{\dagger}\mathsf{D} = \Lambda_{0}\mathsf{U}_{0}^{\dagger}$, we get the $\mathcal{O}(\epsilon)$ correction%
\footnote{For further higher-order corrections to $\Lambda$, see Eq.~(19) of Ref.~\cite{weigert1993}.
Equation~\eqref{eq:omega1} is almost identical to Eqs.~(22)--(24) of Ref.~\cite{venaille2023}, except for the commutator term.
In Ref.~\cite{venaille2022} the commutator term vanishes trivially since $\Lambda_{0}$ is considered to be a scalar.}
%
\begin{equation}
  \Lambda_{1} = \left(\mathsf{U}_{0}^{\dagger}\mathsf{D}_{1}\mathsf{U}_{0} +
  \frac{i}{2}\mathsf{U}_{0}^{\dagger}\left\{\mathsf{D}_{0},\Lambda_{0}\right\} - \frac{i}{2}\mathsf{U}_{0}^{\dagger}\left\{\mathsf{U}_{0},\Lambda_{0}\right\}\right) + \left[\Lambda_{0},\mathsf{U}_{0}^{\dagger}\mathsf{U}_{1}\right].
  \label{eq:omega1}
\end{equation}
%
Above $[~,~]$ denotes the matrix commutator.
We can find $\mathsf{U}_{0}$ and $\Lambda_{0}$ by diagonalizing $\mathsf{D}_{0}$.
The matrix $\mathsf{D}_{1}$ (if it is nonzero) can be found by expanding the symbol matrix $\mathsf{D}$.
But that will not let us find $\Lambda_{1}$ since we still need $\mathsf{U}_{1}$ to evaluate the commutator term $[\Lambda_{0},\mathsf{U}_{0}^{\dagger}\mathsf{U}_{1}]$.
To proceed, we recall that we want $\Lambda$ to be diagonal at all order, which means that $\Lambda_{1}$ must be a diagonal matrix as well.
The $\alpha\beta$th entry of the commutator term evalutes to
%
\begin{equation}
%  \begin{aligned}
    \left[\Lambda_{0},\mathsf{U}_{0}^{\dagger}\mathsf{U}_{1}\right]_{\alpha\beta} = %\Lambda_{0,\alpha\gamma}\left(\mathsf{U}_{0}^{\dagger}\mathsf{U}_{1}\right)_{\gamma\beta} -
  %\left(\mathsf{U}_{0}^{\dagger}\mathsf{U}_{1}\right)_{\alpha\gamma}\Lambda_{0,\gamma\beta}\\
    %&= \lambda_{0}^{(i)}\delta_{\alpha\gamma}\left(\mathsf{U}_{0}^{\dagger}\mathsf{U}_{1}\right)_{\gamma\beta}
  %-\left(\mathsf{U}_{0}^{\dagger}\mathsf{U}_{1}\right)_{\alpha\gamma} \lambda_{0}^{(k)}\delta_{\gamma\beta}\\
    \left[\lambda_{0}^{(i)} - \lambda_{0}^{(j)}\right]\left(\mathsf{U}_{0}^{\dagger}\mathsf{U}_{1}\right)_{\alpha\beta},
%  \end{aligned}
    \label{eq:diagonal}
\end{equation}
%
where we have made use of the fact that $\Lambda_{0}$ is diagonal with $\alpha\beta$th entry $\Lambda_{0,\alpha\beta} = \lambda_{0}^{(i)}\delta_{\alpha\beta}$.
And we see that the diagonal entries (with $\alpha=\beta$) of the commutator vanish.
So the commutator term does not contribute towards the diagonal entries of $\Lambda_{1}$ and we can find the $\mathcal{O}(\epsilon^{1})$ correction $\lambda_{1}$ by carefully evaluating diagonal entries of the remaining terms in the RHS of Eq.~XXX.
Before we do that, we need to discuss the role of the commutator term further.
In fact, without this crucial term, the expansion will break down.

\subsection{Role of the commutator term}

None of our arguments so far guarantees that $\Lambda_{1}$ is diagonal.
In fact, the off-diagonal entries of four matrices in the RHS of Eq.~\eqref{eq:omega1} is generally not equal to zero.
The only way for $\Lambda_{1}$ to be diagonal then is if these off-diagonal entries somehow cancel each other.
We don't have the freedom to choose the off-diagonal entries in the first three terms since they only involve the matrices $\mathsf{U}_{0}$ and $\Lambda_{0}$, both of which are constrained by Eq.~\eqref{eq:omega0}.
However, the commutator term involves the matrix $\mathsf{U}_{1}$, which is something we haven't found yet.
At the same time, $\mathsf{U}_{1}$ isn't a completely arbitrary matrix because on demanding unitarity of $\mathsf{U}$ at $\mathcal{O}(\epsilon)$ we get an additional equation that puts constraints on $\mathsf{U}_{1}$:
%
\begin{equation}
  \mathsf{U}_{0}^{\dagger}\mathsf{U}_{1} + \mathsf{U}_{1}^{\dagger}\mathsf{U}_{0} + \frac{i}{2}\left\{\mathsf{U}_{0}^{\dagger}, \mathsf{U}_{0}\right\}= 0.
  \label{eq:unitarity}
\end{equation}
%
This equation isn't good enough to determine $\mathsf{U}_{1}$ completely, which is good for us since that gives us some freedom in choosing the off-diagonal elements of the commutator term in the way we want.
To simplify the commutator term we first define $\mathsf{X} = \mathsf{U}_{0}^{\dagger}\mathsf{U}_{1}$, and from the previous equation we have
%
\begin{equation}
  \mathsf{X} + \mathsf{X}^{\dagger} = -\frac{i}{2}\left\{\mathsf{U}_{0}^{\dagger}, \mathsf{U}_{0}\right\}.
\end{equation}
%
We see that Hermitian part of $\mathsf{X}$, given by $\mathsf{A} = (\mathsf{X} + \mathsf{X}^{\dagger})/2$, is fixed by the above equation, and $\mathsf{A} = (-i/4)\{\mathsf{U}_{0}^{\dagger},\mathsf{U}_{0}\}$.
This still leaves us with the possibility of picking the anti-Hermitian part of $\mathsf{X}$, which we denote by $i\mathsf{B}$. Here $\mathsf{B}$ is a Hermitian matrix defined by $\mathsf{B} = -i(\mathsf{X} - \mathsf{X}^{\dagger})/2$.\footnote{%
  Writing the anti-Hermitian part of $\mathsf{X}$ this way might look nonstandard and it is more natural to write it as $(\mathsf{X} - \mathsf{X}^{\dagger})/2$.
  This is to ensure that the diagonal entries of matrix $\mathsf{B}$ are real numbers, for the sake of a future argument.
Since the matrices $\mathsf{A}$ and $\mathsf{B}$ are both Hermitian, they have real diagonals.
  These matrices each have $n^{2}$ independent entries composed of $n^{2} - n$ complex off-diagonal entries and $n$ real diagonal entries.
  All $n^{2}$ entries of $\mathsf{A}$ are fixed by $\mathsf{U}_{0}$.
  That leaves us with the freedom to pick the $n^{2}$ remaining entries of $\mathsf{B}$, which is good enough to ensure that $\Omega_{1}$ remains diagonal.
}
After replacing $\mathsf{X} = \mathsf{U}_{0}^{\dagger}\mathsf{U}_{1}$ in the commutator term with $\mathsf{A} + i\mathsf{B}$ and using Eqs.~\eqref{eq:omega1} and \eqref{eq:diagonal}, we can find the off-diagonal entries of $\mathsf{B}$ that will ensure that $\Lambda_{1}$ remains a diagonal matrix:
%
\begin{equation}
  \mathsf{B}_{\alpha\beta} = i\frac{\mathsf{Q}_{\alpha\beta}}{\lambda_{0}^{(\alpha)} - \lambda_{0}^{(\beta)}},
  \quad \text{with }\alpha \neq \beta,
\end{equation}
%
where
%
\begin{equation}
  \mathsf{Q} = \mathsf{U}_{0}^{\dagger}\mathsf{H}_{1}\mathsf{U}_{0} + \frac{i}{2}\mathsf{U}_{0}^{\dagger}\left\{\mathsf{D}_{0},\mathsf{U}_{0}\right\} - \frac{i}{2}\mathsf{U}_{0}^{\dagger}\left\{\mathsf{U}_{0},\Lambda_{0}\right\}
  -\frac{i}{4}\Lambda\left\{\mathsf{U}_{0}^{\dagger}, \mathsf{U}_{0}\right\} +
  \frac{i}{4}\left\{\mathsf{U}_{0}^{\dagger}, \mathsf{U}_{0}\right\}\Lambda.
\end{equation}
%
Although we have found an expression that gives the off-diagonal elements of $\mathsf{B}$, nothing in our arguments so far guaratees the Hermiticity of $\mathsf{B}$ and we have to explicitly check this.%
\footnote{Littlejohn and coworkers~\cite{littlejohn1991,weigert1993} seem to not have stressed this subtle point in their papers and they do not prove the Hermiticity of $\mathsf{B}$ explicitly.
  At first glance, we might think that $\mathsf{B}$ would Hermitian by construction because we \emph{took} $\mathsf{\mathsf{A}}$ and $i\mathsf{B}$ to be the Hermitian and anti-Hermitian parts of $\mathsf{X}$.
  This is not true however, as we obtained $\mathsf{A}$ from requiring unitarity of $\mathsf{U}$ at $\mathcal{O}(\epsilon)$ and we are now attempting to find $\mathsf{B}$ from Eq.~XXX, which is an independent equation that guarantees the diagonalizability of $\mathsf{D}$ at $\mathcal{O}(\epsilon)$.
  However, there is no obvious reason why unitarity at $\mathcal{O}(\epsilon)$ should be compatible with diagonalizability at $\mathcal{O}(\epsilon)$.
  Indeed, we still need an additional requirement, i.e., the Hermiticity of $\mathsf{D}_{1}$, for $\mathsf{B}$ to be Hermitian.
}
From Eq.~XXX, we see that the matrix $\mathsf{Q}$ should be Hermitian if $\mathsf{B}$ is to be Hermitian.
For two general matrices $\mathsf{F}$ and $\mathsf{G}$ we have $\{\mathsf{F},\mathsf{G}\}^{\dagger} = -\{\mathsf{G}^{\dagger},\mathsf{H}^{\dagger}\}$
Along with the assumption that $\mathsf{D}_{1}$ is Hermitian, we then find,
%
\begin{equation}
  \mathsf{Q}^{\dagger} =  \mathsf{U}_{0}^{\dagger}\mathsf{D}_{1}\mathsf{U}_{0} + \frac{i}{2}\left\{\mathsf{U}_{0}^{\dagger},   \mathsf{D}_{0}\right\}\mathsf{U}_{0} - \frac{i}{2}\left\{\Lambda_{0}, \mathsf{U}_{0}^{\dagger}\right\}\mathsf{U}_{0}
  - \frac{i}{4}\left\{\mathsf{U}_{0}^{\dagger}, \mathsf{U}_{0}\right\}\Lambda
  + \frac{i}{4}\Lambda\left\{\mathsf{U}_{0}^{\dagger}, \mathsf{U}_{0}\right\},
\end{equation}
%
which can be further simplified to show that $\mathsf{Q}^{\dagger} = \mathsf{Q}$,%
\footnote{The simplification proceeds (by explicitly computing matrix entries) by first showing that
  $\{\mathsf{U}_{0}^{\dagger}, \mathsf{D}_{0}\}\mathsf{U}_{0} = \{\mathsf{U}_{0}^{\dagger}, \mathsf{U}_{0}\}\Lambda - \mathsf{U}_{0}^{\dagger}\{\mathsf{U}_{0}, \Lambda\} - \{ ^{1}\mathsf{U}_{0}^{\dagger}, ^{3}\mathsf{U}_{0}\}^{2}\mathsf{D}_{0}$,
  and
  $\{\Lambda_{0}, \mathsf{U}_{0}^{\dagger}\} = \Lambda_{0}\{\mathsf{U}_{0}^{\dagger},\mathsf{U}_{0}\} - \mathsf{U}_{0}^{\dagger}\{\mathsf{D}_{0},\mathsf{U}_{0}\} - \{^{1}\mathsf{U}_{0}^{\dagger},^{3}\mathsf{U}_{0}\}^{2}\mathsf{D}_{0}$.
  Here the superscripts 1, 2, and 3 appearing before the matrices denote the order in which they are to be multiplied.
  Upon using these results in the RHS of Eq.~XXX, we see that $\mathsf{Q}=\mathsf{Q}^{\dagger}$.
}
from which the Hermiticity of $\mathsf{B}$ follows.

Clearly, our scheme would break down if the symbol matrix $\mathsf{D}$ has an $\mathcal{O}(\epsilon^{0})$ degeneracy, i.e., if $\mathsf{D}_{0}$ is degenerate with $\lambda_{0}^{(\alpha)} = \lambda_{0}^{(\beta)}$ for some $\alpha \neq \beta$.
Clearly, we cannot determine the diagonal entries $\mathsf{B}_{\alpha\alpha}$ from our results so far.
However, we can always take them to be zero, since they do not affect the unitarity of $\mathsf{U}$ to $\mathcal{O}(\epsilon)$.
The symbol matrix $\mathsf{U}$ to $\mathcal{O}(\epsilon)$ is
%
\begin{equation}
  \mathsf{U} = \mathsf{U}_{0} + \epsilon \mathsf{U}_{1} + \mathcal{O}(\epsilon^{2}) = \mathsf{U}_{0}\left[\mathsf{I}_{n} + \epsilon\left(\mathsf{A} + i\mathsf{B}' + i\mathsf{B}'' \right)\right] + \mathcal{O}(\epsilon^{2}),
\end{equation}
%
where we made use of $\mathsf{U}_{1} = \mathsf{U}_{0}(\mathsf{A} + i\mathsf{B})$ and wrote $\mathsf{B}$ as the sum of its diagonal part $\mathsf{B}'$ and off-diagonal part $\mathsf{B}''$.%
\footnote{Since $\mathsf{B}$ is Hermitian, its diagonal part $\mathsf{B}'$ is a real matrix.}
Now, note that we have complete freedom in choosing phase factors for the columns of $\mathsf{U}_{0}$, namely the eigenvectors of $\mathsf{D}_{0}$.
Rephasing the $\alpha$th column by $e^{-i\epsilon\mathsf{B}'_{\alpha\alpha}}$ turns
%
\begin{equation}
    \mathsf{U}_{0} \to
      \begin{pmatrix}
        e^{-i\epsilon\mathsf{B}'_{11}}\bm{\tau}^{(1)} &
        e^{-i\epsilon\mathsf{B}'_{22}}\bm{\tau}^{(2)} &
        \cdots &
        e^{-i\epsilon\mathsf{B}'_{nn}}\bm{\tau}^{(n)} &
      \end{pmatrix}
      = \mathsf{U}_{0}(\mathsf{I}_{n} - i\epsilon\mathsf{B}') + \mathcal{O}(\epsilon^{2})
\end{equation}
%
Under the gauge transformation, the matrices $\mathsf{A} \to \mathsf{A} + \mathcal{O}(\epsilon)$ and $\mathsf{B}'' \to \mathsf{B}'' + \mathcal{O}(\epsilon)$.
Thus, the RHS of Eq.~XXX becomes
%
\begin{equation}
  \left[\mathsf{U}_{0}(\mathsf{I}_{n} - i\epsilon\mathsf{B}') + \mathcal{O}(\epsilon^{2})\right]\left[\mathsf{I}_{n} + \epsilon\left(\mathsf{A} + i\mathsf{B}' + i\mathsf{B}'' + \mathcal{O}(\epsilon)\right)\right]
  =
  \mathsf{U}_{0}\left[\mathsf{I}_{n} + \epsilon\left(\mathsf{A} + i\mathsf{B}'' \right)\right] + \mathcal{O}(\epsilon^{2}).
\end{equation}
%
In other words, we can obsorb any nonzero diagonal entries of $\mathsf{B}$ by a suitable rephasing of the columns of $\mathsf{U}_{0}$.
Thus, without loss of generality we take the diagonal part $\mathsf{B}'$ to be zero.

\begin{example}[A simple ``coupled'' differential operator]
Consider an operator $\hat{\mathsf{D}}$ (with symbol $\mathsf{D}$) defined by
%
\begin{equation}
  \hat{\mathsf{D}} =
  \begin{pmatrix}
    \hat{p} & -\lambda\\
    -\lambda & \hat{p}
  \end{pmatrix}
  \qquad\text{and}\qquad
  \mathsf{D} =
  \begin{pmatrix}
    p & -\lambda\\
    -\lambda & p
  \end{pmatrix}.
\end{equation}
%
The wave equation $\hat{\mathsf{D}}\Psi = 0$ defined by this operator can be trivially shown to be equivalent to the uncoupled ODEs $\partial_{x}^{2}\Psi_{\alpha} + \lambda^{2}\Psi_{\alpha} = 0$ in components $\Psi_{\alpha}$ of $\Psi$.
Solving this ODE, we find $\Psi_{1} = A_{+}e^{i \lambda x} + A_{-}e^{-i\lambda x}$ and $\Psi_{2} = A_{+}e^{i \lambda x} - A_{-}e^{-i\lambda x}$.
%
By diagonalizing the symbol matrix $\mathsf{D}$, we find $\mathsf{U}$ and $\Lambda$, which can be transformed back to find the diagonalized operator $\hat{\Lambda}$:
%
\begin{equation}
  \mathsf{U} = \frac{1}{\sqrt{2}}
  \begin{pmatrix}
    1 & 1\\
    1 & -1
  \end{pmatrix},\enspace
  \Lambda =
  \begin{pmatrix}
    p - \lambda & 0\\
    0 & p + \lambda
  \end{pmatrix},\enspace
  \text{and}\enspace
  \hat{\Lambda} =
  \begin{pmatrix}
    \hat{p} - \lambda & 0\\
    0 & \hat{p} + \lambda
  \end{pmatrix}.
\end{equation}
%
The uncoupled system given by $\hat{\Lambda}\Phi = 0 $ has solutions $\Phi_{1} = B_{+}e^{i\lambda x}$ and $\Phi_{2} = B_{-}e^{-i\lambda x}$.
The solution to the original system can be recovered by $\Psi = \hat{\mathsf{U}}\Phi$, which gives us\footnote{The operator $\hat{\mathsf{U}}$ is equal to its symbol $\mathsf{U}$ since its matrix entries are constants.}
%
\begin{equation}
  \Psi = \frac{1}{\sqrt{2}}
  \begin{pmatrix}
    1 & 1\\
    1 & -1
  \end{pmatrix}
  \begin{pmatrix}
    B_{+}e^{i\lambda x}\\
    B_{-}e^{-i\lambda x}
  \end{pmatrix}
  =
  \begin{pmatrix}
    A_{+}e^{i\lambda x} + A_{-}e^{-i\lambda x}\\
    A_{+}e^{i\lambda x} - A_{-}e^{-i\lambda x}
  \end{pmatrix},
\end{equation}
%
where we have set $A_{\pm} = B_{\pm}/\sqrt{2}$, and have found the expected solution.
\end{example}



\ifsustyle
  %! TEX root = thesis.tex
% vim: ft=tex et sts=2 sw=2


\chapter*{Curriculum Vit\ae}
\addcontentsline{toc}{chapter}{Curriculum Vit\ae}
\thispagestyle{empty}

\def\hangpar{\noindent\hangindent=2em}

\subsection*{Education}

\hangpar Syracuse University, Syracuse, New York, USA; Aug.~2017--Aug.~2023.\\
Ph.D.~in Physics, August~2023.

\hangpar Indian Institute of Technology Kanpur, Kanpur, Uttar Pradesh, India; Jul.~2009--Jul.~2014. Integrated M.Sc.~in Physics, Feb.~2015.

\subsection*{Employment}

\hangpar Research and/or Teaching Assistant, Syracuse University, Aug.~2017--Aug.~2023.

\hangpar Project Associate, Indian Institute of Technology Kanpur, Sep.~2014--Oct.~2015.

% \subsection*{Publications}

% \begin{flexlabelled}{*}{1em}{0.5em}{0em}{1.5em}{0em}
%   \providecommand{\doix}[2]{\href{https://dx.doix.org/#1}{#2}}
%   \providecommand{\arxiv}[2]{\href{https://arxiv.org/abs/#1}{arXiv:#1 [#2]}}
%   \setlength{\itemsep}{-0.25em}
%   \item[5.] M.~Mannattil, J.~M.~Schwarz, and C.~D.~Santangelo, ``Thermal Fluctuations of Singular Bar-Joint Mechanisms,'' \arxiv{2112.04279}{cond-mat.soft}.
%   \item[4.] M.~Mannattil, A.~Pandey, M.~K.~Verma, and S.~Chakraborty, ``On the applicability of low-dimensional models for convective flow reversals at extreme Prandtl numbers,'' \doix{10.1140/epjb/e2017-80391-1}{Eur.~Phys.~J.~B~\textbf{90}, 259~(2017)}, \arxiv{1711.01510}{physics.flu-dyn}.
%   \item[3.] M.~Mannattil, H.~Gupta, and S.~Chakraborty, ``Revisiting Evidence of Chaos in X-ray Light Curves: The Case of GRS~1915+105,'' \doix{10.3847/1538-4357/833/2/208}{Astrophys.~J.~\textbf{833}, 208~(2016)}, \arxiv{1611.02264}{astro-ph.HE}.
%   \item[2.] A.~Tandon, M.~Schr\"{o}der, M.~Mannattil, M.~Timme, and S.~Chakraborty, ``Synchronizing noisy nonidentical oscillators by transient uncoupling,'' \doix{10.1063/1.4959141}{Chaos \textbf{26}, 094817~(2016)}, \arxiv{1611.02298}{nlin.CD}.
%   \item[1.] M.~Schr\"{o}der, M.~Mannattil, D.~Dutta, S.~Chakraborty, and M.~Timme, ``Transient Uncoupling Induces Synchronization,'' \doix{10.1103/PhysRevLett.115.054101}{Phys.~Rev.~Lett.~\textbf{115}, 054101~(2015)}, \arxiv{1508.06545}{nlin.CD}.
% \end{flexlabelled}

\fi
% %! TEX root = thesis.tex
% vim: ft=tex et sts=2 sw=2

\chapter{Comments and TODO}

\section{Comments on Paper 1}

\begin{enumerate}
  \item In the derivation for regular points, it's not readily clear that the matrix in the exponential would be positive definite.  But this is so since it's the Hessian of a function ($=U(q) + |\hat{\vartheta}(q) - \vartheta|^{2}$) which has a minimum at $q_{i}$.
\end{enumerate}

\section{Comments on Paper 2}

\begin{enumerate}

  \item What's the connection between the Faddeev--Popov method and the coarea formula?  See Zee's book on QFT.
  \item Language: point \emph{in} manifold or point \emph{on} manifold? N.B. a manifold is a set.
  \item I didn't use the constant-rank theorem since that'd require more explanation, e.g., there could be rows of the Jacobian that are \emph{not} independent if the requirement is only constant rank.  Perhaps we can add that preimage theorem, constant-rank theorem, etc., are variations of the implicit function theorem.
  \item I'm being purposefully sloppy in the definition of the tangent space.
    The vectors $\bm{v}$ should be picked from $T_{\bm{q}} \mathcal{Q}$ (in which case I'll need to define that first) and not $\mathbb{R}^n$.
    Here it's okay since $\mathcal{Q}$ is always considered as a submanifold of $\mathbb{R}^n$, which makes $T_{\bm{q}}\mathcal{Q} = \mathbb{R}^n$.
  \item What's the guarantee that a rank deficiency leads to a bifurcation?  Also, rank-deficiency singularities could be parameterization singularities.  I think CS singularities require rank deficiency, but rank deficiency doesn't always guarantee CS singularity.
  \item A zero mode is technically a motion on the tangent bundle $TM$ since you want $q \in \Omega$ and $v \in T_{q}\Omega$.
  \item Even though $q$ and $v$ belong to different spaces we are taking $q \to q + v$.  Although I'm confident that there's nothing wrong with this, I want to understand this better.  Goes back to CM where we take $q \approx q_0 + v t$.
  \item What's the connection between energy near a singularity and the ``affine energy'' in Lubensky et al.'s RPP review (Eq.~3.10)?
  \item Even though zero modes corresponding to flexes don't cost energy, we still have $x\trans\hess f_ix \neq 0$.  This is because flex is ``nonlinear''.  However, the partition function calculation is still fine because the projection operator will kill these nonzero terms.
  \item In our discussion, we have focused on self stress states of points that belong to the constraint manifold.
  However, we should note all points $q \notin \Omega$ for which $C(q)$ drops rank, also admit self stress states.
  Discuss issues, e.g., drop in rank doesn't guarantee singularity, etc.
  \item Can you call an ordinary function positive definite or positive semidefinite?
  \item If $A + B = C$, where $A, B, C$ are vector spaces, does one say that $A$ and $B$ spans $C$ or $A$ and $B$ generates $C$?  Or are both wrong?
  \item Plot $\beta F(\xi)$ instead of saying that you're plotting free-energy in units of $\beta^{-1}$.
  \item Histogram method for free energy is also called ``visited states method''.
  \item What's the guarantee that states of stress (and singularities) are gauge invariant w.r.t. the local body frame of the mechanism?
    Gauge invariance in the sense of Littlejohn's papers on few-body systems.
  \item Adjacency matrix = sign(compatibility matrix)?
  \item To write the partition function as the Laplace's transform of the density of states, shouldn't the energy function $U(\bm{q})$ foliate $\mathbb{R}^{n}$? Coarea formula.

  \item Out of plain buckling of thin plates, Foppel-von Karman limit, stresses.

    Strain tensor (Audoly's book Eq.~6.60) for a displacement field $(u_{1}, u_{2}, f)$ is
    \begin{equation}
      \epsilon_{ij} = \frac{1}{2}\left(\partial_{i} u_{j} + \partial_{j} u_{i} + \partial_{i}f \partial_{j}f\right)
    \end{equation}
    The first two terms is equivalent to $\mathsf{C}\bm{v}$ and the last quadratic term is $\bm{w}(\bm{u})$.
    Mechanical equilibrium requires (Eq.~6.64 of Audoly's book):
    \begin{equation}
      \partial_{i}\partial_{j} f \sigma_{ij} = 0.
    \end{equation}
    Is this equivalent to the Fredholm alternative $\bm{\sigma}\cdot\bm{w}(\bm{u}) = 0$?
    Also in the strain $\partial_{i}f \partial_{j} f$ has 3 independent components, whereas there are only two independent displacement fields, namely, $u_{1}$ and $u_{2}$.  Thus, for a given set of $u$, an arbitrary $f$ will not solve $\epsilon_{ij} = 0$.  Only those satisfying $K = 0$ will work (i.e., isometries).
    How is this related to the Fredholm alternative, and the solvability of the constraint map equation?
\end{enumerate}

\section{Waves}

\begin{enumerate}
  \item Parity of the filament operator.%
    \footnote{%
    More quantum mechanically, this can be shown by considering the commutation of $\widehat{\mathsf{D}}$ with the operators $\mathsf{P}_{\pm} = \diag\left(\hat{\pi}, \pm\hat{\pi}\right)$, where $\hat{\pi}$ is the usual parity operator~\cite{cohen-tannoudji2019}.
      Clearly, $\widehat{\mathsf{P}}_{\pm}\Psi(x) = \left[\zeta(-x), \pm u(-x)\right]$ so that
      the eigenstates of $\mathsf{P}_{+}$ always have $\zeta(x)$ and $u(x)$ of the same parity, whereas the eigenstates of $\mathsf{P}_{-}$ always have $\zeta(x)$ and $u(x)$ of different parity.
      Furthermore, for odd and even $m(x)$, we can show that $\widehat{\mathsf{D}}$ commutes with $\widehat{\mathsf{P}}_{+}$ and $\widehat{\mathsf{P}}_{-}$, respectively.
      As commuting operators share the same eigenstates (assuming nondegeneracy), this proves the claim made above.%
    }
\end{enumerate}

\subsection{Crap}

Applying the rank-nullity theorem to the compatibility matrix, we arrive at the Maxwell--Calladine theorem.\footnote{Originally devised by Maxwell in the context of tensegrity structures.}
%
\begin{theorem}[Maxwell--Calladine index theorem]
  Given a mechanism with $n$ degrees of freedom and $m$ constraints in a state of self stress, the number of zero modes $z$ and the number of self stresses $s$ satisfy the relation
  \begin{equation*}
    n - m = z - s.
  \end{equation*}
\end{theorem}

\section{Unused}

\subsection{Effect of pinning the vertices}

The number of self stresses depend strongly on the number of pinned vertices -- a square with an internal vertex, but pinned corners has two states of self stress.  The same square has only one state of self stress if the corners are not pinned.
This can be physical explained on the basis of force balance.

\begin{figure}
  \begin{center}
    \includegraphics{unused/origami2/selfstress.pdf}
  \end{center}
  \caption{
    Two independent states of self stress in an origami with two internal vertices.
    The lengths of the self stress arrows on the $i$th edge is proportional to $\sqrt[4]{\sigma_{i}}$.
  }
  \label{fig:origami2_selfstress}
\end{figure}
%
\begin{figure}
  \begin{center}
    \includegraphics{unused/pinned/pinned.pdf}
  \end{center}
  \caption{
    \textsf{\textbf{(a)}} A quadrilateral framework with one self stress.
    \textsf{\textbf{(b)}} The same framework has two independent self stresses when the corners are pinned.
  }
  \label{fig:quad_pinned}
\end{figure}

\begin{figure}
  \begin{center}
    \includegraphics{unused/zerodof.pdf}
  \end{center}
\caption{A zero \ac{dof} linkage without self stress.  Note how the two constraint manifolds $\mathcal{M}_1$ and $\mathcal{M}_2$ are transverse to each other.}
  \label{fig:hello}
\end{figure}


\begin{figure}
  \begin{center}
    \includegraphics{unused/zerodof_spring.pdf}
  \end{center}
\caption[foo]{A zero \ac{dof} linkage without self stress.  Note how the two constraint manifolds $\mathcal{M}_1$ and $\mathcal{M}_2$ are transverse to each other.}
  \label{fig:hello2}
\end{figure}
\pagebreak

\section{Energy (unused)}
\label{sec:Energy (unused}


The potential energy of the entire framework, at a point $\bar{\bm{q}} \in \Sigma$, under a general perturbation $\bar{\bm{q}} \to \bar{\bm{q}} + \bm{v}$, with $\bm{v} \in \mathbb{R}^n$, to the lowest order in $\bm{v}$ is
\begin{equation}
  U(\bm{v}) \approx \frac{1}{2}\bm{v}\trans\mathsf{C}\trans\mathsf{K}K\mathsf{C}\bm{v} = \frac{1}{2}\bm{v}\trans\mathsf{D}D\bm{v}\,,\label{eq:energy1}
\end{equation}
where $\mathsf{K}K = \diag(K_1, K_2, \ldots, K_m)$ is the $m\times m$ diagonal matrix of spring constants $K_i = \phi_i''(\bar{\ell_i})$ and $\mathsf{D}D = \mathsf{C}\trans\mathsf{K}K\mathsf{C}$ is the $n\times n$ dynamical matrix.
Since all the potential functions $\phi_i$ have a minimum at $\ell_i = \bar{\ell_i}$, the spring constants $K_i$ are positive nonzero numbers and the matrix $\mathsf{K}K$ is positive definite.
However, the dynamical matrix $\mathsf{D}D$ is, in general, only positive semidefinite since $\ker{\mathsf{C}}$ need not be empty.

If point $\bar{\bm{q}} \in \Sigma$ does not admit a state of self stress, then the dynamical matrix $\mathsf{D}D$ has $z = n - m$ zero eigenvalues, each corresponding to a zero mode.
The other $n - z = m$ eigenvalues of $\mathsf{D}D$ are nonzero and correspond to the normal modes of the spring framework.
Essentially, the number of normal modes is the codimenison of the manifold $\Sigma$.
Since $\Sigma$ is a smooth manifold at $\bar{\bm{q}}$, we can identify the $z$-dimensional space of zero modes with the tangent space $T_{\bar{\bm{q}}}\Sigma$ at that point.
Similarly, the $m$-dimensional space of normal modes can be identified with the normal space $N_{\bar{\bm{q}}}$.
% Each zero mode in $X$ corresponds to a smooth deformation of the framework that does not change spring lengths and hence do not cost energy to \emph{any order}.
% However, as in Section~TODO, we are often interested in the energy change due perturbations that change the spring lengths.
% In other words, the perturbations that move the point $\bm{q}$ away from the shape space $\Sigma$.
% Thus, we focus on perturbations $\ybm \in Y = N_{\bm{q}}\Sigma$, in which case the energy to the lowest order is
% \begin{equation}
%   U(\ybm) \approx \frac{1}{2}\ybm \mathsf{D}D \ybm\,.
% \end{equation}

On the other hand, if the framework admits self-stress states at a point $\bar{\bm{q}} \in \Sigma$, then $\mathsf{C}$ is not full rank and there is no well defined tangent space or normal space at $\bar{\bm{q}}$.
Nonetheless, one can still define the $z$-dimensional subspace of the zero modes as $\ker{\mathsf{C}}$.
From Eq.~\eqref{eq:mcindex} we have $z = n - m + s$, with $s$ being the dimension of the subspace of self stresses.
Similarly, the 2space of normal modes is $(\ker \mathsf{C})^\perp$, the orthogonal complement of $\ker{\mathsf{C}}$ in $\mathbb{R}^n$ under the standard Euclidean dot product.
Clearly, the number of normal modes is $n - z = m - s$.

However, unlike the case without self stresses, now, zero modes in do not \emph{always} correspond to deformations of the framework that preserve the spring lengths.
Therefore, they \emph{do} contribute to the energy at higher orders and one has to analyze their contributions more systematically.
We can decompose any general perturbation $\bm{v} \in \mathbb{R}^n$ as

\section{Introduction}

A \emph{framework} can be broadly defined as a mechanical system comprised of rigid parts that move under constraints.%
\footnote{There is some inconsistency in the definition of a framework.
  For instance, in most engineering contexts~\cite{hartenberg1964,hunt1978,myszka2012}, a framework is considered to be a subelement of a larger machine, or is synonymous with it.
  On the other hand, some authors~\cite{connelly2022} often define a framework to be a specific deformation of a mechanical system allowed by its constraints, e.g., a rotor with two degrees of freedom and one constraint is said to possess one framework.
  In this thesis, we prefer the engineering definition and a framework always refers to a mechanical system or its subelements, and not its individual motions.}
A framework could something simple like a linear rotor to something complex like an internal-combustion engine.
A large class of frameworks are modeled as frameworks comprising of joints connected by rigid bars.

\section{Rigidity theory}

See the notes by \citet{connelly2022} for an introduction to tensegrity structures and the primer by \citet{williams2003} for a slightly advanced treatment.

Forces involved in a state of self stress obey the strong form of Newton's third law and thus cannot possibly result in an unbalanced torque \cite[\S 1.2]{goldstein2002}.
Thus, a tensegrity under self stress is in a state of mechanical equilibrium.
The equilibrium may or may not be stable: again, think of the example with a particle tethered to two walls using springs that are under compression (unstable) or under elongation (stable).

Note that SS exists outside of tensegrity structures.
The only requirement is that all particles interact via central forces.
E.g., one can think of electrostatic analogies, or sticky colloidal clusters.

SS is caused due to linear dependence of constraints.
They might be independent nonlinearly, but on a linear level they are dependent.
Linear constraints are, simply put, hyperplanes.

Maxwell--Calladine theorem is a finite-dimensional toy ``index'' theorem~\cite[\S 2.2]{nakahara2003}.

See the thesis \cite{lengyel2002}.

\subsection{Four-bar linkage}

Originally analyzed by Franz Grashof~\cite[pp.~113--118]{grashof1883}.

\section{Self stress}

Note that sometimes it is customary to define self stresses as belonging to the cokernel of the rigidity matrix, i.e., $\sigma_{\mathsf{R}}$ satisfying $\mathsf{R}\trans\sigma_{\mathsf{R}} = 0$.
As $\mathsf{C}\trans = \mathsf{R}\trans\mathsf{L}^{-1}$, for every $\sigma \in \coker\mathsf{C}$, we have
%
\begin{equation}
  %\mathsf{\sigma}_{\mathsf{C}}\mathsf{C} = (\sigma_{\mathsf{C}} \mathsf{L}^{-1})\mathsf{R} = 0,
  \mathsf{C}\trans\sigma = \mathsf{R}\trans\left(\mathsf{L}^{-1}\mathsf{\sigma}\right) = 0
\end{equation}
%
so that $\sigma_{\mathsf{R}} = \mathsf{L}^{-1}\sigma$.




\section{Unused: Hard vs. soft constraints}

Trimer discussed in Refs.~\cite{kampen1981,kampen1984} (and also in Refs.~\cite[Section 15.1]{frenkel2001} and \cite{walter2011})

\section{CV under a transformation that is not smooth}

The free energy difference becomes
%
\begin{equation}
  \Delta\free{\xi} = -\beta^{-1}\log\left[\frac{\sqrt{\pi}\left(\abs{\cos\xi'} + \abs{\sin\xi'}\right)}{2 + \sqrt{X}D_{-1/2}(0)\left(1 + Y\right)}\right]
\end{equation}
\begin{equation}
  \begin{aligned}
    \Delta\free{\xi} &= \beta^{-1}\log\left[2 + \sqrt{X}D_{-1/2}(0)\left(1 + Y\right)\right] \\
                     &\qquad -\beta^{-1}\log\Big\{2\abs{\cos\xi'}^{-1} + \sqrt{X}\big[\exp(-X^{2}\xi'^{4})D_{-1/2}(-2X\xi'^{2})\\
                     & \phantom{\qquad-\beta^{-1}\log\Big\{2\abs{\cos\xi'}^{-1} + \sqrt{X}\big[}
                 \quad+Y\exp(-X^{2}Y^{2}\xi'^{4})D_{-1/2}(-2XY\xi'^{2})\big]\Big\}   \end{aligned}
\end{equation}
%
Here $\xi' = \pi\xi/4$.

\subsubsection*{Cone-plane intersection}

A singularity is formed at the origin when a cone $z^2 = x^2 + y^2$ intersects with the $yz$ plane.
However, there's no lowering of the free energy since the intersection isn't a nontransversal intersection.
The only plane tangent to the $yz$ plane is the $yz$ plane itself, which is not tangential to the cone at the origin.
One might object that the cone itself ceases to be a manifold at the origin.
However, one can always ``file off'' the tip of the cone and make it infinitesimally smooth like a physicist would do.
This would still produce no lowering of the free energy.


\appendix

%! TEX root = thesis.tex
% vim: ft=tex et sts=2 sw=2

\chapter{Mathematical miscellanea}
\label{app:math}

In this appendix we collect some useful mathematical results and derivations that are frequently referenced to in this dissertation.
For more detailed discussions, refer to standard references in the theory of manifolds, measure theory, and semiclassical physics.

\section{Manifolds}

A manifold, roughly speaking, a set that looks Euclidean\footnote{It is often said that manifolds are smooth sets that look locally ``flat'' and hence can be mapped to an open set in Euclidean space.
This is a bit misleading as the case of the sphere $S^2$ illustrates.
The entire northern hemisphere (except the north pole) of $S^2$ can be mapped to the real plane $\mathbb{R}^{2}$ using stereographic projection.
This means that the two hemispheres of $S^2$ can be covered entirely using two charts.
But neither hemispheres are flat structures.
Flatness also reminds me of curvature, which isn't required for the definition of a smooth manifold, which is much more primitive structure.}

A \emph{regular point} is a point in the domain $X$ where the Jacobian of the map is full rank.
A \emph{regular value} is a point $y \in Y$ such that the Jacobian of all the points in the preimage $f^{-1}(y) \subset X$ is full rank.
Sard's theorem ensures that almost all points in $Y$ are regular values.
%
\begin{theorem}[Sard's theorem]
  The set of critical values of any smooth map $f: X \to Y$, has zero measure in the codomain $Y$.
\end{theorem}

In other words, almost all points in $Y$ are regular values.
As an example, consider $f: \mathbb{R} \to \mathbb{R}$ defined by $x \mapsto x^3 - x$.
The Jacobian in this case is simply the derivative $f'(x) = 3x^2 - 1$, which vanishes only for $x = 3^{-1/2}$.
Hence, every point other than $x = 3^{-1/2}$ is a regular point.%
\footnote{A word of warning is appropriate here: Sard's theorem does not imply that the set of critical points in the domain $X$ is a measure zero subset.
For example, if we consider a constant map, say $f(x) = c \in Y$, then all points in $X$ are critical points.}

One might be tempted to apply the regular value theorem to the energy function itself.
But this wouldn't work since for $E=\|\bm{f}\|^2$ to vanish, $\bm{f}$ must itself vanish, which in turn would make $\nabla E = 2 \bm{f}\nabla\bm{f} = \bm{0}$, making it impossible to apply the regular value theorem on the energy function.

The dimension of tangent space is commonly called a \ac{dof}.

\begin{theorem}[Preimage theorem]
\end{theorem}

\section{Laplace's method}

To evaluate
%
\begin{equation}
  \int_{a}^{b} \dd{x}\, g(x) e^{-\beta U(x)},
\end{equation}
when $\beta$ is large, we first expand $U(x)$ to $\mathcal{O}(x^{2})$ around a critical point $x_{0}$ with $U'(x_{0}) = 0$.
This turns the above integral into a standard Gaussian integral
%
\begin{equation}
  \int_{a}^{b} \dd{x}\, g(x_{0}) \exp\left\{-\beta\left[U(x_{0}) +  U''(x_{0})(x-x_{0})^{2}\right]\right\},
\end{equation}

\begin{figure}
  % \begin{center}
  %   \includegraphics[scale=1.0]{file}
  % \end{center}
  \caption{Plot of the function $U(x) = 9x^{2} + \sin^{2}{3x} + (1-x)\sin^{2}{6x}$ and $e^{-\beta U(x)}$ in the interval $[-1,1]$ for $\beta = 0.1, 0.2, \ldots, 6.4$.  For larger values of $\beta$ the plot is clearly indistinguishable from that of a Gaussian with a width of $U''(0)$.}
  \label{fig:}
\end{figure}

\subsection{Degenerate case}

\section{Coarea formula}

Jacobian determinants are the corrective factors relating the elements of areas of the domains and images of functions (Morgan's book).

The coarea formula,\footnote{Sometimes the coarea formula is misleadingly written~\cite{hartmann2007,hartmann2007a} with the determinant of $(\nabla\hat{\xi})\trans\nabla\hat{\xi}$ in the denominator.  But the Jacobian $\nabla\hat{\xi}$ does not have full column rank when $m < n$ and the determinant $\det\,(\nabla\hat{\xi})\trans\nabla\hat{\xi}$ vanishes.}
%
\begin{theorem}[Coarea formula]
  Given an integrable function $\phi: \mathbb{R}^n \to \mathbb{R}$ and
  Consider a map $\hat{\xi}: \mathbb{R}^n \to \mathbb{R}^m$ (with $m \leq n$) whose level sets foliate $N \subseteq \mathbb{R}^{n}$ we have
  \begin{equation}
    \begin{aligned}
      \int_{\mathbb{R}^n} \dd{\bm{q}}\, \phi(\bm{q}) &= \int_{\mathbb{R}^m} \dd{\xi}\,\int_{\mathbb{R}^n} \dd{\bm{q}}\, \delta\left[\hat{\xi}(\bm{q}) - \xi\right] \phi(\bm{q})\\
                                                     &= \int_{\mathbb{R}^m} \dd{\xi}\,\int_{\bm{q} \in \hat{\xi}^{-1}(\xi)} \frac{\dd{\Omega(\bm{q})}}{|\det \nabla\hat{\xi}(\nabla\hat{\xi})^T|^{1/2}} \phi(\bm{q})\,,
    \end{aligned}
  \end{equation}
  where $\dd\Omega(\bm{q})$ is the area element on the level set $\hat{\xi}^{-1}(\xi)$.
\end{theorem}
\begin{proof}
  %
  \begin{equation}
    \begin{aligned}
      \int_{\mathbb{R}^{n}} \dd\bm{q}\,\delta\left[\hat{\xi}(\bm{q}) - \xi\right] \phi(\bm{q}) &=
      \lim_{\alpha \to \infty} \left(\frac{\alpha}{2\pi}\right)^{m} \int_{\mathbb{R}^{n}} \dd\bm{q}\, \exp\left(-\tfrac{1}{2}\alpha\Abs{\hat{\xi}(\bm{q}) - \xi}^{2}\right) \phi(\bm{q})
    \end{aligned}
  \end{equation}
  %
  \qed
\end{proof}

Foliation intuitively means that there is a unique hypersurface that passes through each point.

\begin{example}[Integration in polar coordinates]
  As a simple illustration of the coarea formula, consider evaluating the double integral $\int_{\mathbb{R}^{2}} \dd{x}\dd{y}\, \phi(x,y)$, where $\phi(x, y)$ is some integrable function of $(x, y)$.
  The standard polar angle $\theta$ can be computed using the map $\hat{\theta}(x, y) = \tan^{-1}(x, y)$.\footnote{Here $\tan^{-1}(x, y): \mathbb{R}^{2} \setminus \{(0,0)\} \to (-\pi, \pi]$ is the two-argument variant of the inverse tangent, sometimes denoted as $\mathrm{atan2}(y, x)$ in numerical software.  We cannot use $\tan^{-1}(y/x)$ to compute $\theta$ since the range of principal values of $\tan^{-1}(\cdot)$ is conventionally restricted to $(-\pi/2, \pi/2)$.}
  The level set $\hat{\theta}^{-1}(\theta)$ is a straight line starting at the origin (but excluding it) and making an angle of $\theta$ with the positive $x$ axis.
  It is clear that for values of $\theta \in (-\pi, \pi]$, these level sets foliate the entire $\mathbb{R}^{2}$ plane (excluding the origin).
  Choose $\xi = \theta$ and $\hat{\xi}(x, y) = \hat{\theta}(x, y)$ so that
  %
  \begin{equation}
    \int_{\mathbb{R}^{2}} \dd{x}\dd{y}\, \phi(x, y) = \int_{-\pi}^{\pi} \dd\theta\, \int_{\mathbb{R}^{2}} \dd{x}\dd{y}\, \delta[\tan^{-1}(x, y) - \theta] \phi(x, y).
  \end{equation}
  %
  Now, $\nabla\hat{\theta} = \nabla\tan^{-1}(x, y) = (x^{2} + y^{2})^{-1}\begin{pmatrix}-y & x\end{pmatrix}$.
  We can parameterize the points on $\hat{\theta}^{-1}(\theta)$ in terms of $r > 0$ as $(r\cos{\theta}, r\sin{\theta})$.
  Then, $\nabla\hat{\theta} = r^{-1}\begin{pmatrix}-\sin\theta & \cos\theta\end{pmatrix}$ and $\det\,\nabla\hat{\theta}(\nabla\hat{\theta})\trans = r^{-2}$.
  Putting this in the coarea formula and noting that the surface measure on $\hat{\theta}^{-1}(\theta)$ is just $\dd{r}$, we arrive at
  \begin{equation}
    \int_{\mathbb{R}^{2}} \dd{x}\dd{y}\, \phi(x, y) = \int_{-\pi}^{\pi} \dd\theta\, \int_0^{\infty} \dd{r}\, r \phi(r, \theta),
  \end{equation}
  which is the standard double integral of a function expressed in polar coordinates.\footnote{A similar example is discussed in many books on field theory in the context of the Fadeev--Popov method, e.g., the ones by \citet[Section 7.2]{ryder1996} and \citet[Part III.4]{zee2010}.}
  \altqed
\end{example}

Other examples: density of states.  See paper~\cite{gillespie1983}.

\section{Operators and symbols}

An operator-symbol correspondence is an association between operators defined on some Hilbert space and ordinary $c$-numbered functions on the phase space, called symbols~\cite[\S 2.3.1]{chaichian2001}.

The symmetrized product $\big\llbracket \widehat{A}_{1}^{k_{1}} \widehat{A}_{2}^{k_{2}} \cdots \widehat{A}_{n}^{k_n} \big\rrbracket$ of $n$ noncommuting operators $\widehat{A}_{i}$ is defined as the coefficient of
%
\begin{equation}
  \frac{k!}{k_{1}!k_{2}!\cdots k_{n}!} a_{1}^{k_{1}} a_{2}^{k_{2}} \cdots a_{n}^{k_{n}}
\end{equation}
%
in the multinomial expansion of $\left(a_{1}\widehat{A}_{1} + a_{2}\widehat{A}_{2} + \cdots + a_{n}\widehat{A}_{n}\right)^{k}$ with $k = k_{1} + k_{2} + \cdots + k_{n}$.

\subsection{Weyl symbols}

The Weyl symbol of an operator $\widehat{A}$ is defined by
%
\begin{equation}
  A(x, k) = \int \dd{s}\, e^{-iks/\epsilon} \Bra{x + \tfrac{1}{2}s}\widehat{A}\Ket{x - \tfrac{1}{2}s}.
  \label{app:eq:weyl_def}
\end{equation}
%
A consequence of the above definition is that if operator $\widehat{A}$ is Hermitian with $\widehat{A}^{\dagger} = \widehat{A}$, then, taking the complex conjugate of Eq.~\eqref{app:eq:weyl_def}, we find
%
\begin{equation}
  A^{*}(x, k) = \int \dd{s}\, e^{+iks/\epsilon} \Bra{x + \tfrac{1}{2}s}\widehat{A}\Ket{x - \tfrac{1}{2}s}^{*}
  = \int \dd{s}\, e^{+iks/\epsilon} \Bra{x - \tfrac{1}{2}s}\widehat{A}\Ket{x + \tfrac{1}{2}s} = A(x, k),
\end{equation}
%
showing that the symbol $A(x, k)$ is a real function of $x$ and $k$.

\begin{example}
  For a one-dimensional operator $\widehat{A} = a\hat{x}^{n}$, with $a \in \mathbb{C}$ and $n \in \mathbb{Z}$, on making use of $\bra{x + \frac{1}{2}s}a\hat{x}^{n}\ket{x - \frac{1}{2}s} = a \left(x - \frac{1}{2}s\right)^{n}\braket{x + \frac{1}{2}s | x - \frac{1}{2}s} = a\left(x - \frac{1}{2}s\right)^{n} \delta(s)$ in Eq.~XXX, we find $A = a x^{n}$.
  This also means that the operator $\widehat{A} = \sum_{n} a_{n}\hat{x}^{n}$, which is a polynomial in $\hat{x}$, with $a_{n}$ being the coefficients, has the symbol
  $A = \sum_{n} a_{n}{x}^{n}$, an ordinary polynomial in $x$.
  More generally, in $d$ dimensions, if $\widehat{A}$ is a multivariate polynomial in the components of $\hat{\bm{x}}$, i.e., $\widehat{A} = \sum_{n_{i}} a_{n_{1}n_{2}\cdots n_{d}} \hat{x}_{1}^{n_{1}}\hat{x}_{2}^{n_{2}}\cdots \hat{x}_{d}^{n_{d}}$, then the corresponding Weyl symbol is $A = \sum_{n_{i}} a_{n_{1}n_{2}\cdots n_{d}} {x}_{1}^{n_{1}}{x}_{2}^{n_{2}}\cdots {x}_{d}^{n_{d}}$.%
  \footnote{Here, we could have alternatively considered the operator $\sum_{n} a_{n} \hat{\bm{x}}^{n}$, but raising a vector operator to a power is not readily obvious.
    In any case, the generalization we have considered does cover common cases such as $\widehat{A} = \hat{\bm{x}}\cdot\hat{\bm{x}}$.
  In this case, in two dimensions, for instance, the polynomial coefficients will all be zero except for $a_{20} = 1$ and $a_{02} = 1$, and the Weyl symbol would trivially be $x_{1}^{2} + x_{2}^{2}$.}
\end{example}

\section{Star product of symbols}

How can we express the symbol of the product of two operators in terms of their individual symbols?
For example, if $\widehat{C} = \widehat{A}\widehat{B}$, then how is the symbol $C$ related to the symbols $A$ and $B$?
As $\widehat{A}$ and $\widehat{B}$ are general noncommuting operators, there is no apriori reason to assume that $C = AB$, and indeed $C \neq AB$, in general.
%
\begin{equation}
  C(x, k) = 2\int \dd{x'}\, \dd{x''} \,e^{-2ikx} \bra{x + x'}\widehat{A}\ket{x''}\bra{x''}\widehat{B}\ket{x - x'}.
\end{equation}
%
Above we have set $x' \to 2x'$ in the usual Wigner transform expression.
Using the Weyl transform to express the matrix elements $\bra{x + \frac{1}{2}x'}\widehat{A}\ket{x''}$ and $\bra{x''}\widehat{B}\ket{x - \frac{1}{2}x'}$ in terms of the symbols $A$ and $B$, we find
%
\begin{equation}
  \begin{aligned}
    C(x, k) = \frac{2}{(2\pi)^{2}}& \bigg\{\int \dd{x'}\, \dd{x''}\, \dd{k_{1}}\, \dd{k_{2}}\,e^{-2i[kx - k_{1}(x + x' - x'') - k_{2}(-x + x' + x'')]}\\
                                           &\qquad\times A\left[\tfrac{1}{2}(x + x' + x''), k_{1}\right]\, B\left[\tfrac{1}{2}(x - x' + x''), k_{2}\right]\bigg\}.
  \end{aligned}
\end{equation}
%
Setting $x_{1} = \frac{1}{2}(x + x' + x'')$ and $x_{2} = \frac{1}{2}(x - x' + x'')$ and changing variables, we find\footnote{An extra Jacobian factor equal to 2 appears while transforming $(x', x'') \to (x_{1}, x_{2})$~\cite[Eq.~(2.3.23) and Problem~2.3.8]{chaichian2001}}
%
\begin{equation}
  \begin{aligned}
    C(x, k) &= \frac{1}{\pi^{2}} \int \dd{x_{1}}\, \dd{x_{2}}\, \dd{k_{1}}\, \dd{k_{2}}\,e^{-2i[(k_{1} - k)(x_{2} - x) - (k_{2} - k)(x_{1} - x)]} A(x_{1}, k_{1})\, B(x_{2}, k_{2})\\
                     &= \frac{1}{\pi^{2}} \int \dd{x_{1}}\, \dd{x_{2}}\, \dd{k_{1}}\, \dd{k_{2}}\,e^{2i(k_{2}x_{1} - k_{1}x_{2})}\,e^{2i(kx_{2} - k_{2}x)} A(x_{1}, k_{1})\, B(x + x_{2}, k + k_{2}).\label{eq:moyal1}
  \end{aligned}
\end{equation}
%
In the last step, we have set $x_{2} \to x + x_{2}$ and $k_{2} \to k + k_{2}$, with the hope of Taylor expanding $B$ around $(x, k)$.
Note that
%
\begin{equation}
  \begin{aligned}
    B(x + x_{2}, k + k_{2}) &= B(x, k) + [\partial_{x} B(x, k)]x_{2} + [\partial_{k} B(x, k)] k_{2} + \tfrac{1}{2} [\partial^{2}_{x} B(x, k)] x_{2}^{2}\\
    &\qquad + \tfrac{1}{2} [\partial^{2}_{k} B(x, k)] k_{2}^{2} + [\partial_{x}\partial_{k} B(x, k)] x_{2}k_{2} + \cdots \\
    &= \left[1 + x_{2}\partial_{x} + \tfrac{1}{2}x_{2}^{2}\partial^{2}_{x} + \cdots \right]\,
  \left[1 + k_{2}\partial_{k} + \tfrac{1}{2}k_{2}^{2}\partial^{2}_{k} + \cdots \right] B(x, k)\\
    &= e^{x_{2}\partial_{x}} e^{k_{2}\partial_{k}} B(x, k).\label{eq:moyal2}
  \end{aligned}
\end{equation}
%
Above, we have made use of the fact that the partial derivatives are with respect to $x$ and $k$ so that $x_{2}$ and $k_{2}$ can be treated as constants that can be moved around.
To simplify this further, we introdue new notation: the operators $\ogets{\partial_{x}}$ and $\ogets{\partial_{k}}$ act \emph{only} on the terms to their left and operators $\oto{\partial_{x}}$ and $\oto{\partial_{k}}$ act \emph{only} on the terms to their right.
Then,
%
\begin{equation}
  \begin{aligned}
    e^{-2ik_{2}x} e^{\frac{i}{2}\ogets{\partial_{x}}\oto{\partial_{k}}} &= e^{-2ik_{2}x} \left[1 +  \tfrac{i}{2}\ogets{\partial_{x}}\oto{\partial_{k}} - \tfrac{1}{8}\ogets{\partial^{2}_{x}}\oto{\partial^{2}_{k}} + \cdots \right]\\
                                                            &= e^{-2ik_{2}x} \left[1 +  k_{2}\oto{\partial_{k}} + \tfrac{1}{2}k_{2}^{2}\oto{\partial^{2}_{k}} + \cdots \right]\\
                                                            &= e^{-2ik_{2}x}e^{k_{2}\partial_{k}}
  \end{aligned}\label{eq:moyal3}
\end{equation}
%
Similarly, we see that
%
\begin{equation}
  e^{2ikx_{2}}e^{-\frac{i}{2}\ogets{\partial_{k}}\oto{\partial_{x}}} = e^{2ikx_{2}}e^{x_{2}\partial_{x}}.\label{eq:moyal4}
\end{equation}
%
Using Eqs.~\eqref{eq:moyal2}--\eqref{eq:moyal4}, and defining $\hat{\mathcal{L}} = \ogets{\partial_{x}} \oto{\partial_{k}} - \ogets{\partial_{k}}\oto{\partial_{x}}$, we can turn Eq.~\eqref{eq:moyal1} into
%
\begin{equation}
  \begin{aligned}
    C(x, k) &= \frac{1}{\pi^{2}} \left[\int \dd{x_{1}}\, \dd{k_{1}}\, \dd{x_{2}}\, \dd{k_{2}}\,e^{2i[k_{2}(x_{1} - x) - x_{2}(k_{1} - k)]} A(x_{1}, k_{1})\, e^{\frac{i}{2}\hat{\mathcal{L}}}\right] B(x, k)\\
                     &= \left[\int \dd{x_{1}}\, \dd{k_{1}}\, \delta(x_{1} - x)\,\delta(k_{1} - k) A(x_{1}, k_{1})\, e^{\frac{i}{2}\hat{\mathcal{L}}}\right] B(x, k)\\
                     &= A(x, k)e^{\frac{i}{2}\hat{\mathcal{L}}} B(x, k).
  \end{aligned}
\end{equation}
%
In the first step above, we have put brackets around the integral (which is now an operator) to emphasize that $B(x, k)$ can be taken outside it.
The final step defines the Moyal (or ``star'' $\star$) product, which can be used to compute the symbol of the product of two operators in terms of their respective symbols:
%
\begin{equation}
  \begin{aligned}
    C(x, k) = A(x, k)\star B(x, k) &= A(x, k)e^{\frac{i\epsilon}{2}\hat{\mathcal{L}}} B(x, k)\\
                                                             &= A(x, k)\left[1 + \frac{i\epsilon}{2}\left(\ogets{\partial_{x}}\oto{\partial_{k}} - \ogets{\partial_{k}}\oto{\partial_{x}}\right) + \mathcal{O}(\epsilon^{2}) \right] B(x, k)\\
                                                             &= A(x, k)B(x, k) + \frac{i\epsilon}{2}\left\{A(x, k), B(x, k)\right\} + \mathcal{O}(\epsilon^{2}).
  \end{aligned}
\end{equation}


% Bibliography ---------------------------------------------------------

% Use * as the footnote symbol.
\setcounter{footnote}{0}
\def\thefootnote{\fnsymbol{footnote}}

% BibTeX version in draft mode.
\ifdraftdoc
  \bibliographystyle{apsrev4-1}
  \setlength{\bibsep}{0pt plus 0.3ex} % compress bibliography
  \def\thefootnote{\fnsymbol{footnote}}
  \def\bibpreamble{{\normalsize \emph{Note.---} All references that are publicly available on the Internet have archival versions at the Wayback Machine.%
    \footnote{\url{https:/web.archive.org}}}}
  \small
  \bibliography{library}
% BibLaTeX version only if not in draft mode.
\else
  \defbibnote{myprenote}{\hypersetup{allcolors=general}% without this, footnote link will be in nord11.
    \emph{Note.---} All references that are publicly available on the Internet have archival versions at the Wayback Machine.\footnote{\url{https:/web.archive.org}}%
    \vspace\baselineskip
  }
  \hypersetup{linkcolor=backref} % color of back references
  \def\bibfont{\small}
  \setlength\bibitemsep{0pt}
  \printbibliography[prenote=myprenote]
\fi

% Colophon -------------------------------------------------------------

\ifsustyle
  \relax
\else
  \ifdeadtree
    \cleardoublepage
    \cleartooddpage
  \else
    \newpage
  \fi
  \thispagestyle{empty}
  \centering
  \phantom{}
  \vfill
  {\normalsize\sffamily\bfseries Colophon}\\[1.5em]
  \begin{minipage}{0.75\textwidth}
    \fussy
    \small This dissertation was produced using \LaTeX---a set of macros for the {\TeX} typesetting system---and the \emph{memoir} class.
    Figures and plots were created using a combination of Inkscape, Mathematica, and Matplotlib.
    The body text is set in 10~pt \emph{Utopia} and the text in the figures is set in 8~pt (and occasionally 7~pt) \emph{{\TeX} Gyre Heros}, a free clone of \emph{Helvetica}.
    Chapter and section headings are set in \emph{Inter} and mathematics is set in Fourier-GUT\emph{enberg}.\\\\
    Much time was saved during the writing process thanks to several free software projects including
    BibTool,
    Debian GNU/Linux,
    Exuberant Ctags,
    Gimp,
    Git,
    ImageMagick,
    Inkscape,
    latexmk,
    Matplotlib,
    optipng,
    pdfsizeopt,
    \textsc{pdf}\TeX,
    pdftk,
    Python,
    qpdfview,
    \TeX~Live,
    Vim,
    and Vim\TeX.
    Much time was lost yak shaving some of these tools to perfection.
  \end{minipage}
  \vfill

\small
\emph{The lyf so short, the craft so longe to lerne.}
\fi

\end{document}
