%! TEX root = thesis.tex
% vim: ft=tex nospell et sts=2 sw=2

Following the usual convention in physics literature, vectors in $\mathbb{R}^{n}$ are usually set in bold Latin letters, e.g., $\bm{v}$.
Matrices are set in \textsf{sans serif} type, e.g., $\mathsf{M}$.
Operators are distinguished with a hat, e.g., $\hat{a}$.
Integrals without explicit limits are to be integrated over the entire range (usually $-\infty$ to $\infty$) of the integration variable(s).
%Indices are usually $i, j, k, l, m, n, \ldots$, unless there is a chance of confusion with imaginary number $i$, in which case we start from $j$ or use Greek letters.
Unless explicitly indicated, repeated indices are to be summed over as usual.
Other notational conventions are listed below.\\

% TODO: sort greek/latin letters.
\begin{tabular}{ll}
  $\Abs{\bm{v}}$ & Euclidean norm $\sqrt{\bm{v}\trans\bm{v}}$ of a vector $\bm{v} \in \mathbb{R}^{n}$\\
  $\mathsf{I}_n$ & $n\times n$ identity matrix\\
  $\nabla \phi$ & gradient of $\phi: \mathbb{R}^n \to \mathbb{R}$ considered as a row vector, or\\
                & the $m\times n$ Jacobian matrix of a map $\phi: \mathbb{R}^n \to \mathbb{R}^m$\\
  $(\nabla\phi)\trans$ & transpose gradient of $\phi: \mathbb{R}^n \to \mathbb{R}$ considered as a column vector, or\\
                & the $n\times m$ transpose of the Jacobian matrix of a map $\phi: \mathbb{R}^n \to \mathbb{R}^m$\\
  $\hess \phi$ & $n\times n$ Hessian matrix of a scalar function $\phi: \mathbb{R}^n \to \mathbb{R}$\\
  $\det\mathsf{M}$ & determinant of matrix $\mathsf{M}$\\
  $\mathcal{O}(\cdot)$ & of the order of\\
  $\mathsf{C}$ & compatibility matrix\\
  $\bm{\sigma}$ & self stress $\in \ker\mathsf{C}\trans$\\
  $\beta$ & inverse temperature (usually)\\
  $\xi$ & collective variable\\
  $\mathscr{P}(\xi)$ & marginal density of a collective variable $\xi$\\
  $\mathscr{A}(\xi)$ & free energy of a collective variable $\xi$\\
  $k$, $l$ & wave number\\
  $\eta$ & Poisson's ratio (usually)
\end{tabular}

\section*{Abbreviations}

\def\aclabelfont#1{\textsc{\MakeLowercase{#1}}}

% TODO: Replace XXXXX~~~ with the longest acronym.
\begin{acronym}[XXXXX~~~]\itemsep-0.25\baselineskip
  \acro{cv}[CV]{collective variable(s)}
  \acro{dof}[DOF]{degree(s) of freedom}
  \acro{pdf}[PDF]{probability density function}
\end{acronym}
