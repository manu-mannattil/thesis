%! TEX root = thesis.tex
% vim: ft=tex nospell et sts=2 sw=2

Following the usual convention in physics literature, vectors in $\mathbb{R}^{n}$ are set in bold Roman letters, e.g., $\bm{v}$.
Conventions for vectors in curvilinear coordinates vary.
Matrices (including matrix operators) are set in sans serif, e.g., $\mathsf{M}$.
Operators are distinguished with a hat, e.g., $\hat{D}$.
Integrals without explicit limits are to be integrated over the entire range (usually $-\infty$ to $\infty$) of the integration variable(s).
Indices are usually $i, j, k, l, m, n, \ldots$, unless there is a chance of confusion with imaginary number $i$, in which case we start from $j$.
And as usual, repeated indices are to be summed over.
Other notation conventions are listed below.
\vskip\baselineskip

\begin{tabular}{ll}
  $\Abs{\bm{a}}$ & norm $\sqrt{\bm{a}\trans\bm{a}}$ of vector $\bm{a} \in \mathbb{R}^{n}$\\
  $\mathsf{I}_n$ & $n\times n$ identity matrix\\
  $\nabla \phi$ & gradient of $\phi: \mathbb{R}^n \to \mathbb{R}$ considered as a row vector, or\\
                & the $m\times n$ Jacobian matrix of a map $\phi: \mathbb{R}^n \to \mathbb{R}^m$\\
  $(\nabla\phi)\trans$ & transpose gradient of $\phi: \mathbb{R}^n \to \mathbb{R}$ considered as a column vector, or\\
                & the $n\times m$ transpose of the Jacobian matrix of a map $\phi: \mathbb{R}^n \to \mathbb{R}^m$\\
  $\hess \phi$ & $n\times n$ Hessian matrix of $\phi: \mathbb{R}^n \to \mathbb{R}$\\
  $\det\mathsf{A}$ & determinant of the matrix $\mathsf{A}$\\
  $\mathscr{P}(\xi)$ & Marginal density of collective variable $\xi$\\
  $\mathscr{A}(\xi)$ & Free energy of collective variable $\xi$\\
  $k$, $l$ & Wave number\\
  $\eta$ & Poisson's ratio
\end{tabular}

\section*{Abbreviations}

\def\aclabelfont#1{\textsc{\MakeLowercase{#1}}}

% TODO: Replace XXXXX~~~ with the longest acronym.
\begin{acronym}[XXXXX~~~]\itemsep-0.25\baselineskip
  \acro{cas}[CAS]{computer algebra system(s)}
  \acro{cv}[CV]{collective variable(s)}
  \acro{dof}[DOF]{degree(s) of freedom}
  \acro{pdf}[PDF]{probability density function}
\end{acronym}
