% vim: ft=tex nospell et sts=2 sw=2
%! TEX root = thesis
% Common packages and preamble.

\usepackage[T1]{fontenc}
\usepackage{emem}

% BibTeX
\usepackage[numbers,sort&compress]{natbib}

% BibLaTeX
% \usepackage[english]{babel}
% \usepackage[backend=biber,style=phys,eprint=true,biblabel=brackets,pageranges=false,backref]{biblatex}
% \DefineBibliographyStrings{english}{
%   % Formatting of back references.
%   backrefpage = {p.~},
%   backrefpages = {pp.~},
% }
% \DeclareFieldFormat{titlecase}{#1} % titlecase book titles
% \addbibresource{library.bib}

\usepackage[smaller]{acronym}

%%% End notes ----------------------------------------------------------

% % memoir allows customization of the endnotes heading, but it doesn't
% % seem to add the heading to the TOC.  But I want to add customized
% % headings to the TOC, so we use a blank heading and define a \section*
% % the usual way.
% % Patch \printpagenotes to use a \section*.
% \let\oldprintpagenotes\printpagenotes
% \def\printpagenotes{%
%   \section*{Notes for Chapter \thechapter}
%   \addcontentsline{toc}{section}{Notes for Chapter \thechapter}
%   \oldprintpagenotes
% }
%
% \makepagenote                                           % enable endnotes
% \renewcommand*{\notesname}{Chapter Notes}               % endnotes heading
% \renewcommand*{\notedivision}{}                         % suppress endnote title
% \renewcommand*{\pagenotesubhead}[3]{}
%
% % Backlinks.  This messes up hanging indentation.
% % \notepageref
%
% % Hanging indent for endnotes.
% % https://tex.stackexchange.com/a/57515
% \renewcommand*\prenotetext{%
%   \list{}{%
%     \setlength\leftmargin{2em}%
%     \setlength\topsep{-\baselineskip}}
%   \item}
% \renewcommand*\postnotetext{\endlist\bigskip\medskip}
% \renewcommand*\idtextinnotes[1]{#1.\vspace*{-\baselineskip}}
% \renewcommand*\notenuminnotes[1]{#1.\vspace*{-\baselineskip}}

%%% Fonts --------------------------------------------------------------

% Fonts: Times (serif) + Optima (sans).
% Inspired by textbooks published by Pearson (e.g., 3rd edition of
% Goldstein's Classical Mechanics) and the Notre Dame Journal of Formal Logic.
% \usepackage{mathtime}
% \usepackage[scale=0.96]{classico}

% Inspired by Statistical Mechanics of Lattice Systems by Sacha Friedli and Yvan Velenik.
% https://www.unige.ch/math/folks/velenik/smbook/index.html
\usepackage[widespace]{fourier}
\usepackage{bm}
\usepackage[scale=0.92]{tgheros}
\DeclareMathAlphabet{\mathsf}{\encodingdefault}{\sfdefault}{\mddefault}{n}
\SetMathAlphabet{\mathsf}{bold}{\encodingdefault}{\sfdefault}{\bfdefault}{n}
\usepackage[scale=0.90]{roboto}

% https://narkive.com/eANktZj3.5
\makeatletter
\def\big{\bBigg@{1}}
\def\Big{\bBigg@{1.5}}
\def\bigg{\bBigg@{2.4}}
\def\Bigg{\bBigg@{3.2}}
\makeatother

%\renewcommand*{\backrefalt}[4]{\ifcase#1\else{Cited on pp.~#2}\fi}
%\renewcommand*{\backref}[1]{}
%\renewcommand*{\backrefalt}[4]{%
%  \ifcase#1%
%  \or{{\color{gray}Cited on p.~#2.}}%
%  \else{Cited on pp.~#2.}%
%  \fi}

\graphicspath{{figures/}}

\usepackage[allcolors=steel,colorlinks]{hyperref}
\urlstyle{same}
\RequirePackage[capitalize]{cleveref}

% \hypersetup{
%   pdfauthor={Manu Mannattil},
%   pdftitle={This Is a Really Long Title for My PhD Thesis},
%   pdfsubject={cond-mat.soft, cond-mat.stat-mech, cond-mat.other, physics.chem-ph, math.DG},
%   pdfkeywords={linkages, free energy},
% }

% Chapter-wise equation numbers.
\numberwithin{equation}{chapter}

%%% Commands -----------------------------------------------------------

% SU style toggle.
\newif\ifsustyle
\def\ifsu#1{\ifsustyle#1\fi}
\def\ifnotsu#1{\ifsustyle\else#1\fi}

% Command to typeset with more separation between characters and words.
% For reasons I don't understand, using \MakeUppercase doesn't work inside
% widetext, so you have to manual type in all caps (if you want that).
% Requires microtype for \textls: https://tex.stackexchange.com/q/23921
% Note also that this is not the same command as REVTeX's \widetext.
\def\widetext#1{\textls[120]{%
  \spaceskip=1.5\fontdimen2\font plus 1.5\fontdimen3\font minus 1.5\fontdimen4\font%
  {#1}%
}}

%%% Math commands ------------------------------------------------------

\def\free#1{\ensuremath\mathcal{A}_{\hat{#1}}(#1)}
\def\mpd#1{\ensuremath\mathcal{P}_{\hat{#1}}(#1)}
